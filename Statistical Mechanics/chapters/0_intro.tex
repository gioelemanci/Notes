\chapter*{Introduction}

Statistical Mechanics stands as one of the pillars of modern theoretical physics. It provides the formal bridge between the microscopic world—governed by the dynamical laws of Classical or Quantum Mechanics—and the macroscopic world described by Thermodynamics.

In this course, we move beyond the first phenomenological approach to thermodynamics. Our goal is to derive macroscopic observables (such as temperature, pressure, heat capacity, and magnetization) from the fundamental Hamiltonian describing a system of $N$ interacting particles, where $N$ is of the order of the Avogadro number ($N \sim 10^{23}$).

\subsection*{The Fundamental Problem}

Consider a macroscopic sample of gas or a magnetic solid. A complete dynamical description would require solving the equations of motion for every constituent particle. In a classical framework, this means tracking $6N$ variables in phase space $\{q_i(t), p_i(t)\}$. This task is not only computationally impossible but, more importantly, physically redundant.

We do not care about the exact position of every atom in a gas cylinder; we care about the pressure exerted on the walls and the temperature of the gas. The central insight of Statistical Mechanics is that we can trade \textit{exact dynamical knowledge} for \textit{probabilistic information}. By assuming that the system explores its accessible phase space according to specific probability distributions, we can predict macroscopic behavior with overwhelming accuracy.

\subsection*{The Thermodynamic Limit}

A recurring theme in these notes is the \textbf{Thermodynamic Limit}:
\[
    N \to \infty, \quad V \to \infty, \quad \frac{N}{V} = n = \text{const}
\]
Strictly speaking, sharp thermodynamic behavior—such as phase transitions and non-analyticities in the free energy—only emerges in this limit. While finite-size effects are important in mesoscopic physics, the foundational theory relies on $N$ being large enough that relative fluctuations, which typically scale as $1/\sqrt{N}$, become negligible.

\subsection*{Structure of the Notes}

These notes are organized to guide you from classical foundations to advanced quantum field theoretical methods applied to statistical systems.

\subsubsection{Part I: Classical Foundations (Chapters 1--5)}
We begin by reviewing the axiomatic structure of \textbf{Thermodynamics} (Ch. 1) and \textbf{Classical Mechanics} (Ch. 2), specifically Liouville's theorem and the concept of phase space flow. We then construct the three fundamental ensembles:
\begin{itemize}
    \item The \textbf{Microcanonical Ensemble} (Ch. 3) for isolated systems, defining entropy via the counting of microstates ($S = k_B \ln \Omega$).
    \item The \textbf{Canonical Ensemble} (Ch. 4) for systems in thermal equilibrium with a bath, introducing the partition function $Z$.
    \item The \textbf{Grand Canonical Ensemble} (Ch. 5) for systems exchanging both energy and particles, crucial for studying phase transitions and quantum gases.
\end{itemize}

\subsubsection{Part II: Interactions and Phase Transitions (Chapter 6)}
One of the most fascinating aspects of many-body physics is how simple microscopic interactions give rise to complex collective phenomena. We will study \textbf{Phase Transitions}, focusing on the breakdown of mean-field theory, the concept of Spontaneous Symmetry Breaking, and the universality of Critical Exponents. The Ising and Heisenberg models will serve as our primary playgrounds.

\subsubsection{Part III: Quantum Statistical Mechanics (Chapters 7--10)}
As we approach low temperatures, classical descriptions fail (e.g., the freezing out of degrees of freedom, the Gibbs paradox). We reformulate statistical mechanics using the \textbf{Density Matrix} formalism (Ch. 7).

A significant portion of this course is dedicated to \textbf{Second Quantization} (Ch. 8). We will move from wavefunctions to the formalism of creation and annihilation operators on Fock Space. This is the natural language for indistinguishable particles and forms the basis for modern condensed matter theory.

Finally, we apply these tools to \textbf{Quantum Gases} (Ch. 10). We will derive the Fermi-Dirac and Bose-Einstein distributions and study their low-temperature limits, culminating in the phenomenology of \textbf{Bose-Einstein Condensation (BEC)} and superfluidity—macroscopic quantum phenomena that have been experimentally realized in ultracold atomic gases.

\vspace{1cm}
\textbf{Prerequisites:} \textit{A solid understanding of Thermodynamics, Hamiltonian Mechanics, and basic Quantum Mechanics is assumed.}