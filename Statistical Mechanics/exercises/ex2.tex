\subsection*{Microcanonical Ensemble}

A totally isolated system, at temperature T and volume V, is described by the \textit{microcanonical ensemble}: both the energy $E$ and the number of particles $N$ are conserved.

We assume that all points on the hypersurface $S_{\mathcal{H}=E} \subset \mathcal{M}_{N}$, corresponding to the energy $E$, are equiprobable. Thus the (microcanonical) probability density reads:

\[\rho_{mc}(q_i,p_i) = \frac{\delta(\mathcal{H}-E)}{\omega(E)} \tag{1}\]

where $\omega(E) = \int_{S_E} dS_{\mathcal{H}}$ is the density of states, i.e. the area of the hypersurface $S_{\mathcal{H}=E}$.

Thermodynamics is recovered from the universal Boltzmann function for entropy:

\[S_{mc} \equiv k_B \log \omega(E)\]

which coincides with the thermodynamic entropy $S_{th}$, in the thermodynamic limit.

In the same limit, the entropy can be equivalently calculated as:

\[S_{mc} \equiv k_B \log \omega(E) = k_B \log \Gamma(E) = k_B \log \Sigma(E)\]

with $\omega(E) = \partial \Sigma / \partial E$ and $\Gamma(E) = \omega(E) \Delta E$.

\subsection*{Exercise 3 - Ideal gas in 3d, microcanonical ensemble}

In this exercise we will derive the thermodynamics of an ideal gas from its microcanonical description.

\begin{enumerate}
    \item Prove that:
    \[
    \log \Gamma(E) = \log \omega(E) + \log \Delta E = \log \Sigma(E) + \log(3N/(2E)) + \log \Delta E
    \]
    Divide the obtained expressions by the total number of particles $N$, to obtain the entropy per particle $s_{mc}$, showing that they all coincide, in the thermodynamic limit.   
    
    \item Starting from the result obtained in Ex. 2 of Set 1:
    \[
    \Sigma(E) = \frac{V 4\pi (2m)^{3/2}}{3h^3} E^{3/2}
    \]
    show that the entropy $S = k_B \log \Sigma(E)$, in the thermodynamic limit (i.e. neglecting terms that increase slower than $N$), is given by:
    \[
    S = k_B \left\{
    \begin{array}{ll}
    \frac{3}{2}N + N\log\left[V\left(\frac{4\pi mE}{3Nh^2}\right)^{3/2}\right] & \textit{distinguishable particles} \\
    \\
    \frac{5}{2}N + N\log\left[\frac{V}{N}\left(\frac{4\pi mE}{3Nh^2}\right)^{3/2}\right] & \textit{indistinguishable particles}
    \end{array}
    \right.
    \]
    \textit{Remark. You need to use Stirling formulas: $\log N! = N\log N - N$ for $N$ large and similarly $\log \Gamma(x) \simeq x\log x - x$ for $x$ large.}
    
    Are the obtained expressions of $S$ extensive? Discuss the result.
    
    \item Recalling that $TdS = dE + pdV$, calculate:
    \begin{itemize}
        \item $\left.\frac{\partial S}{\partial E}\right|_V$ and derive the expression for the internal energy:
        \[E = \frac{3}{2} N k_B T\]
        
        \item $\left.\frac{\partial S}{\partial V}\right|_E$ and derive the expression for the equation of state:
        \[pV = N k_B T\]
    \end{itemize}
    
\end{enumerate}

\vspace{2cm}
\textbf{Solution:}

\newpage

\subsection*{Canonical Ensemble}

The canonical ensemble describes a system, with a given volume V and a number of particles N, in thermodynamical equilibrium with an environment at a given temperature $T$, with which it can exchange energy.

The canonical probability density reads:

\[\rho_C = \frac{1}{Z_N} e^{-\beta \mathcal{H}(q_i,p_i)} \tag{2}\]

where $Z_N = Z_N[V,T]$ is the canonical partition function:

\[Z_N = \int_{\mathcal{M}_N} d\Omega\; e^{-\beta \mathcal{H}(q_i,p_i)} \tag{3}\]

Thermodynamic potentials can be derived through the formulas:

\[E = \langle \mathcal{H}(q_i,p_i) \rangle_c = -\frac{\partial \ln Z_N}{\partial \beta} \tag{4}\]

\[F = -\frac{1}{\beta} \ln Z_N \tag{5}\]

\[S = \frac{E - F}{T} \tag{6}\]

One can then obtain other thermodynamical quantities, such as:

\[p = -\left. \frac{\partial F}{\partial V} \right|_{T,N} \tag{7}\]

\[\mu = \left. \frac{\partial F}{\partial N} \right|_{T,V} \tag{8}\]

and

\[C_V = T \left. \frac{\partial S}{\partial T} \right|_{V,N} = \left. \frac{\partial E}{\partial T} \right|_{V,N} \tag{9}\]

\[C_p = T \left. \frac{\partial S}{\partial T} \right|_{p,N} = \left. \frac{\partial E}{\partial T} \right|_{p,N} + p \left. \frac{\partial V}{\partial T} \right|_{p,N} \tag{10}\]

\subsection*{Exercise 4 - Ideal gas in 3d, canonical ensemble}

In this exercise we will derive the thermodynamics of an ideal gas from its canonical description, supposing particles indistinguishable.

\begin{enumerate}
    \item Recalling that, for a gas of free non-relativistic particles, we have:
    \[
    \mathcal{M}_N = V^N \times \mathbb{R}^{3N}, \quad \mathcal{H} = \sum_i \frac{p_i^2}{2m}
    \]
    show that the canonical partition function $Z$ is given by:
    \[
    Z = \frac{V^N}{N! \lambda_T^{3N}}
    \]
    having defined the thermal wavelength
    \[
    \lambda_T \equiv \sqrt{\frac{h^2}{2\pi m k_B T}}
    \]
    Derive an expression for $\ln Z$ in the large $N$ limit.
    
    \textit{Remark. Recall Stirling formula and the formula for Gaussian integral:}
    \[
    \int_{-\infty}^{\infty} dx\, e^{-\alpha x^2} = \sqrt{\frac{\pi}{\alpha}}
    \]
    
    \item Derive the internal energy $E$ and the free energy $F$. 
    
    \item Derive the entropy $S$, for both the case of distinguishable and indistinguishable particles.
    
    Calculate the value $T_*$ of the temperature at which $S$ becomes negative. How can you interpret this result?
        
    \item Calculate the pressure $p$, and derive the equation of state.
    
    How does an isothermal curve look like in the $p-V$ plane?
     
    \item Calculate the chemical potential $\mu$ and draw a graph of it as function of temperature T, at a fixed density of particles $n$.
        
    \item Calculate the specific heats per particle: $c_v = C_v/N$ and $c_p = C_p/N$.

\end{enumerate}

\vspace{2cm}
\textbf{Solution:}

\newpage