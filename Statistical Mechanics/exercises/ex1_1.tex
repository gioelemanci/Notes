\section{Classical Statistical Mechanics}

\subsection*{Phase Space}

We work on a phase space $\mathcal{M}_{N}$ of $N$ particles in $d$ dimensions with coordinates$(q_{i},p_{i})$, $i=1,\cdots,N$. We will denote the volume element with:

\[
    d\Omega = \frac{\prod_{i}dq_{i}dp_{i}}{h^{dN}\,\xi_{N}} = \frac{d\mathcal{H}\,dS_{\mathcal{H}}}{h^{dN}\,\xi_{N}}
\]
where $\xi_{N}=N!$ if the particles are indistinguishable while $\xi_{N}=1$ if they are distinguishable.

We recall that the density of states $\omega(E)$ is given by
\[
    \omega(E) \equiv \int_{\mathcal{M}_{N}} \frac{\prod_{i}dq_{i}dp_{i}}{h^{dN}\,\xi_{N}} \delta(\mathcal{H}(q_{i},p_{i})-E) = \int_{S_{E}}\frac{dS_{\mathcal{H}}}{h^{dN}\,\xi_{N}} = \frac{\partial\Sigma(E)}{\partial E}
\]
where $\Sigma(E)$ is the volume of the region of states having energies between 0 and $E$:
\[
    \Sigma(E) = \frac{1}{h^{dN}\,\xi_{N}}\int_{0}^{E}d\mathcal{H}\int_{S_{\mathcal{H}}}dS_{\mathcal{H}}
\]

\subsection*{Exercise 1 - Density of States of a Single Particle}

\begin{enumerate}
    \item Describe the phase space $\mathcal{M}_{1}$, specify the Hamiltonian $\mathcal{H}$ for the following systems:
          \begin{enumerate}
              \item[(a)] A single non-relativistic particle in dimensions $d=1,2,3$, confined in an interval $L$, an area $A$ or a volume $V$ respectively;
              \item[(b)] A single harmonic oscillator in 1 dimension;
              \item[(c)] A single ultra-relativistic particle in 3 dimensions.
          \end{enumerate}

    \item Show that the density of states $\omega(E)$ is given by:

          Case (a)
          \[
              \omega(E) = \frac{L\sqrt{2m}}{h}E^{-1/2},\quad d=1
          \]
          \[
              \omega(E) = \frac{A\pi 2m}{h^{2}},\quad d=2
          \]
          \[
              \omega(E) = \frac{V2\pi(2m)^{3/2}}{h^{3}}E^{1/2},\quad d=3
          \]

          Case (b)
          \[
              \omega(E) = \frac{2\pi}{h\omega}
          \]

          Case (c)
          \[
              \omega(E) = \frac{4\pi V}{3c^{3}h^{3}}E^{3}
          \]
\end{enumerate}

\vspace{2cm}
\textbf{Solution:}

\textbf{1. Phase space and Hamiltonians}

In $d$ dimensions, each particle is described by $d$ coordinates $q_i$ and $d$ conjugate momenta $p_i$, so that ${\cal M}_N = \mathbb{R}^{dN} \times \mathbb{R}^{dN} = \mathbb{R}^{2dN}$. To avoid divergences, we confine the system in a finite volume $V$ (denoted with $L$ or $A$ in $d=1,2$), so that ${\cal M}_N = V^N \times \mathbb{R}^{dN}$.

Hamiltonians:

\begin{itemize}
    \item \textbf{Case (a)} Free non-relativistic particle:
          \[
              {\cal H} = \frac{p^2}{2m}
          \]
          where $p^2 \equiv (p^x)^2$ ($d=1$), $p^2 \equiv (p^x)^2 + (p^y)^2$ ($d=2$), etc.

    \item \textbf{Case (b)} Harmonic oscillator in 1 dimension:
          \[
              {\cal H} = \frac{p^2}{2m} + \frac{m\omega^2}{2} q^2
          \]
          with $q^2 \equiv (q^x)^2$, $p^2 \equiv (p^x)^2$.

    \item \textbf{Case (c)} Free ultra-relativistic particle in 3 dimensions:
          \[
              {\cal H} = c\, |p|
          \]
          where $|p| \equiv \sqrt{(p^x)^2 + (p^y)^2 + (p^z)^2}$.
\end{itemize}

\textbf{2. Density of states $\omega(E)$}

\textbf{Case (a) $d=1$:}
\[
    \Sigma(E) = \frac{1}{h} \int_L dq \int_{0 \leq |p|^2/2m \leq E} dp = \frac{L}{h} \int_{-\sqrt{2mE}}^{\sqrt{2mE}} dp^x = \frac{L 2\sqrt{2m}}{h} E^{1/2}
\]
\[
    \omega(E) = \frac{L\sqrt{2m}}{h} E^{-1/2}
\]

\textbf{Case (a) $d=2$:}
Using polar coordinates in $p$-space:
\[
    \Sigma(E) = \frac{1}{h^2} \int_A d^2q \int_{0 \leq |p|^2/2m \leq E} d^2p = \frac{A 2\pi}{h^2} \int_0^{\sqrt{2mE}} |p| d|p| = \frac{A \pi (2m)}{h^2} E
\]
\[
    \omega(E) = \frac{A \pi (2m)}{h^2}
\]

\textbf{Case (a) $d=3$:}
Using spherical coordinates in $p$-space:
\[
    \Sigma(E) = \frac{1}{h^3} \int_V d^3q \int_{0 \leq |p|^2/2m \leq E} d^3p = \frac{V 4\pi}{h^3} \int_0^{\sqrt{2mE}} |p|^2 d|p| = \frac{V 4\pi (2m)^{3/2}}{3h^3} E^{3/2}
\]
\[
    \omega(E) = \frac{V 2\pi (2m)^{3/2}}{h^3} E^{1/2}
\]

\textbf{Case (b):}
Using change of coordinates $x = \sqrt{\frac{m}{2}} q$, $y = \sqrt{\frac{1}{2m}} \omega p$:
\[
    \Sigma(E) = \frac{1}{h} \int_{0 \leq \frac{p^2}{2m} + \frac{m\omega^2}{2} q^2 \leq E} dq dp = \frac{2}{h\omega} \int_{0 \leq x^2 + y^2 \leq E} dx dy = \frac{2\pi}{h\omega} E
\]
\[
    \omega(E) = \frac{2\pi}{h\omega}
\]

\textbf{Case (c):}
Using spherical coordinates in $p$-space:
\[
    \Sigma(E) = \frac{1}{h^3} \int_V d^3q \int_{0 \leq c|p| \leq E} d^3p = \frac{V 4\pi}{h^3} \int_0^{E/c} |p|^2 d|p| = \frac{V 4\pi}{3c^3 h^3} E^3
\]
\[
    \omega(E) = \frac{V 4\pi}{c^3 h^3} E^2
\]

\newpage

\subsection*{Some Integrals}

In the following we will often use the Gamma function (or Euler function) which can be defined through an integral representation:
\[
    \Gamma(x) \equiv \int_{0}^{\infty} dt\, e^{-t} t^{x-1}
\]
It satisfies the following useful relations:
\[
    \Gamma(n+1) = n!, \quad \Gamma(x+1) = x\Gamma(x), \quad \Gamma(1/2) = \sqrt{\pi}
\]
In particular, this function is useful to calculate the volume of a sphere of radius $r$ in $\mathbb{R}^{m}$, i.e. of the region $0\leq\sum_{j=1}^{m}x_{j}^{2}\leq r^{2}$, which is given by:
\[
    V_{m} = \frac{\pi^{m/2}}{\Gamma\left(\frac{m}{2}+1\right)} r^{m}
\]

\subsection*{Exercise 2 - Density of States of N Particles}

\begin{enumerate}
    \item Describe the phase space $\mathcal{M}_{N}$, specify the Hamiltonian $\mathcal{H}$ for:
          \begin{enumerate}
              \item[(a)] A non-relativistic ideal gas in $d$ dimensions: i.e. a gas of non-interacting particles whose kinetic energy is non-relativistic;
              \item[(b)] A gas of harmonic oscillators in 1 dimension: i.e. a gas of non-interacting particles whose kinetic energy is non-relativistic and is subject to a harmonic potential with frequency $\omega$.
          \end{enumerate}

    \item Show that the density of states $\omega(E)$ is given by:

          Case (a)
          \[
              \omega(E) = \frac{V^{N}}{\xi_{N}\,\Gamma\left(dN/2\right)}\left(\frac{2\pi m}{h^{2}}\right)^{dN/2}E^{dN/2-1}
          \]

          Case (b)
          \[
              \omega(E) = \frac{1}{\xi_{N}\Gamma(dN)}\left(\frac{1}{\hbar\omega}\right)^{dN}E^{dN-1}
          \]
\end{enumerate}

\vspace{2cm}
\textbf{Solution:}

\textbf{1. Phase space and Hamiltonians}

See solution of Ex.1 point 1.

\textbf{2. Density of states $\omega(E)$}

\textbf{Case (a):}
Using spherical coordinates in momentum space and formula for volume of $m$-sphere:
\[
    \Sigma(E) = \frac{1}{h^{dN} \xi_N} \left( \prod_i \int_V d^d q_i \right) \int_{0 \leq \sum_i |p_i|^2 \leq 2mE} \left( \prod_i d^d p_i \right)
\]
\[
    = \frac{V^N}{\xi_N h^{dN}} \frac{1}{\Gamma(dN/2 + 1)} (2\pi m E)^{dN/2} = \frac{V^N}{\xi_N (dN/2) \Gamma(dN/2)} \left( \frac{2\pi m E}{h^2} \right)^{dN/2}
\]
\[
    \omega(E) = \frac{V^N}{\xi_N \Gamma(dN/2)} \left( \frac{2\pi m}{h^2} \right)^{dN/2} E^{dN/2 - 1}
\]

\textbf{Case (b):}
Using change of coordinates $p_j = \sqrt{2mE} x_j$ and $q_j = \sqrt{2E/m\omega^2} x_{dN+j}$ for $j=1,\cdots,dN$:
\[
    \Sigma(E) = \int_{0 \leq \sum_i \left( \frac{p_i^2}{2m} + \frac{m\omega^2}{2} q_i^2 \right) \leq E} \frac{\prod_i d^d q_i d^d p_i}{h^{dN} \xi_N}
\]
\[
    = \frac{1}{h^{dN} \xi_N} (2mE)^{dN/2} \left( \frac{2E}{m\omega^2} \right)^{dN/2} \int_{0 \leq \sum_{j=1}^{2dN} x_j^2 \leq 1} \left( \prod_{j=1}^{2dN} dx_j \right)
\]
\[
    = \frac{1}{\xi_N} \left( \frac{2E}{h\omega} \right)^{dN} \frac{\pi^{dN}}{\Gamma(dN + 1)} = \frac{1}{\xi_N (dN) \Gamma(dN)} \left( \frac{2\pi E}{h\omega} \right)^{dN}
\]
\[
    \omega(E) = \frac{1}{\xi_N \Gamma(dN)} \left( \frac{1}{\hbar\omega} \right)^{dN} E^{dN - 1}
\]

\newpage