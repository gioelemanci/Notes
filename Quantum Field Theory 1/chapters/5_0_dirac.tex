\chapter{Dirac theory}

The starting point is the lagrangian, then we show the most general solution to the equations of motion, then we proceed to quantize the system and promote the observables to operators on the Fock Space.
The lagrangian has to be Lorentz invariant.
Lorentz group admits a spinor representation, which for dirac translates In
\[
    \psi_D \to \dots
\]
and \(\Sigma_{\mu \nu} = \dots \) where
\[
    \gamma^{\mu \nu} \dots
\]
repects the clifford algebra \(\dots \) and are \(4\times 4\) complex matrices.
We adopt the Weyl (or chiral) representation of dirac matrices
\[
    \dots
\]
Sometimes the generators can be expressed As
\[
    S_{\mu \nu}\dots
\]
or by introducing \textbf{spinorial indices}:
\[
    \dots
\]
where we have Lorentz transformed the coordinates too.

\section{Action and Lagrangian}

We want to construct a lorentz invariant lagrangian dependent upon \(\psi\)
\[
    \psi^{\dagger} \dots
\]
The first attempt i to consider the spinor bilinear \(\psi^{\dagger} \psi\) as a building block; it's going to be a real number, but we have to check if it is Lorentz invariant.

\subsection{Lorentz scalar as a building block}

We have seen that
\[
    \psi \to  \psi^{\prime} = S \psi, \quad \psi^{\dagger} \to  \psi^{\dagger \prime} = S^{\dagger} \psi^{\dagger},
\]
and the product
\[
    \dots
\]
is not a lorentz scalar, it would require \(S^{\dagger} = S^{-1}\), which is not the case, since this representation is not unitary
\[
    \dots
\]
and therefore the spinor bilinear is not a good building block for our lagrangian. Let's check why \(S^{\dagger} \neq S^{-1}\):
\[
    \begin{aligned}
        S = \dots \quad \implies \begin{dcases}
                                     , ; \\
                                     .
                                 \end{dcases} \\
    \end{aligned}
\]
The generators \(S^{\mu \nu}\) are \textit{anti-hermitian}, while
\[
    \dots
\]
the generators \(\Sigma^{\mu \nu}\) are indeed hermitian
\[
    \dots
\]
and if \((\gamma^{\mu})^{\dagger} = \pm \gamma^{\mu}\) only if \(\iff\) all the gamma matrices are hermitian or anti-hermitian, so let's just check from the representations: we find out this is not possible since from the clifford algebra
\[
    \begin{aligned}
        \{\gamma^{\mu} ,\, \gamma^{\nu}\} = 2 \eta^{\mu \nu} \mathbb{I}_4 \implies \dots
        C^{-1} \gamma^0 C C^{-1} \gamma^0 C = \mathbb{I}_4 \implies (\gamma^0_{\text{diag}})^2 = \mathbb{I}_4
    \end{aligned}
\]
so \(\gamma^0\) is harmitian, with real eigenvalues \(\lambda_i^2 = 1\) while \(\gamma^i\) is antihermitian with imaginary eigenvalues \(\lambda_i^2 = -1\).

We then need to find another building block for our lagrangiuan.
Note that \((\gamma^{\mu})^{\dagger} = \gamma^0 \gamma^{\mu} \gamma^0\) \(\forall \mu\). Lets check it:
\[
    \begin{aligned}
        \mu = 0, \quad &   \\
        \mu = i, \quad & .
    \end{aligned}
\]
If we now look at \((S^{\mu \nu})^{\dagger}\)
\[
    \dots
\]
where we have expanded the commutator, recognized identities. So the right relation is \((S^{\mu \nu})^{\dagger} = \gamma^0 S^{\mu \nu} \gamma^0\). Remeber we are looking for the scalar building block of the lagrangian. Following this result we can write
\[
    S^{\dagger} = e^{\frac{1}{2}\omega_{\mu \nu}(S^{\mu \nu})^{\dagger}} = \dots
\]
Let's factor \(\gamma^0\) out again, manipulating the expression until
\[
    S^{\dagger} = \gamma^0 \left( \dots \right) \gamma^0 = \gamma^0 e^{\frac{1}{2}\omega_{\mu \nu}S^{\mu \nu}} \gamma^0 = \gamma^0 S^{\mu \nu} \gamma^0.
\]
This tells u that the right building block for our lagrangian is
\[
    \overline{\psi}(x)\psi(x),
\]
using the \textbf{adjoint Dirac Spinor} \(\overline{\psi}(x) = \psi^{\dagger} \gamma^0\). So in the end let's finally check the invarinace of this scalar:
\[
    \begin{aligned}
        \dots
    \end{aligned}
\]
it is indeed lorentz invariant. \(\overline{\psi} \psi\) is the lorentz scalar we will use to build our lagrangian.

But this is not enough. If we remember the KG Lagrangian
\[
    KG \mathcal{L}
\]
the analog of the last term would be \(m \overline{\psi}(x)\psi(x)\), while the first one, where the indices are contracted into a loremtz scalar, but that involves derivative. we have to look at anothe spinor bilinear for the kinetic term.

\subsection{Kinetic term}

Consider a different spinor bilinear
\[
    \overline{\psi}(x) \gamma^{\mu} \psi(x)
\]
which in the end is a vector of four components, so we can ask ourselves if that is a Lorentz vector or not, So that
\[
    \overline{\psi}^{\prime} \gamma^{\mu} \psi^{\prime} = \Lambda^{\mu}_{\nu} \overline{\psi}\gamma^{\nu} \psi
\]
If this is true, we can contract that \(\mu\) indices in order to obtain a lorentz scalar (invariant)
\[
    \overline{\psi} \gamma^{\mu}\partial_{\mu}\psi.
\]
So we found another building block, if true. Let's check it
\[
    \overline{\psi} \gamma^{\mu} \psi \to \overline{\psi}^{\prime} \gamma^{\mu} \psi^{\prime} = \overline{\psi} S^{-1}\gamma^{\mu} S \psi,
\]
So
\[
    \Lambda = e^{-\frac{i}{2} \dots } = \dots
\]
and in parallel
\[
    S = e^{\dots } = \dots
\]
Where we are analizyng \(4\times 4\) matrices, the first from the foundamental representation \(\mathrm{S=}(3,1)\), the second from the Dirac representation; also \(\Sigma^{\mu \nu}\) and \(M^{\mu \nu}\) are generators of lorentz transformation in foundamental representation. if we use the representation of \(M^{\mu \nu}\)
\[
    (\mathcal{M}^{\rho \sigma})^{\mu}_{\ \nu} \gamma^{\nu} = \dots
\]
Now we want some kind of relation among (??) since this last expression is the corresponding of \(\Lambda^{\mu}_{\ \nu} \overline{\psi} \gamma^{\nu} \psi\):
\[
    S^{\rho \sigma} = \frac{1}{4}\left[\gamma^{\rho},\, \gamma^{\sigma} \right] = \dots
\]
and
\[
    \left\{ \gamma^{\rho},\, \gamma^{\sigma} \right\} = 2 \eta^{\rho \sigma} \iff \dots
\]
Now we can compute the following commutator
\[
    [S^{\rho \sigma},\,\gamma^{\mu}] = \frac{1}{2} \dots
\]
So using the previous \(\gamma^{\sigma} \gamma^{\rho} = 2\eta^{\rho \sigma} - \gamma^{\rho} \gamma^{\sigma}\) we get to
\[
    [S^{\rho \sigma},\,\gamma^{\mu}] = \dots
\]
so we can take the previous computation and recognize
\[
    (\mathcal{M}^{\rho \sigma})^{\mu}_{\ \nu} \gamma^{\nu} = - [S^{\rho \sigma},\,\gamma^{\mu}].
\]
And now finally
\[
    \Lambda^{\mu}_{\ \nu} \gamma^{\nu} = \left[\delta^{\mu}_{\ \nu} + \frac{1}{2} \omega_{\rho \sigma} (\mathcal{M}^{\rho \sigma})^{\mu}_{\ \nu} + O(\omega_{\rho \sigma}^2)\right]\gamma^{\nu} = \gamma^{\mu} -\frac{1}{2} \omega_{\rho \sigma}[S^{\rho \sigma},\,\gamma^{\mu}] + O(\omega_{\rho \sigma}^2),
\]
but at an infinitesimal level it is identical to
\[
    S^{-1} \gamma^{\mu} S = \dots
\]
So we finally found
\[
    S^{-1} \gamma^{\mu} S = \Lambda^{\mu}_{\ \nu} \gamma^{\nu}.
\]
This implies that the Dirac bilinear \(\overline{\psi} \gamma^\mu \psi\) transforms as a Lorentz four vector since
\[
    \overline{\psi} \gamma^{\mu} \psi \to \overline{\psi}^{\prime} \gamma^{\mu} \psi^{\prime} = \overline{\psi} S^{-1} \gamma^{\mu} S \psi = \Lambda^{\mu}_{\ \nu} \overline{\psi} \gamma^{\mu} \psi.
\]

We can now obtain another Lorentz scalar by contracting this four vector with another one\dots
Examples:
\begin{itemize}
    \item \(\overline{\psi} \gamma^{\mu} \partial_\mu \psi\), good for kinetic terms;
    \item \(\overline{\psi} \gamma^{\mu}A_\mu \psi\), which is a contraction with a field in vectorial representation, this term represents interaction between spin \(\tfrac12\) and spin 1 gauge bosons (e.g. photons);
    \item \(\overline{\psi} \gamma^{\mu \nu} \psi\), where there is a Lorentz tensor, i.e. \(\overline{\psi}^{\prime} \gamma^{\mu \nu} \psi^{\prime} = \Lambda^{\mu}_{\ \rho} \Lambda^{\nu}_{\sigma} \overline{\psi} \gamma^{\rho \sigma} \psi\).
\end{itemize}
We can introduce the \textbf{slash notation} for indicating an object contracted with a gamma matrix.

We can now find an expression for the Lorentz invariant Dirac Lagrangian:
\[
    \mathcal{L} = \overline{\psi} i \gamma^{\mu} \partial_{\mu} \psi - m \overline{\psi} \psi = \overline{\psi} \left(i \gamma^{\mu} \partial_{\mu} - m\right) \psi.
\]
Comments
\begin{itemize}
    \item the \(i\) factor ensures the lagrangian to be real: we have to check the realness of
          \[
              (\overline{\psi} \psi)^{\dagger} = \dots = \overline{\psi} \psi.
          \]
          \[
              (\overline{\psi} \gamma^{\mu} \partial_{\mu} \psi)^{\dagger} = \dots = - \overline{\psi} \gamma^{\mu} \partial_{\mu} \psi.
          \]
          That's why we need the imaginary unit.\footnote{Since this lagrangian is integrated to obtain the Lagrangian, we can always use the trick of integrating by parts and then neglect the boundary term.}
    \item Some unit analysis in mass units:
          \[
              \begin{aligned}
                  [S] = 0, \quad [\mathrm{d}^4 x]=-4 \quad \implies [\mathcal{L}] = 4, \\
                  [\partial_\mu] = 1,\quad [m]= 1 \quad \implies [\psi]=[\overline{\psi}] = \frac{3}{2}.
              \end{aligned}
          \]
          So this field has a different mass dimention from KG scalar field, which had: \([\psi]=1\).
    \item KG Lagrangian contains two derivatives \(\partial_\mu \psi \partial^{\mu} \psi\), while Dirac's only one: The KG lagrangian was of the second order, while Dirac's of the first; it changes the order of the equations of motion.
    \item Upon quantization the Dirac theory will describe particles/antiparticles (for whose we needed a complex field for the charge dof, which is our case) with spin \(\tfrac12\) and mass \(m\); in principle we have four complex dof, which are LH/RH and spin up/down (doubled because 4 for particle and 4 for antiparticle real dof).
\end{itemize}




------



\section{Dirac Equation}

The Dirac Lagrangian is
\[
    \dots
\]
where \(\overline{\psi} = \psi^{\dagger} \gamma^0\)  is the Dirac conjugate. We can find the equations of motion using the Euler-Lagrange equations for fields:
\[
    \dots
\]
notice that \(\frac{\partial}{\partial \overline{\psi}} \mathcal{L} = 0\), in fact \(\dots \) and so \(\dots \) too. Meanwhile the derivatives with respect to \(\psi\) are
\[
    \dots
\]
Hence we get the Dirac equation
\[
    \dots
\]
where the arrow on the derivatives tells us it is to be applied to the left. This is a first order partial differential equation for the spinor field \(\psi(x)\).

We can build such an equation of motion just thatks to the presence of \(\gamma^{\mu}\) which grants lorentz invariance. Dirac equation si a first order differential equation, while for KG we could only get a second order one. Furthermore KG is a scalar equation, while Dirac is a spinor equation (vectorial in spinor space, with 4 components).

Notice that dirac equation mixes components of the spinor, since \(\gamma^{\mu}\) are \(4\times 4\) matrices:
\[
    \psi = \begin{pmatrix}
        \psi_1 \\
        \psi_2 \\
        \psi_3 \\
        \psi_4
    \end{pmatrix}, \quad \gamma^0 = \begin{pmatrix}
        0            & \mathbb{I}_2 \\
        \mathbb{I}_2 & 0
    \end{pmatrix}, \quad \gamma^i = \begin{pmatrix}
        0         & \sigma^i \\
        -\sigma^i & 0
    \end{pmatrix}.
\]
So in components in the spinor space, Dirac equations reads:
\[
    i \begin{pmatrix}
        \partial_t \psi_3 \\
        \partial_t \psi_4 \\
        \partial_t \psi_1 \\
        \partial_t \psi_2
    \end{pmatrix} + i \begin{pmatrix}
        \partial_x \psi_4  \\
        \partial_x \psi_3  \\
        -\partial_x \psi_2 \\
        -\partial_x \psi_1
    \end{pmatrix} \dots
\]
in general it becomes a system of four coupled first order differential equations, mixing the four components of the spinor.

However each component \(\psi_{\alpha}(x)\) satosfy the KG equation, since in effect each components describe a dof of a relativistic particle with mass \(m\): if we multiply Dirac equation by \((i \gamma^{\mu} \partial_{\mu} + m)\) from the left (it is still valid since zero multiplied by anything remains xero), we get
\[
    \begin{aligned}
        (i \gamma^{\mu} \partial_{\mu} + m)(i \gamma^{\nu} \partial_{\nu} - m) \psi(x) = 0                                \\
        \implies \left(-\gamma^{\mu} \gamma^{\nu} \partial_{\mu} \partial_{\nu} - m^2\right) \psi(x) = 0                  \\
        \implies \left(-\frac{1}{2}\{\gamma^{\mu},\,\gamma^{\nu}\} \partial_{\mu} \partial_{\nu} - m^2\right) \psi(x) = 0 \\
        \implies \left(-\eta^{\mu \nu} \partial_{\mu} \partial_{\nu} - m^2\right) \psi(x) = 0                             \\
        \implies (\Box + m^2) \psi(x) = 0.
    \end{aligned}
\]
In terms of matrices we have used the clifford algebra to simplify the product of gamma matrices. So each component of the dirac spinor satisfies the KG equation, thus \((\Box + m^2) \psi_{\alpha} (x) = 0\) \(\forall \alpha = 1,\,2,\,3,\,4\).

\subsection{Chiral Spinors}

Chirality means that Dirac representation \((\tfrac12,\,0) \oplus (0,\,\tfrac12)\) can be decomposed into two irreducible representations, \textbf{Weyl representation}, of the Lorentz group. Weyl or Chiral spinors are two-component objects (complex dof) with different transformation properties:
\begin{itemize}
    \item \textbf{Left-Handed} weyl spinors: \(\psi_L \sim (\tfrac12,\,0)\), transform under \(S_L\) only;
    \item \textbf{Right-Handed} weyl spinors: \(\psi_R \sim (0,\,\tfrac12)\), transform under \(S_R\) only.
\end{itemize}

We can write the Dirac spinor as a combination of two weyl spinors:
\[
    \psi_D = \begin{pmatrix}
        \psi_L^{(w)} \\
        \psi_R^{(w)}
    \end{pmatrix}, \quad \psi_L^{(w)} \xleftrightarrow{\text{Parity}} \psi_R^{(w)}.
\]

\paragraph{Chirality operator.} In order to project out the two chiral components from a Dirac spinor we can introduce the chirality operator
\[
    \gamma^5 = \dots
\]
using which we can define projectors able to select the desired chiral component from the Dirac spinor:
\[
    \begin{aligned}
        \psi_L = \\
        \psi_R =.
    \end{aligned}
\]
Thus the dirac spinor is the sum of the two chiral components \(\psi_D = \psi_L + \psi_R\) and the projectors satisfy the usual properties:
\[
    \begin{dcases}
        P_L^2 = P_L \\
        P_L^{\dagger} = P_L
    \end{dcases} \quad \dots
\]
Note that the eigenvalues of \(\gamma^5\) are \(\pm 1\), so the chirality operator measures the chirality of a spinor:
\[
    \dots
\]
We will see how chirality is related to helicity in the massless limit.

\paragraph{Lagrangian and Chirality.}To understand better, let's write the dirac lagrangian in terms of chiral components (which is very useful for statistical field theory and the standard model, also for understanding the massless limit):
\begin{enumerate}
    \item aa
          \[
              \begin{aligned}
                  \overline{\psi} \gamma^{\mu} \psi = \psi^{\dagger} \gamma^0 \gamma^{\mu} \left(\psi_L + \psi_R\right) = \dots
              \end{aligned}
          \]

          Now exploiting hermitianity of the gamma matrices we can write
          \[
              \overline{\psi} \gamma^{\mu} \psi = \overline{\psi}_R \gamma^{\mu} \psi_R + \overline{\psi}_L \gamma^{\mu} \psi_L.
          \]
          This found vector current does not mix chiral components, its parity invariant; this is not good for electroweak interactions and interpretation \(\mathrm{SU}(2)_L\).

    \item aa
          \[
              \overline{\psi} \gamma^{\mu} \gamma^{5} \psi = \dots
          \]
          same steps as before, but now we find another current, called \textbf{axial vector current}: it changes sign under parity and not good again for EW interpretation.

    \item aa
          \[
              \frac{1}{2} (V-A) = \overline{\psi} \dots
          \]
          V-A current violates parity, and only LH components enter the interarction: it is a good candidate for EW interactions.

    \item mass term
          \[
              \overline{\psi} \psi 0 = \psi^{\dagger} \gamma^0 (P_R^2 + P_L^2) \psi = \dots =
          \]
          with respect to before we dont have \(\gamma^{\mu}\), so we have to put a minus while writing in terms of chiral components: mass term mixes LH and RH components:
          \[
              \dots
          \]
\end{enumerate}

Now we have everything to write the dirac lagrangian in terms of chiral components:
\[
    \mathcal{L} = \overline{\psi}_L i \gamma^{\mu} \partial_{\mu} \psi_L + \overline{\psi}_R i \gamma^{\mu} \partial_{\mu} \psi_R - m \left(\overline{\psi}_L \psi_R + \overline{\psi}_R \psi_L\right).
\]
We can see that in the massless limit the two chiral components decouple, and we get two independent Weyl equations for each chiral component: this is important, since kinetic terms evolves independently the two chiral components, while the mass term allow us to perform a boost and mix the two chiralities: since the particle is massive there will be frames where the particle is RH and others where it is seen as LH.

If you take an electron whith both the chiral components, only the LH component will interact weakly, while the RH will not; but since the electron is massive you can always boost to a frame where the electron appears as RH, so both components are needed to describe a massive fermion. When we see only the RH component, the electron will not seem to interact weakly.

\paragraph{Dirac in Weyl components.}
If we write the Dirac equation in terms of chiral components we get:
\[
    \dots
\]
which is a system of two coupled first order coupled differential equations
\[
    \begin{dcases}
        \dots \\
        \dots \\
    \end{dcases}
\]
In the massless limit the two equations decouple and we get two Weyl equations:
\[
    \begin{dcases}
        \dots \\
        \dots \\
    \end{dcases}
\]
and we call this last set of equations the \textbf{Weyl equations} for massless fermions.
Now it is clearer the meaning of Left and Right-Handed Weyl spinors: in terms of operators we have
\[
    i \partial_t = \hat{H},\quad -i \nabla = \hat{\mathbf{p}}, \quad \bs{\sigma} = \hat{\mathbf{S}},
\]
so that we can see the last two equations as
\[
    \dots
\]
Now, dropping the hat notation for operators (so that we can use it to indicate versors), we find clearly that the weyl components are eigenstate of the \textbf{helicity}:
\[
    \begin{dcases}
        \dots \\
        \dots
    \end{dcases}
\]
where helicity operator is defined as \(\mathbf{S} \cdot \mathbf{p}\); since the eigenvalues are \(\pm 1\), we can affirm that in the massless limit chirality and helicity coincide: they have same eigenstates and eigenvalues.

If the neutrinos were massless, only the LH component would exist, since only that interacts weakly; but since neutrinos have a small mass, both chiral components exist, even if the RH component has never been observed (it interacts only gravitationally, so it is very difficult to detect it).

We now want to prove that under parity we can pass from one weyl component to the other:
\[
    \psi_L^{(w)} \xleftrightarrow{\text{Parity}} \psi_R^{(w)}.
\]
Let's start from the Dirac equation:
\[
    \begin{aligned}
        \dots
    \end{aligned}
\]
which, given the action of the parity:
\[
    \dots
\]
then transforms the Dirac equation as
\[
    \dots
\]
But how does \(\psi\) transform under parity?
\[
    \dots
\]
but gamma matrices satisfy
\[
    \dots
\]
so we gave
\[
    \dots
\]
[...] sofi [...]

and we find finally that
\[
    \psi^{\prime} = \begin{pmatrix}
        \dots
    \end{pmatrix} = \begin{pmatrix}
        0            & \mathbb{I}_2 \\
        \mathbb{I}_2 & 0
    \end{pmatrix} \begin{pmatrix}
        \psi_L^{(w)} \\
        \psi_R^{(w)}
    \end{pmatrix},
\]
so that
\[
    \begin{dcases}
        \psi_L^{\prime (w)} = \psi_R^{(w)}, \\
        \psi_R^{\prime (w)} = \psi_L^{(w)}.
    \end{dcases}
\]

------

\subsection{Solutions of the Dirac Equation}

Each component of the dirac spinor \(\psi(x)\) satisfies the KG equation, so we can look for plane wave solutions of the form
\[
    \psi_{\alpha}(x) = u_{\alpha}(\mathbf{p}) e^{-i p_\mu x^{\mu}} = u_{\alpha}(\mathbf{p}) e^{-i (E_{\mathbf{p}} t - \mathbf{p} \cdot \mathbf{x})},
\]
where \(\alpha = 1,\,2,\,3,\,4\) and \(u_{\alpha}(\mathbf{p})\) are complex coefficients depending on the momentum \(\mathbf{p}\) (it's a 4-component vector). Plugging this ansatz into the Dirac equation we get
\[
    (i \gamma^{\mu} \partial_{\mu} - m) \psi(x) = 0, \quad \implies \quad (\gamma^{\mu} p_{\mu} - m) u(\mathbf{p}) = 0.
\]
which in spinorial representation reads
\[
    \left[ \begin{pmatrix}
            0 & 1 \\
            1 & 0
        \end{pmatrix} p_0 + \begin{pmatrix}
            0         & \sigma^i \\
            -\sigma^i & 0
        \end{pmatrix} p_i - m \begin{pmatrix}
            1 & 0 \\
            0 & 1
        \end{pmatrix}\right] u(\mathbf{p}) = \begin{pmatrix}
        -m                         & p^{\mu} \sigma_{\mu} \\
        p^{\mu} \bar{\sigma}_{\mu} & -m
    \end{pmatrix} u(\mathbf{p}) = 0,
\]
where we have defined \(\sigma^{\mu} = (1,\,\sigma^i)\) and \(\bar{\sigma}^{\mu} = (1,\,-\sigma^i)\). Writing the spinor \(u(\mathbf{p})\) in terms of its two-component chiral parts
\[
    u(\mathbf{p}) = \begin{pmatrix}
        u_L(\mathbf{p}) \\
        u_R(\mathbf{p})
    \end{pmatrix},
\]
we get the system of equations
\[
    \begin{dcases}
        (p^{\mu} \sigma_{\mu}) u_R(\mathbf{p}) = m u_L(\mathbf{p}), \\
        (p^{\mu} \bar{\sigma}_{\mu}) u_L(\mathbf{p}) = m u_R(\mathbf{p}) .
    \end{dcases}
\]
It's a system of two coupled equations for the two chiral components of the spinor. We can solve for one component in terms of the other; for example, solving for \(u_R(\mathbf{p})\) from the second equation and plugging it into the first, we get
\[
    (p^{\mu} \sigma_{\mu})(p^{\nu} \bar{\sigma}_{\nu}) = (p_0 + p_i \sigma^i)(p_0 - p_j \sigma^j) = p_0^2 - - p_i p_j \sigma^i \sigma^j = p_0^2 - \mathbf{p}^2 = m^2,
\]
where we have used the clifford algebra of the Pauli matrices \(\sigma^i \sigma^j = \delta^{ij} + i \epsilon^{ijk} \sigma^k\) and the relativistic dispersion relation \(p_0^2 - \mathbf{p}^2 = m^2\). Thus we can write
\[
    \begin{aligned}
        u_L(\mathbf{p})                             & = A (p^{\mu} \sigma_{\mu}) \chi,                             \\
        (p^{\mu} \bar{\sigma}_{\mu}) u_L(\mathbf{p} & = (p^{\mu} \bar{\sigma}_{\mu}) (p^{\mu} \sigma_{\mu}) A \chi \\
                                                    & = m^2 A \chi = m u_R(\mathbf{p})                             \\
        \implies u_R(\mathbf{p})                    & = m A \chi,
    \end{aligned}
\]
and similarly for \(u_L(\mathbf{p})\):
\[
    \begin{aligned}
        u_R(\mathbf{p})                       & = B (p^{\mu} \bar{\sigma}_{\mu}) \xi,                       \\
        (p^{\mu} \sigma_{\mu}) u_R(\mathbf{p} & = (p^{\mu} \sigma_{\mu}) (p^{\mu} \bar{\sigma}_{\mu}) B \xi \\
                                              & = m^2 B \xi = m u_L(\mathbf{p})                             \\
        \implies u_L(\mathbf{p})              & = m B \xi,
    \end{aligned}
\]
where \(\chi\) and \(\xi\) are arbitrary two-component spinors and \(A\), \(B\) are normalization constants. Thus the general solution for \(u(\mathbf{p})\) can be written as
\[
    u(\mathbf{p}) = A \begin{pmatrix}
        (p^{\mu} \sigma_{\mu}) \chi \\
        m \chi
    \end{pmatrix},
\]
where now we want to symmetrize the solution by choosing \(A = \frac{1}{m}\) and \(\chi = \sqrt{p^{\mu} \bar{\sigma}_{\mu}} \xi\), where \(\xi\) is a constant two-component spinor respecting
\[
    \xi^{\dagger} \xi = 1.
\]
Thus we get the final expression for the \textbf{positive frequency solution} of the Dirac equation:
\[
    u_L(\mathbf{p}) = \dots
\]
\[
    u_R (\mathbf{p}) = \dots
\]
\[
    \psi(x) = \dots
\]

We can also find \textbf{negative frequency solutions} of the Dirac equation by considering
\[
    \psi(x) = v(\mathbf{p}) e^{i p_{\mu} x^{\mu}},
\]
where \(v(\mathbf{p})\) satisfies
\[
    (\gamma^{\mu} p_{\mu} + m) v(\mathbf{p}) = \begin{pmatrix}
        m                          & p^{\mu} \sigma_{\mu} \\
        p^{\mu} \bar{\sigma}_{\mu} & m
    \end{pmatrix} v(\mathbf{p}) = 0,
\]
and following the same steps as before we get
\[
    v(\mathbf{p}) = C \begin{pmatrix}
        \sqrt{p^{\mu} \sigma_{\mu}} \eta \\
        -\sqrt{p^{\mu} \bar{\sigma}_{\mu}} \eta
    \end{pmatrix}.
\]

If we were to apply the Hamiltonian on these ansatzs we would get
\[
    \hat{H} \psi(x) = i \partial_t (u(\mathbf{p}) e^{-i p_{\mu} x^{\mu}}) =,
\]
\[
    \hat{H} \psi(x) = i \partial_t (v(\mathbf{p}) e^{i p_{\mu} x^{\mu}}) =.
\]
Thus \(u(\mathbf{p})\) are positive energy solutions while \(v(\mathbf{p})\) are negative energy solutions of the Dirac equation.

\begin{example}
    In the rest frame of a massive particle
    \[
        \mathbf{p} = 0, \quad E_{\mathbf{p}} = p^0 = m.
    \]
    Thus the positive frequency solutions read
    \[
        u(\mathbf{0}) = \begin{pmatrix}
            \sqrt{p^{\mu} \sigma_\mu} \xi \\
            \sqrt{p^{\mu} \bar{\sigma}_\mu} \xi
        \end{pmatrix} = \sqrt{m} \begin{pmatrix}
            \xi \\
            \xi
        \end{pmatrix},
    \],
    so that the plane wave solution is
    \[
        \psi(x) = \sqrt{m} \begin{pmatrix}
            \xi \\
            \xi
        \end{pmatrix} e^{-i m t}.
    \]

    Now recalling the Lorentz transformation for spinors
    \[
        \psi^{\prime \, \alpha}(x^{\prime}) = S(\Lambda)^{\alpha}_{\beta} \psi^{\beta}(x), \quad S(\Lambda)^{\alpha}_{\beta} = (e^{-\frac{i}{2} \omega_{\mu \nu} S^{\mu \nu}})^{\alpha}_{\beta},
    \]
    where the algebra generators of the Lorentz transformation are
    \[
        S^{\mu \nu} = \frac{i}{4} [\gamma^{\mu},\, \gamma^{\nu}]
    \]
    with parameters \(\omega_{\mu \nu}\) depending on the boost/rotation we are performing. For a pure spatial rotation we have
    \[
        S^{ij} = -\frac{i}{2} \begin{pmatrix}
            \sigma^k & 0        \\
            0        & \sigma^k
        \end{pmatrix}, \quad (i \neq j)
    \]
    where \(k\) is the axis of rotation, and the three parameters
    \[
        \omega_{ij} = - \epsilon_{ijk} \theta^k, \quad (i \neq j)
    \]
    are related to the rotation angles around the three spatial axes. Thus the spinor transformation under a spatial rotation reads
    \[
        e^{\frac{1}{2} \omega_{ij} S^{ij}} = \begin{pmatrix}
            e^{i \frac{\theta^k}{2} \sigma^k} & 0                                 \\
            0                                 & e^{i \frac{\theta^k}{2} \sigma^k}
        \end{pmatrix},
    \]
    and if we apply this transformation to the rest-frame spinor we get
    \[
        \psi(x) = \sqrt{m} \begin{pmatrix}
            \xi \\
            \xi
        \end{pmatrix} e^{-i m t} \to  \psi^{\prime}(x^{\prime}) = \begin{pmatrix}
            e^{i \frac{\theta^k}{2} \sigma^k} & 0                                 \\
            0                                 & e^{i \frac{\theta^k}{2} \sigma^k}
        \end{pmatrix} \psi(x) = \sqrt{m} \begin{pmatrix}
            e^{i \frac{\theta^k}{2} \sigma^k} \xi \\
            e^{i \frac{\theta^k}{2} \sigma^k} \xi
        \end{pmatrix} e^{-i m t},
    \]
    so that both chiral components transform in the same way under spatial rotations, as expected since in the rest frame chirality and helicity are not defined
    \[
        \xi \to \xi^{\prime} = e^{i \frac{\theta^k}{2} \sigma^k} \xi.
    \]
    This is the representation of the standard \(\mathrm{SO}(2)\) transformation for spin \(\tfrac12\) objects
    \[
        \mathbf{S} = \frac{1}{2} \hbar \bs{\sigma}.
    \]
    We are considering particles at rest with spin \(\tfrac12\), so we can choose the basis where the spin is aligned along the \(z\)-axis:
    \[
        \xi_+ = \begin{pmatrix}
            1 \\
            0
        \end{pmatrix}, \quad \xi_- = \begin{pmatrix}
            0 \\
            1
        \end{pmatrix},
    \]
    which are eigenstates of the spin operator \(S_z = \frac{\hbar}{2} \sigma^3\) with eigenvalues \(\pm \frac{\hbar}{2}\).
\end{example}

\begin{example}
    Now consider a particle with spin up along the \(z\)-axis moving with momentum \(\mathbf{p} = (0,\,0,\,p_z)= (0,\,0,\,p)\) along the \(z\)-axis
    \[
        p^{\mu} = (E_{\mathbf{p}},\,0,\,0,\,p_z), \quad E_{\mathbf{p}} = \sqrt{p_z^2 + m^2} = E.
    \]
    We have the positiv frequency solution as
    \[
        \psi(x) = \begin{pmatrix}
            \sqrt{p^{\mu} \sigma_{\mu}} \xi_+ \\
            \sqrt{p^{\mu} \bar{\sigma}_{\mu}} \xi_+
        \end{pmatrix}e^{-i p_{\mu} x^{\mu}},
    \]
    where computing the square roots we get
    \[
        \psi(x) = \begin{pmatrix}
            \begin{pmatrix}
                \sqrt{E + p} & 0            \\
                0            & \sqrt{E - p}
            \end{pmatrix}\begin{pmatrix}
                             1 \\
                             0
                         \end{pmatrix} \\
            \begin{pmatrix}
                \sqrt{E - p} & 0            \\
                0            & \sqrt{E + p}
            \end{pmatrix}\begin{pmatrix}
                             1 \\
                             0
                         \end{pmatrix}
        \end{pmatrix} e^{-i (E t - p z)} = \begin{pmatrix}
            \sqrt{E + p} \\
            0            \\
            \sqrt{E - p} \\
            0
        \end{pmatrix} e^{-i (E t - p z)}.
    \]
    where we have a right handed helicity state since both momentum and spin are aligned along the \(z\)-axis, but both chiral components since for massive particles chirality and helicity do not coincide. Now we can consider the massless limit \(m \to 0\), so that \(E = p\) and we get that the left chiral component vanishes
    \[
        \psi(x) = \begin{pmatrix}
            \sqrt{2 p} \\
            0          \\
            0          \\
            0
        \end{pmatrix} e^{-i p (t - z)},
    \]
    which is a purely right-handed spinor, as expected since in the massless limit chirality and helicity coincide:
    \[
        \begin{aligned}
            \psi_L (x) & = 0,                        \\
            \psi_R (x) & = \sqrt{2 p} \begin{pmatrix}
                                          1 \\
                                          0
                                      \end{pmatrix}.
        \end{aligned}
    \]
\end{example}

\subsection{Useful Formulae}

We want to introduce some useful formulae for manipulating Dirac spinors and gamma matrices, in order to comprehend better the structure of the quantum theory.

\paragraph{Inner product.}
We have to define a basis \(\xi^n\) and \(\eta^n\) for the two-component spinors used to build the positive and negative frequency solutions of the Dirac equation (with \(n = 1,\,2\)):
\[
    \xi^{n \, \dagger} \xi^m = \delta^{nm}, \quad \eta^{n \, \dagger} \eta^m = \delta^{nm},
\]
with trivial example
\[
    \xi^1 = \begin{pmatrix}
        1 \\
        0
    \end{pmatrix}, \quad \xi^2 = \begin{pmatrix}
        0 \\
        1
    \end{pmatrix}.
\]
Thus solutions of the Dirac equation can be labeled with this index and distinguished into different spin states:
\[
    u^{n\,\dagger}(\mathbf{p}) u^m(\mathbf{p}) = \begin{pmatrix}
        \xi^{n\, \dagger} \sqrt{p^{\mu} \sigma_\mu} & \xi^{n\, \dagger} \sqrt{p^{\mu} \bar{\sigma}_\mu}
    \end{pmatrix} \begin{pmatrix}
        \sqrt{p^{\mu} \sigma_\mu} \xi^m \\
        \sqrt{p^{\mu} \bar{\sigma}_\mu} \xi^m
    \end{pmatrix}
\]
which can be computed as
\[
    \begin{aligned}
        u^{n\,\dagger}(\mathbf{p}) u^m(\mathbf{p}) & = \xi^{n\, \dagger} (p^{\mu} \sigma_\mu) \xi^m + \xi^{n\, \dagger} (p^{\mu} \bar{\sigma}_\mu) \xi^m \\
                                                   & = \dots = 2 p_0 \xi^{n\, \dagger} \xi^m = 2 E_{\mathbf{p}} \delta^{nm}.
    \end{aligned}
\]
Instead if we have
\[
    \overline{u}^{n}(\mathbf{p}) u^m(\mathbf{p}) = \begin{pmatrix}
        \xi^{n\, \dagger} \sqrt{p^{\mu} \sigma_\mu} & \xi^{n\, \dagger} \sqrt{p^{\mu} \bar{\sigma}_\mu}
    \end{pmatrix} \begin{pmatrix}
        0 & 1 \\
        1 & 0
    \end{pmatrix} \begin{pmatrix}
        \sqrt{p^{\mu} \sigma_\mu} \xi^m \\
        \sqrt{p^{\mu} \bar{\sigma}_\mu} \xi^m
    \end{pmatrix},
\]
and we get (\(\overline{u}^n = u^{n\,\dagger} \gamma^0\))
\[
    \begin{aligned}
        u^{n\,\dagger}(\mathbf{p}) \gamma^0 u^m(\mathbf{p}) & = \xi^{n\, \dagger} (p^{\mu} \sigma_\mu) \xi^m + \xi^{n\, \dagger} (p^{\mu} \bar{\sigma}_\mu) \xi^m \\
                                                            & = \dots = 2 m \xi^{n\, \dagger} \xi^m = 2 m \delta^{nm}.
    \end{aligned}
\]
Now we can summarize these two results as
\[
    \begin{dcases}
        u^{n\,\dagger}(\mathbf{p}) u^m(\mathbf{p}) = 2 E_{\mathbf{p}} \delta^{nm}, \\
        \overline{u}^n (\mathbf{p}) u^m(\mathbf{p}) = 2 m \delta^{nm}.
    \end{dcases}
\]
Similarly for the negative frequency solutions we have\TODO{Compute these explicitly.}
\[
    \begin{dcases}
        v^{n\,\dagger}(\mathbf{p}) v^m(\mathbf{p}) = 2 E_{\mathbf{p}} \delta^{nm}, \\
        \overline{v}^n (\mathbf{p}) v^m(\mathbf{p}) = -2 m \delta^{nm}.
    \end{dcases}
\]
Now we can compute the mixed products
\[
    \overline{u}^n (\mathbf{p}) v^m(\mathbf{p}) = \dots = 0,
\]
and similarly
\[
    \overline{v}^n (\mathbf{p}) u^m(\mathbf{p}) = \dots = 0.
\]
Thus we have the orthogonality relations, which do not mix positive and negative frequency solutions.
We have also
\[
    u^{n\,\dagger}(\mathbf{p}) v^m(-\mathbf{p}) = \dots = 0,
\]
and similarly
\[
    v^{n\,\dagger}(\mathbf{p}) u^m(-\mathbf{p}) = \dots = 0.
\]

\paragraph{Outer products.}
We can compute the outer products of the spinors, starting from the positive frequency solutions:
\[
    \sum_{n} u^n(\mathbf{p}) \overline{u}^n(\mathbf{p}) = \sum_n \begin{pmatrix}
        \sqrt{p^{\mu} \sigma_\mu} \xi^n \\
        \sqrt{p^{\mu} \bar{\sigma}_\mu} \xi^n
    \end{pmatrix} \begin{pmatrix}
        \xi^{n\, \dagger} \sqrt{p^{\mu} \sigma_\mu} & \xi^{n\, \dagger} \sqrt{p^{\mu} \bar{\sigma}_\mu}
    \end{pmatrix} \begin{pmatrix}
        0 & 1 \\
        1 & 0
    \end{pmatrix},
\]
where we have used the definition of \(\overline{u}^n\). Computing the sum over the basis we get
\[
    \sum_{n} u^n(\mathbf{p}) \overline{u}^n(\mathbf{p}) = \sum_{n} \begin{pmatrix}
        \sqrt{p^{\mu} \sigma_\mu} \xi^n \xi^{n\, \dagger} \sqrt{p^{\mu} \overline{\sigma}_\mu}            & \sqrt{p^{\mu} \sigma_\mu} \xi^n \xi^{n\, \dagger} \sqrt{p^{\mu} \sigma_\mu}            \\
        \sqrt{p^{\mu} \overline{\sigma}_\mu} \xi^n \xi^{n\, \dagger} \sqrt{p^{\mu} \overline{\sigma}_\mu} & \sqrt{p^{\mu} \overline{\sigma}_\mu} \xi^n \xi^{n\, \dagger} \sqrt{p^{\mu} \sigma_\mu}
    \end{pmatrix} = (\gamma^{\mu} p_{\mu} + m).
\]
Now if we use the completeness relation for the basis
\[
    \sum_{n} \xi^n \xi^{n\, \dagger} = \sum_{n} \begin{pmatrix}
        1 \\
        0
    \end{pmatrix} \begin{pmatrix}
        1 & 0
    \end{pmatrix} = \begin{pmatrix}
        1 & 0 \\
        0 & 1
    \end{pmatrix} = \mathbb{I}_2,
\]
then we find
\[
    \sum_{n} u^n(\mathbf{p}) \overline{u}^n(\mathbf{p}) = \begin{pmatrix}
        \sqrt{(p \sigma)(p \overline{\sigma})} & (p \sigma)                             \\
        (p \overline{\sigma})                  & \sqrt{(p \overline{\sigma})(p \sigma)}
    \end{pmatrix} = \begin{pmatrix}
        m                          & p^{\mu} \sigma_{\mu} \\
        p^{\mu} \bar{\sigma}_{\mu} & m
    \end{pmatrix} = \gamma^{\mu} p_{\mu} + m \mathbb{I}_4.
\]
Similarly for the negative frequency solutions we have
\[
    \sum_{n} v^n(\mathbf{p}) \overline{v}^n(\mathbf{p}) = \dots = \gamma^{\mu} p_{\mu} - m \mathbb{I}_4.
\]

\section{Quantizing Dirac Theory}
We aim to quantize the theory as we did for the scalar KG field, promoting the classical fields to operators acting on a suitable Hilbert space. Starting from Lagrangian formalism, we aim to compute the conjugate momenta, Hamiltonian density, and Hamiltonian operator for the Dirac field.

We know \(\pi = \frac{\partial \mathcal{L}}{\partial \dot{\psi}}\), with the Dirac lagrangian density
\[
    \mathcal{L} = \overline{\psi} (i \gamma^{\mu} \partial_{\mu} - m) \psi,
\]
we have
\[
    \pi =\overline{\psi} i \gamma^0 = i \psi^{\dagger}.
\]

Now, exactely as for the scalar field, we have to impose canonical commutation relations between the field and its conjugate momentum at equal times, only now we have to take into account that the Dirac field is a spinor, so we have to consider each component separately; so in Schrödinger picture we impose
\[
    \begin{aligned}
        [\psi_{\alpha}(\mathbf{x}),\, \pi_{\beta}(\mathbf{y})]  & = i \delta_{\alpha \beta} \delta^{(3)}(\mathbf{x} - \mathbf{y}), \\
        [\psi_{\alpha}(\mathbf{x}),\, \psi_{\beta}(\mathbf{y})] & = [\pi_{\alpha}(\mathbf{x}),\, \pi_{\beta}(\mathbf{y})] = 0.
    \end{aligned}
\]
We will see how this choice leads to inconsistencies, so we will have to modify it. In fact we will obtain either negative energy states or negative norm states, both unphysical; thus we will have to impose anticommutation relations instead of commutation relations, in order to obtain a consistent quantum theory for spin \(\tfrac12\) particles. This is related to the spin-statistics theorem, which states that particles with half-integer spin are fermions and must obey Fermi-Dirac statistics, while particles with integer spin are bosons and obey Bose-Einstein statistics.

\subsection{Quantizing with canonical commutation relations}

Let us write the general solution of the Dirac equation in terms of linear combinations of positive and negative frequency solutions:
\[
    \psi(\mathbf{x}) = \sum_{s=1}^2 \int \frac{d^3 p}{(2 \pi)^3} \frac{1}{\sqrt{2 E_{\mathbf{p}}}} \left( \hat{b}_{\mathbf{p}}^s u^s(\mathbf{p}) e^{i \mathbf{p} \cdot \mathbf{x}} + \hat{c}^{s\, \dagger}_{\mathbf{p}} v^s(\mathbf{p}) e^{-i \mathbf{p} \cdot \mathbf{x}} \right),
\]
and its adjoint
\[
    \psi^{\dagger}(\mathbf{x}) = \sum_{s=1}^2 \int \frac{d^3 p}{(2 \pi)^3} \frac{1}{\sqrt{2 E_{\mathbf{p}}}} \left( \hat{b}_{\mathbf{p}}^{s\, \dagger} u^{s\, \dagger}(\mathbf{p}) e^{-i \mathbf{p} \cdot \mathbf{x}} + \hat{c}^s_{\mathbf{p}} v^{s\, \dagger}(\mathbf{p}) e^{i \mathbf{p} \cdot \mathbf{x}} \right).
\]
We understand that the operators \(\hat{b}_{\mathbf{p}}^s\) and \(\hat{b}_{\mathbf{p}}^{s\, \dagger}\) are annihilation and creation operators for particles with momentum \(\mathbf{p}\) and spin \(s\), while \(\hat{c}_{\mathbf{p}}^s\) and \(\hat{c}_{\mathbf{p}}^{s\, \dagger}\) are annihilation and creation operators for antiparticles with momentum \(\mathbf{p}\) and spin \(s\). So we can interpret the negative frequency solutions as antiparticles, as we did for the scalar field.\footnote{Of course this field can consider the charge degrees of freedom, since it is a complex field.}

If we now impose the canonical commutation relations on the field operators, we find that the commutation relations for the creation and annihilation operators are
\[
    \begin{aligned}
        [\hat{b}_{\mathbf{p}}^s,\, \hat{b}_{\mathbf{q}^{r}\, \dagger}] & = (2 \pi)^3 \delta^{(3)}(\mathbf{p} - \mathbf{q}) \delta^{s r},  \\
        [\hat{c}_{\mathbf{p}}^s,\, \hat{c}_{\mathbf{q}}^{r\, \dagger}] & = -(2 \pi)^3 \delta^{(3)}(\mathbf{p} - \mathbf{q}) \delta^{s r},
    \end{aligned}
\]
with all other commutators vanishing.
The minus sign in the second commutation relation is problematic, since it leads to negative norm states when we compute the norm of single antiparticle states, as we will see.

It is easy to compute that from these commutation relations we can derive the first relations presented in this section \(\left[\hat{\psi}_{\alpha}(\mathbf{x}),\, \hat{\psi}_{\beta}^\dagger(\mathbf{y})\right] = \delta_{\alpha \beta}\delta^{(3)}(\mathbf{x} - \mathbf{y})\):\footnote{The second term being proportional to \(\hat{\pi} \) as we have seen.}
\[
    \begin{aligned}
        \left[\hat{\psi}(\mathbf{x}),\, \hat{\psi}^\dagger(\mathbf{y})\right] & = \sum_{r,s} \int \frac{\mathrm{d}^3 p \mathrm{d}^3 q}{(2 \pi)^6} \frac{1}{2 \sqrt{E_{\mathbf{p}} E_{\mathbf{q}}} } \left( [\hat{b}_{\mathbf{p}}^s,\, \hat{b}_{\mathbf{q}}^{r\, \dagger}] u^s(\mathbf{p}) u^{r\, \dagger}(\mathbf{q}) e^{i (\mathbf{p} \cdot \mathbf{x} - \mathbf{q} \cdot \mathbf{y})} + [\hat{c}_{\mathbf{p}}^{s\, \dagger},\, \hat{c}_{\mathbf{q}}^r] v^s(\mathbf{p}) v^{r\, \dagger}(\mathbf{q}) e^{-i (\mathbf{p} \cdot \mathbf{x} - \mathbf{q} \cdot \mathbf{y})} \right) \\
                                                                              & = \sum_{s} \int \frac{\mathrm{d}^3 p}{(2 \pi)^3} \frac{1}{2 E_{\mathbf{p}}} \left( u^s(\mathbf{p}) u^{s\, \dagger}(\mathbf{p}) e^{i \mathbf{p} \cdot (\mathbf{x} - \mathbf{y})} - v^s(\mathbf{p}) v^{s\, \dagger}(\mathbf{p}) e^{-i \mathbf{p} \cdot (\mathbf{x} - \mathbf{y})} \right)                                                                                                                                                                                                         \\
                                                                              & = \dots                                                                                                                                                                                                                                                                                                                                                                                                                                                                                         \\
    \end{aligned}
\]
where we can use the outer product formulae for the spinors
\[
    \begin{dcases}
        \sum_{s=1}^2 u^s(\mathbf{p}) \overline{u}^s(\mathbf{p}) = \gamma^{\mu} p_{\mu} + m, \\
        \sum_{s=1}^2 v^s(\mathbf{p}) \overline{v}^s(\mathbf{p}) = \gamma^{\mu} p_{\mu} - m,
    \end{dcases}
\]
to obtain
\[
    \dots = \int \frac{\mathrm{d}^3 p}{(2 \pi)^3} \left[ \left(p_0 \gamma^0 + p_i \gamma^i + m\right) e^{i \mathbf{p} \cdot (\mathbf{x} - \mathbf{y})} + \left(p_0 \gamma^0 + p_i \gamma^i - m\right) e^{-i \mathbf{p} \cdot (\mathbf{x} - \mathbf{y})} \right],
\]
where now we can perform a change of variable on the second term \(\mathbf{p} \to -\mathbf{p}\), so that \(p_i \to -p_i\) and \(p_0 \to p_0\), and we get
\[
    \dots = \int \frac{\mathrm{d}^3 p}{(2 \pi)^3} \left[ 2 p_0 \gamma^0 e^{i \mathbf{p} \cdot (\mathbf{x} - \mathbf{y})} \right] = \gamma^0 \delta^{(3)}(\mathbf{x} - \mathbf{y}),
\]\TODO{there was a gamma0 missing from the last step of aligned which simplifies in the end.}
which is exactly what we wanted to prove.

If we consider the vacuum state \(\ket{0}\) such that
\[
    \hat{c}^r_{\mathbf{p}} \ket{0} = 0, \quad \forall \mathbf{p},\, r,
\]
and
\[
    \hat{c}^{s\, \dagger}_{\mathbf{p}} \ket{0} = \ket{\mathbf{p},\, s},
\]
but if we now compute the norm of this one-particle state we get
\[
    \bra{\mathbf{p},\,s}\ket{\mathbf{p},\, s} = \bra{0} \hat{c}^s_{\mathbf{p}} \hat{c}^{s\, \dagger}_{\mathbf{p}} \ket{0} = \bra{0} [\hat{c}^s_{\mathbf{p}},\, \hat{c}^{s\, \dagger}_{\mathbf{p}}] \ket{0} = - (2 \pi)^3 \delta^{(3)}(0) \bra{0}\ket{0} < 0,
\]
which is negative, leading to negative norm states, which are unphysical. Thus we have to modify our initial assumption, since maybe we have made a wrong choice in imposing the dagger on the creation operator; maybe the creation operator should be defined without the dagger, so that
\[
    \hat{c}^r_{\mathbf{p}} \ket{0} \neq 0 \to \hat{c}^r_{\mathbf{p}} \ket{0} = \ket{\mathbf{p},\, r},
\]
and if we swap the dagger in the commutation relations we get the opposite sign indeed:
\[
    \left[\hat{c}^{r,\,\dagger}_{\mathbf{p}},\, \hat{c}^s_{\mathbf{q}} \right] = - (2 \pi)^3 \delta^{(3)}(\mathbf{p} - \mathbf{q}) \delta^{r s}.
\]
But this interpretation is not consistent, since when we consider the energy solutions they will be negative. As we will see, the solution to this problem is to impose anticommutation relations instead of commutation relations, so that we can have positive norm states for both particles and antiparticles.

Let us ignore the probability interpretation for now and compute the Hamiltonian operator for the Dirac field, to show that this is not the solution of the problem. Starting from the Hamiltonian density
\[
    \begin{aligned}
        \mathcal{H} & = \pi \dot{\psi} - \mathcal{L} = i \psi^{\dagger} \partial_t \psi - \overline{\psi} (i \gamma^{\mu} \partial_{\mu} - m) \psi = \dots \\
                    & = \overline{\psi} ( - i \gamma^i \partial_i + m) \psi,
    \end{aligned}
\]
where we have used the Dirac equation to simplify the expression. Now we can write the Hamiltonian operator by promoting the fields to operators and using the
\[
    \begin{aligned}
        ( - i \gamma^i \partial_i + m) \psi(x) & = \sum_{s} \int \frac{\mathrm{d}^3 p}{(2 \pi)^3} \frac{1}{\sqrt{2 E_{\mathbf{p}}}} \left( \hat{b}_{\mathbf{p}}^s ( - i \gamma^i \partial_i + m) u^s(\mathbf{p}) e^{i \mathbf{p} \cdot \mathbf{x}} + \hat{c}_{\mathbf{p}}^{s\, \dagger} ( - i \gamma^i \partial_i + m) v^s(\mathbf{p}) e^{-i \mathbf{p} \cdot \mathbf{x}} \right) \\
                                               & = \sum_{s} \int \frac{\mathrm{d}^3 p}{(2 \pi)^3} \frac{1}{\sqrt{2 E_{\mathbf{p}}}} \left( \hat{b}_{\mathbf{p}}^s (-\gamma^{i} p_{i} + m) u^s(\mathbf{p}) e^{i \mathbf{p} \cdot \mathbf{x}} + \hat{c}_{\mathbf{p}}^{s\, \dagger} (\gamma^{i} p_{i} + m) v^s(\mathbf{p}) e^{-i \mathbf{p} \cdot \mathbf{x}} \right),
    \end{aligned}
\]
where we can now use the dirac equations for the spinors
\[
    (\gamma^{\mu} p_{\mu} - m) u^s(\mathbf{p}) = 0, \quad (\gamma^{\mu} p_{\mu} + m) v^s(\mathbf{p}) = 0,
\]
to get
\[
    \begin{aligned}
        (-\gamma^{i} p_{i} + m) u^s(\mathbf{p}) & = \gamma^{0} p_{0} u^s(\mathbf{p}) = E_{\mathbf{p}} \gamma^{0} u^s(\mathbf{p}),    \\
        (\gamma^{i} p_{i} + m) v^s(\mathbf{p})  & = -\gamma^{0} p_{0} v^s(\mathbf{p}) = - E_{\mathbf{p}} \gamma^{0} v^s(\mathbf{p}).
    \end{aligned}
\]
Thus if we insert these results back into the expression for \(( - i \gamma^i \partial_i + m) \psi(x)\) we get
\[
    \dots = \sum_{s} \int \frac{\mathrm{d}^3 p}{(2 \pi)^3} \sqrt{\frac{E_{\mathbf{p}}}{2}} \gamma^{0} \left( \hat{b}_{\mathbf{p}}^s u^s(\mathbf{p}) e^{i \mathbf{p} \cdot \mathbf{x}} - \hat{c}_{\mathbf{p}}^{s\, \dagger} v^s(\mathbf{p}) e^{-i \mathbf{p} \cdot \mathbf{x}} \right),
\]
and finally we can write the Hamiltonian operator as
\[
    \begin{aligned}
        \hat{H} & = \int \mathrm{d}^3 x \, \hat{\overline{\psi}}(x) \left( - i \gamma^i \partial_i + m \right) \hat{\psi}(x)                                                                                                                                                                                                                                                                                                                                                                                                                                                     \\
                & = \sum_{r,s} \int \frac{\mathrm{d}^{3}\mathbf{x} \mathrm{d}^3 p \mathrm{d}^3 q}{(2 \pi)^6} \frac{1}{2}\sqrt{\frac{E_{\mathbf{p}}}{ E_{\mathbf{q}}}} \left( \hat{b}_{\mathbf{q}}^{r\, \dagger} \overline{u}^r(\mathbf{q}) e^{-i \mathbf{q} \cdot \mathbf{x}} + \hat{c}_{\mathbf{q}}^r \overline{v}^r(\mathbf{q}) e^{i \mathbf{q} \cdot \mathbf{x}} \right) \gamma^{0} \gamma^0 \left( \hat{b}_{\mathbf{p}}^s u^s(\mathbf{p}) e^{i \mathbf{p} \cdot \mathbf{x}} - \hat{c}_{\mathbf{p}}^{s\, \dagger} v^s(\mathbf{p}) e^{-i \mathbf{p} \cdot \mathbf{x}} \right),
    \end{aligned}
\]
since \(\overline{\psi} = \psi^{\dagger} \gamma^0\) and \(\gamma^0 \gamma^0 = \mathbb{I}\). Now we can perform the integration over \(\mathbf{x}\) which gives us a delta function \((2 \pi)^3 \delta^{(3)}(\mathbf{p} - \mathbf{q})\) or \((2 \pi)^3 \delta^{(3)}(\mathbf{p} + \mathbf{q})\) depending on the exponentials, so that we get
\[
    \dots = \sum_{r,s} \int \frac{\mathrm{d}^3 p \mathrm{d}^3 \mathbf{q}}{(2 \pi)^3} \frac{1}{2} \sqrt{\frac{E_{\mathbf{p}}}{E_{\mathbf{q}}}} \left( \hat{b}_{\mathbf{q}}^{r\, \dagger} \hat{b}_{\mathbf{p}}^s u^{r,\,\dagger}(\mathbf{q}) u^s(\mathbf{p}) \delta^{(3)}(\mathbf{p} -\mathbf{q}) + \dots  \right),
\]
where we can now use the delta functions to perform the integration over \(\mathbf{q}\):\footnote{Remember that we are working with four dimensional vectors, but the Hamiltonian is either a number or an operator: in the end we ought to have products among row vectors (daggered) and column ones.}
\[
    \dots = \sum_{r,s} \int \frac{\mathrm{d}^3 p}{(2 \pi)^3} \frac{1}{2} \left( \hat{b}_{\mathbf{p}}^{r\, \dagger} \hat{b}_{\mathbf{p}}^s u^{r,\,\dagger}(\mathbf{p}) u^s(\mathbf{p}) - \hat{c}^r_{\mathbf{p}}\hat{c}^{s,\,\dagger}_{\mathbf{p}} v^{s,\,\dagger}(\mathbf{p}) v^s(\mathbf{p}) - bc + cb [\dots]  \right).
\]
If we remember the inner product formulae for the spinors
\[
    \begin{dcases}
        u^{n\,\dagger}(\mathbf{p}) u^m(\mathbf{p}) = 2 E_{\mathbf{p}} \delta^{nm}, \\
        v^{n\,\dagger}(\mathbf{p}) v^m(\mathbf{p}) = 2 E_{\mathbf{p}} \delta^{nm}, \\
        u^{n\,\dagger}(\mathbf{p}) v^m(-\mathbf{p}) = \dots = 0,                   \\
        v^{n\,\dagger}(\mathbf{p}) u^m(-\mathbf{p}) = \dots = 0,
    \end{dcases}
\]
we note that the last two terms can be simplified since they are proportional to the same inner product with different sign on the momentum, so they vanish. Thus we get
\[
    \hat{H} = \sum_{s} \int \frac{\mathrm{d}^3 p}{(2 \pi)^3} E_{\mathbf{p}} \left( \hat{b}_{\mathbf{p}}^{s\, \dagger} \hat{b}_{\mathbf{p}}^s - \hat{c}^s_{\mathbf{p}} \hat{c}^{s\, \dagger}_{\mathbf{p}} \right).
\]
After this long computation, we chack the physical consistency of this Hamiltonian operator: we are spanning spins and momenta, then we have an energy term \(E_{\mathbf{p}} = \sqrt{\mathbf{p}^2 + m^2} > 0\) multiplied by number operators for particles and antiparticles. The problem is that the c-particle term \(c^s_{\mathbf{p}}c^{s,\,\dagger}\) has a minus sign, associeted than to negative energy contribution. Thus we have fixed the norm problem, but now we have negative energy states, which are also unphysical.

We try to solve it by reordering the operators in the antiparticle term, so that we have \(- \hat{c}^s_{\mathbf{p}} \hat{c}^{s\, \dagger}_{\mathbf{p}} = - \hat{c}^{s\, \dagger}_{\mathbf{p}} \hat{c}^s_{\mathbf{p}} - \left[commutator\right]\) such that
\[
    \hat{H} = \sum_{s} \int \frac{\mathrm{d}^3 p}{(2 \pi)^3} E_{\mathbf{p}} \left( \hat{b}_{\mathbf{p}}^{s\, \dagger} \hat{b}_{\mathbf{p}}^s - \hat{c}^{s\, \dagger}_{\mathbf{p}} \hat{c}^s_{\mathbf{p}} + (2 \pi)^3 \delta^{(3)}(0) \right),
\]
where we have used the commutation relations to swap the operators. Now we have positive energy contributions only, but we have introduced an infinite constant term proportional to \(\delta^{(3)}(0)\), which is divergent, but we can remove it by redefining the zero point of energy and normal ordering, as we did for the scalar field.

Let us verify explicitly that b-type particles have positive energy while c-typ particles have negative energy if we define them with positive norm, by computing the commutator of the Hamiltonian with the creation operators:
\[
    \begin{aligned}
        \left[\hat{H} ,\, b_{\mathbf{p}}^{s,\, \dagger} \right] & = \dots                                         \\
                                                                & = E_{\mathbf{p}} b_{\mathbf{p}}^{s,\, \dagger},
    \end{aligned}
\]
while for the c-type particles we have similarly
\[
    \begin{aligned}
        \left[\hat{H} ,\, c_{\mathbf{p}}^{s,\, \dagger} \right] & = \dots                                         \\
                                                                & = E_{\mathbf{p}} c_{\mathbf{p}}^{s,\, \dagger},
    \end{aligned}
\]
so that both particles have positive energy excitations, but we have to face the negative norm problem again for c-type particles.\TODO{change notation until now so that we dont speak of part and antipart but b-type and c-type particles.}
Now the action of the Hamiltonian on one-particle states is
\[
    \begin{aligned}
        \hat{H} \left( \hat{b}^{s,\,\dagger}_{\mathbf{p}} \ket{0}\right) & = E_{\mathbf{p}} \left( \hat{b}^{s,\,\dagger}_{\mathbf{p}} \ket{0}\right) + \hat{b}^{s,\,\dagger}_{\mathbf{p}} \hat{H} \ket{0} = E_{\mathbf{p}} \left( \hat{b}^{s,\,\dagger}_{\mathbf{p}} \ket{0}\right), \\
        \hat{H} \left( \hat{c}^{s,\,\dagger}_{\mathbf{p}} \ket{0}\right) & = E_{\mathbf{p}} \left( \hat{c}^{s,\,\dagger}_{\mathbf{p}} \ket{0}\right) + \hat{c}^{s,\,\dagger}_{\mathbf{p}} \hat{H} \ket{0} = E_{\mathbf{p}} \left( \hat{c}^{s,\,\dagger}_{\mathbf{p}} \ket{0}\right),
    \end{aligned}
\]
so that both types of particles have positive energy excitations, but we still have the negative norm problem for c-type particles.

If we try quantizing the Dirac field by imposing commutation relations we end up with either negative norm states or negative energy states, both unphysical:
\begin{itemize}
    \item \textbf{b-type} particles are associated to positive frequency solutions, have positive norm and positive energy;
    \item \textbf{c-type} particles are associated to negative frequency solutions, give us problems:
          \begin{enumerate}
              \item if we define them with positive energy states (\(\hat{c}^{s,\,\dagger}_{\mathbf{p}}\) defined as creation operator) they have negative norm: ill-defined Hilbert space;
              \item if we define them with positive norm (\(\hat{c}^s_{\mathbf{p}}\) defined as creation operator) they have negative energy states: energy unbounded from below.\footnote{This is similar to what happened for the classical Dirac equation, where negative frequency solutions had to be reinterpreted as antiparticles to avoid negative energy states; we say unbounded from below because we could create more and more negative energy states by adding more and more c-type particles, leading to an unstable vacuum.}
          \end{enumerate}
\end{itemize}
We thus have to change paradigm of quantization.

\subsection{Quantizing with canonical anticommutation relations}

If we recall what happened for \textbf{bosons} in Klein-Gordon theory, we imposed canonical \textbf{commutation} relations between the field and its conjugate momentum at equal times, leading to commutation relations for creation and annihilation operators
\[
    \ket{\mathbf{p},\,\mathbf{q}} = \hat{a}^{\dagger}_{\mathbf{p}} \hat{a}^{\dagger}_{\mathbf{q}} \ket{0} = \hat{a}^{\dagger}_{\mathbf{q}} \hat{a}^{\dagger}_{\mathbf{p}} \ket{0},
\]
since the creation operators commute
\[
    \left[\hat{a}^{\dagger}_{\mathbf{p}}, \hat{a}^{\dagger}_{\mathbf{q}}\right] = 0.
\]
This means that we can create multiple particles in the same state, leading to \textit{Bose-Einstein statistics}.

We do not find this acceptable for \textbf{fermions}, since they obey the \textit{Pauli exclusion principle}, which states that no two fermions can occupy the same quantum state simultaneously. We need a minus sign when swapping two fermionic operators, thus we have to impose \textbf{canonical anticommutation relations} between the field and its conjugate momentum at equal times, leading to anticommutation relations for creation and annihilation operators:
\[
    \begin{aligned}
        \{\hat{\psi}_{\alpha}(\mathbf{x}),\, \hat{\psi}_{\beta}(\mathbf{y})\} = \{\hat{\psi}_{\alpha}^{\dagger}(\mathbf{x}),\, \hat{\psi}_{\beta}^{\dagger}(\mathbf{y})\} = 0, \\
        \{\psi_{\alpha}(\mathbf{x}),\, \hat{\psi}^{\dagger}_{\beta}(\mathbf{y})\} = \delta_{\alpha \beta} \delta^{(3)}(\mathbf{x} - \mathbf{y}).
    \end{aligned}
\]
This leads to anticommutation relations for creation and annihilation operators:
\[
    \begin{aligned}
        \{\hat{b}_{\mathbf{p}}^s,\, \hat{b}_{\mathbf{q}}^{r\, \dagger}\} & = (2 \pi)^3 \delta^{(3)}(\mathbf{p} - \mathbf{q}) \delta^{s r}, \\
        \{\hat{c}_{\mathbf{p}}^s,\, \hat{c}_{\mathbf{q}}^{r\, \dagger}\} & = (2 \pi)^3 \delta^{(3)}(\mathbf{p} - \mathbf{q}) \delta^{s r},
    \end{aligned}
\]
while all other anticommutators vanish.

Now, as last time, we have to compute the commutator of the Hamiltonian operator
with the creation operators to check if the energy excitations are positive and associated to positive norm states. Repeating the same computation as before, but using anticommutation relations instead of commutation relations, we find
\[
    \hat{H} = \sum_{s} \int \frac{\mathrm{d}^3 \mathbf{p}}{(2 \pi)^3} E_{\mathbf{p}} \left( \hat{b}_{\mathbf{p}}^{s\, \dagger} \hat{b}_{\mathbf{p}}^s - \hat{c}^{s\, \dagger}_{\mathbf{p}} \hat{c}^s_{\mathbf{p}} \right),
\]
where we can now reorder the operators in the c-type particle term introducing a minus sign from the anticommutation relations:
\[
    \hat{H} = \sum_{s} \int \frac{\mathrm{d}^3 \mathbf{p}}{(2 \pi)^3} E_{\mathbf{p}} \left( \hat{b}_{\mathbf{p}}^{s\, \dagger} \hat{b}_{\mathbf{p}}^s + \hat{c}^{s\, \dagger}_{\mathbf{p}} \hat{c}^s_{\mathbf{p}} - (2 \pi)^3 \delta^{(3)}(0) \right),
\]
where now both types of particles have positive energy excitations, if we interpret \(\hat{c}^{\dagger}_{\mathbf{p}}\) as the creation operators and thus the second term as the c-type particle number operator; in the end we have a divergent constant term which we can remove by normal ordering as before.\footnote{This constant term is related to the vacuum energy, which in KG theory came with a plus sign.}
Thus we have finally obtained a consistent quantum theory for spin \(\tfrac12\) particles, by imposing anticommutation relations instead of commutation relations and a normal ordered Hamiltonian defined as
\begin{equation}
    \hat{H} = \sum_{s} \int \frac{\mathrm{d}^3 \mathbf{p}}{(2 \pi)^3} E_{\mathbf{p}} \left( \hat{b}_{\mathbf{p}}^{s\, \dagger} \hat{b}_{\mathbf{p}}^s + \hat{c}^{s\, \dagger}_{\mathbf{p}} \hat{c}^s_{\mathbf{p}} \right).
    \label{eq:Dirac_hamiltonian_normal_ordered}
\end{equation}

We can verify explicitly that both b-type and c-type particles have positive energy excitations, looking at the vacuum state with no particles
\[
    \hat{b}^s_{\mathbf{p}} \ket{0} = 0, \quad \hat{c}^s_{\mathbf{p}} \ket{0} = 0, \quad \forall \mathbf{p},\, s,
\]
and by computing the commutator of the Hamiltonian with the creation operators, with the idea to apply it to one-particle states:
\[
    \begin{aligned}
        \left[\hat{H} ,\, \hat{b}_{\mathbf{p}}^{s,\, \dagger} \right] = E_{\mathbf{p}} \hat{b}_{\mathbf{p}}^{s,\, \dagger}, \quad \left[\hat{H} ,\, \hat{c}_{\mathbf{p}}^{s,\, \dagger} \right] = E_{\mathbf{p}} \hat{c}_{\mathbf{p}}^{s,\, \dagger}, \\
        \left[\hat{H},\, \hat{b}_{\mathbf{p}}^{s}\right] = - E_{\mathbf{p}} \hat{b}_{\mathbf{p}}^{s}, \quad \left[\hat{H},\, \hat{c}_{\mathbf{p}}^{s}\right] = - E_{\mathbf{p}} \hat{c}_{\mathbf{p}}^{s}.
    \end{aligned}
\]
Now on \textbf{one-particle states} we have
\[
    \begin{aligned}
        \ket{\mathbf{p}, s}_b = \hat{b}_{\mathbf{p}}^{s,\, \dagger} \ket{0}, \\
        \ket{\mathbf{p} , s}_c = \hat{c}_{\mathbf{p}}^{s,\, \dagger} \ket{0},
    \end{aligned}
\]
so that the norm of these states is positive:
\[
    \begin{aligned}
        {}_b\bra{\mathbf{q}, r}\ket{\mathbf{p}, s}_b & = \bra{0} \hat{b}_{\mathbf{q}}^r \hat{b}_{\mathbf{p}}^{s,\, \dagger} \ket{0} = \bra{0} \{\hat{b}_{\mathbf{q}}^r,\, \hat{b}_{\mathbf{p}}^{s,\, \dagger}\} \ket{0} = (2 \pi)^3 \delta^{(3)}(0) \bra{0}\ket{0} > 0, \\
        {}_c\bra{\mathbf{q}, r}\ket{\mathbf{p}, s}_c & = \bra{0} \hat{c}_{\mathbf{q}}^r \hat{c}_{\mathbf{p}}^{s,\, \dagger} \ket{0} = \bra{0} \{\hat{c}_{\mathbf{q}}^r,\, \hat{c}_{\mathbf{p}}^{s,\, \dagger}\} \ket{0} = (2 \pi)^3 \delta^{(3)}(0) \bra{0}\ket{0} > 0,
    \end{aligned}
\]
wher the difference with respect to the commutation relation case is the anticommutator used to swap the operators, which does not introduce a minus sign as the commutator did.

Finally, we can compute the action of the Hamiltonian on one-particle states:
\[
    \begin{aligned}
        \hat{H} \left(\hat{b}^{s,\,\dagger}_{\mathbf{p}} \ket{0} \right) & = \left[\hat{H} ,\, \hat{b}^{s,\,\dagger}_{\mathbf{p}} \right] \ket{0} + \hat{b}^{s,\,\dagger}_{\mathbf{p}} \hat{H} \ket{0} = E_{\mathbf{p}} \left(\hat{b}^{s,\,\dagger}_{\mathbf{p}} \ket{0} \right), \\
        \hat{H} \left(\hat{c}^{s,\,\dagger}_{\mathbf{p}} \ket{0} \right) & = \left[\hat{H} ,\, \hat{c}^{s,\,\dagger}_{\mathbf{p}} \right] \ket{0} + \hat{c}^{s,\,\dagger}_{\mathbf{p}} \hat{H} \ket{0} = E_{\mathbf{p}} \left(\hat{c}^{s,\,\dagger}_{\mathbf{p}} \ket{0} \right).
    \end{aligned}
\]
Thus both b-type and c-type particles have positive norm states and positive energy excitations, solving all the problems we had before and leading to a consistent quantum theory for spin \(\tfrac12\) particles.

The question we are now left is: what is the difference between \(b-\)type and \(c-\)type particles? They are both spin \(\tfrac12\) particles with mass \(m\) and positive energy excitations; they are degenerate eigenstates of the Hamiltonian, momentum and spin. The answer is that there has to be another conserved associated to a noether charge symmetry, which distinguishes this two types of particles: this charge is the electric charge, and \(b-\)type particles are associated to particles with charge \(+e\) (like electrons), while \(c-\)type particles are associated to antiparticles with charge \(-e\) (like positrons), as we will see in the next section.

\subsection{Internal Global Symmetry}

We are looking for a conserved charge which distinguishes between \(b-\)type and \(c-\)type particles. To find it, we can search for a global (i.e., spacetime-independent) and internal (i.e., acting on the fields but not on spacetime coordinates) symmetry of the Dirac Lagrangian. It is easy to see that the Dirac Lagrangian is invariant under the global phase transformation
\[
    \psi(x) \to \psi'(x) = e^{-i q \theta} \psi(x), \quad \overline{\psi}(x) \to \overline{\psi}'(x) = e^{i q \theta} \overline{\psi}(x),
\]
where \(q\) is a constant associated to the charge of the field and \(\theta\) is a constant parameter. This is a symmetry since the Lagrangian density
\[
    \mathcal{L} = \overline{\psi}(i \gamma^{\mu} \partial_{\mu} - m) \psi,
\]
depends on \(\psi\) and \(\overline{\psi}\) only in the combinations \(\overline{\psi} \psi\) and \(\overline{\psi} \gamma^{\mu} \partial_{\mu} \psi\), which are invariant under this transformation.
\[
    \mathcal{L} \to \mathcal{L}^{\prime} = \overline{\psi}'(i \gamma^{\mu} \partial_{\mu} - m) \psi' = e^{i q \theta} \overline{\psi}(i \gamma^{\mu} \partial_{\mu} - m) e^{-i q \theta} \psi = \overline{\psi}(i \gamma^{\mu} \partial_{\mu} - m) \psi = \mathcal{L}.
\]
This is a global \(\mathrm{U}(1)\) symmetry, thus associated to a conserved noether current and a one dimensional charge.
\begin{remark}
    If we were to consider a coordinate-dependent phase \(\theta(x)\) we would have a local \(\mathrm{U}(1)\) symmetry, which is the gauge symmetry of quantum electrodynamics (QED), as we will see in the next chapter and the starter point for introducing interactions between the Dirac field and the electromagnetic field. We thus need a global \(\mathrm{U}(1)\) symmetry to have a conserved charge distinguishing between particles and antiparticles, since the derivatives in the Lagrangian would break the local symmetry.
\end{remark}
Using noether's theorem we can compute the conserved current as
\[
    \begin{aligned}
        J^{\mu} & = \dots                                        \\
                & = \dots = q \overline{\psi} \gamma^{\mu} \psi,
    \end{aligned}
\]
which is a conserved vector, and if we differentiate it we get
\[
    \begin{aligned}
        \partial_{\mu} J^{\mu} & = q \left( (\partial_{\mu} \overline{\psi}) \gamma^{\mu} \psi + \overline{\psi} \gamma^{\mu} (\partial_{\mu} \psi) \right) \\
                               & = q \left( - i \overline{\psi} m \psi + i \overline{\psi} m \psi \right) = 0,
    \end{aligned}
\]
where if we use the equations of mortions
\[
    i gamma^{\mu} \partial_{\mu} \psi = m \psi, \quad \partial_{\mu} \overline{\psi} i \gamma^{\mu} = - m \overline{\psi},
\]
we see that the current is indeed conserved
\[
    \partial_{\mu} J^{\mu} = q \left( - i \overline{\psi} m \psi + i \overline{\psi} m \psi \right) = 0.
\]
The associated conserved charge is\TODO{compute it}
\[
    \begin{aligned}
        \hat{Q} & = \int \mathrm{d}^3 \mathbf{x} J^0 = \dots                                                                                                                                               \\
                & = q \int \mathrm{d}^3 \mathbf{x} \, \hat{\overline{\psi}}(x) \gamma^0 \hat{\psi}(x)                                                                                                      \\
                & = q \int \mathrm{d}^3 \mathbf{x} \, \hat{\psi}^{\dagger}(x) \hat{\psi}(x)                                                                                                                \\
                & = q \sum_{s} \int \frac{\mathrm{d}^3 p}{(2 \pi)^3} \left( \hat{b}_{\mathbf{p}}^{s\, \dagger} \hat{b}_{\mathbf{p}}^s - \hat{c}_{\mathbf{p}}^{s\, \dagger} \hat{c}_{\mathbf{p}}^s \right),
    \end{aligned}
\]
where we have used the field expansion in terms of creation and annihilation operators and the inner product relations for the spinors to simplify the expression.
Thus we can now compute the associated charge of one-particle states:\TODO{expand computation for both cases}
\[
    \begin{aligned}
        \hat{Q} \left( \hat{b}_{\mathbf{p}}^{s\, \dagger} \ket{0} \right) & = q \sum_{r} \int \frac{\mathrm{d}^3 \mathbf{q}}{(2 \pi)^3} \left(\hat{b}_{\mathbf{q}}^{r\, \dagger} \hat{b}_{\mathbf{q}}^r - \hat{c}_{\mathbf{q}}^{r\, \dagger} \hat{c}_{\mathbf{q}}^r\right)\hat{b}_{\mathbf{p}}^{s\, \dagger} \ket{0} = q \left( \hat{b}_{\mathbf{p}}^{s\, \dagger} \ket{0} \right), \\
        \hat{Q} \left( \hat{c}_{\mathbf{p}}^{s\, \dagger} \ket{0} \right) & = - q \left( \hat{c}_{\mathbf{p}}^{s\, \dagger} \ket{0} \right),
    \end{aligned}
\]
so that b-type states are particles with charge \(+q\) while c-type states are antiparticles with charge \(-q\), so that we can summarize the results of our quantization procedure as follows:
\begin{itemize}
    \item \(\hat{b}^{s\, \dagger}_{\mathbf{p}}\) create \textbf{particles} with energy \(E_{\mathbf{p}}\), spin \(s\), momentum \(\mathbf{p}\) and charge \(+q\);
    \item \(\hat{c}^{s\, \dagger}_{\mathbf{p}}\) create \textbf{antiparticles} with energy \(E_{\mathbf{p}}\), spin \(s\), momentum \(\mathbf{p}\) and charge \(-q\).
\end{itemize}

\paragraph{Spin-statistics relations.}
We have seen that to quantize the Dirac field we had to impose anticommutation relations between the field and its conjugate momentum, leading to anticommutation relations for creation and annihilation operators. This is related to the \textit{spin-statistics theorem}, which states that particles with integer spin (bosons) obey Bose-Einstein statistics and thus commutation relations, while particles with half-integer spin (fermions) obey Fermi-Dirac statistics and thus anticommutation relations.

We can see its effect by looking at two-particle states: for two fermions we have
\[
    \ket{\mathbf{p},\,s;\; \mathbf{q},\,r} = \hat{b}_{\mathbf{p}}^{s\, \dagger} \hat{b}_{\mathbf{q}}^{r\, \dagger} \ket{0} = - \hat{b}_{\mathbf{q}}^{r\, \dagger} \hat{b}_{\mathbf{p}}^{s\, \dagger} \ket{0} = - \ket{\mathbf{q},\,r;\; \mathbf{p},\,s},
\]
which is antisymmetric under the exchange of the two particles, exactly as required by the \textbf{Pauli exclusion principle} and described by Fermi-Dirac statistics. In particular, if we try to create two fermions in the same state we have indeed \(\ket{\mathbf{p},\,s;\; \mathbf{p},\,s} =- \ket{\mathbf{p},\,s;\; \mathbf{p},\,s} = 0\), showing that no two fermions can occupy the same quantum state simultaneously.

In the beginning, when Dirac was thinking about his equation, he was not aware of this spin-statistics relation, and he tried to interpret the negative energy solutions of his equation as physical states, leading to an unstable vacuum. He then proposed the \textit{Dirac sea} idea, where all negative energy states are filled in the vacuum, and only holes in this sea (i.e., absence of negative energy electrons) can be interpreted as positrons, thus solving the negative energy problem. This creative idea was later abandoned in favor of the quantum field theory approach we have seen, where antiparticles arise naturally from the quantization procedure and the imposition of anticommutation relations for fermionic fields.

\section{Propagators}

If we now move to the Heisenberg picture, we know that the time evolution of operators is given by
\[
    \hat{\psi} (x), \quad x = (t, \mathbf{x}), \quad \partial_{0} \hat{\psi}(x) = i \left[\hat{H},\, \hat{\psi}(x)\right],
\]
which is solved by the time evolution operator
\[
    \hat{\psi}(x) = \sum_{s} \int \frac{\mathrm{d}^3 \mathbf{p}}{(2 \pi)^3} \frac{1}{2\sqrt{E_{\mathbf{p}}}} \left( \hat{b}_{\mathbf{p}}^s u^s(\mathbf{p}) e^{- i p \cdot x} + \hat{c}_{\mathbf{p}}^{s\, \dagger} v^s(\mathbf{p}) e^{i p \cdot x} \right),
\]
and
\[
    \hat{\psi}^{\dagger}(x) = \sum_{s} \int \frac{\mathrm{d}^3 \mathbf{p}}{(2 \pi)^3} \frac{1}{2\sqrt{E_{\mathbf{p}}}} \left( \hat{b}_{\mathbf{p}}^{s\, \dagger} u^{s\, \dagger}(\mathbf{p}) e^{i p \cdot x} + \hat{c}_{\mathbf{p}}^s v^{s\, \dagger}(\mathbf{p}) e^{- i p \cdot x} \right).
\]

Now to ensure that there is no measurable effect outside the light cone, we have to impose that the anticommutator of the field at spacelike separated points vanishes: we define the \textbf{fermionic propagator} as
\[
    \begin{aligned}
        i S_{\alpha \beta}= \{\hat{\psi}_{\alpha}(x),\, \hat{\overline{\psi}}_{\beta}(y)\} \\
        i S(x-y) = \{\hat{\psi}(x),\, \hat{\overline{\psi}}(y)\}.
    \end{aligned}
\]
Now by substituting the field expansions into this expression and using the anticommutation relations for creation and annihilation operators, we find
\[
    \begin{aligned}
        i S(x-y) & = \sum_{r,s} \int \frac{\mathrm{d}^3 \mathbf{p}\mathrm{d}^3 \mathbf{q}}{(2 \pi)^6} \frac{1}{2 E_{\mathbf{p}}E_{\mathbf{q}}} \left( \{\hat{b}_{\mathbf{p}}^s,\, \hat{b}_{\mathbf{q}}^{r,\,\dagger}\} + \dots\right) \\
                 & = \dots
    \end{aligned}
\]
where we have used the anticommutation relations to simplify the expression, and inner (outer?) products of spinors to get
\[
    iS(x-y) = \int \frac{\mathrm{d}^3 \mathbf{p}}{(2 \pi)^3} \frac{1}{2 E_{\mathbf{p}}} \left( (\slashed{p} + m) e^{- i p \cdot (x-y)} + (\slashed{p} - m) e^{i p \cdot (x-y)} \right).
\]

Now recalling the scalar field correlators for KG theory
\[
    D(x-y) = \int \frac{\mathrm{d}^3 \mathbf{p}}{(2 \pi)^3} \frac{1}{2 E_{\mathbf{p}}} \left( e^{- i p \cdot (x-y)} \right),
\]
if we differenciate it with respect to \(x^{\mu}\) we get
\[
    \partial_{\mu} D(x-y) = \int \frac{\mathrm{d}^3 \mathbf{p}}{(2 \pi)^3} \frac{1}{2 E_{\mathbf{p}}}(- i p_{\mu})e^{- i p \cdot (x-y)},
\]
so that we can rewrite the fermionic propagator as
\[
    i S(x-y) = (i \gamma^{\mu} \partial_\mu^{(x)} + m) \left( D(x-y) - D(y-x) \right).
\]
This propagator satisfies the following properties:
\begin{itemize}
    \item For spacelike propagated points \((x-y)^2<0\):
    \[
        D(x-y) - D(y-x) = 0, \implies S(x-y) = 0,
    \]
    meaning that there is no measurable effect outside the light cone, preserving causality for the Dirac theory;
    \item For bosons the last point was ensured by commutation relations
    \[
        \left[\hat{\varphi}(x), \hat{\varphi}(y)\right] = 0 \quad \text{for} \quad (x-y)^2 < 0,
    \]
    while for fermions it is ensured by anticommutation relations:
    \[
        \{\hat{\psi}_{\alpha}(x), \hat{\psi}_{\beta}(y)\} = 0 \quad \text{for} \quad (x-y)^2 < 0;
    \]
    \item There is a problem: we want observables to commute at spacelike separations, while fermionic fields anticommute. The solution is that observables are constructed as bilinear combinations of fermionic fields, like the Hamiltonian density or the current density, which do commute at spacelike separations, thus preserving causality: the hamiltonian for example
    \[
        \hat{H} = \int \frac{\mathrm{d}^3 \mathbf{p}}{(2 \pi)^3} E_{\mathbf{p}} \left( \hat{b}_{\mathbf{p}}^{s\, \dagger} \hat{b}_{\mathbf{p}}^s + \hat{c}^{s\, \dagger}_{\mathbf{p}} \hat{c}^s_{\mathbf{p}} \right),
    \]
    where the two main terms are fermionic bilinears respecting anti-commutation relations, thus their product commutes at spacelike separations:
    \[
        \begin{aligned}
            \hat{O}_1 = A_1 B_1, \quad \hat{O}_2 = A_2 B_2, \\
            \hat{O}_1 \hat{O}_2 = A_1 B_1 A_2 B_2 = - A_1 A_2 B_1 B_2 = A_2 A_1 B_2 B_1 = \hat{O}_2 \hat{O}_1,
        \end{aligned}
    \]
    if each of the operators \(A_i, B_i\) anticommute with each other, like our fermionic fields.
\end{itemize}

The last thing we can check is that the fermionic propagator satisfies the Dirac equation:
\[
    i \gamma^{\mu} \partial_{\mu}^{(x)} S(x-y) - m S(x-y) =0,
\]
since we can write the propagator in terms of the scalar propagator as
\[
    \begin{aligned}
        \frac{1}{i}(i \gamma^{\mu} \partial_{\mu}^{(x)} - m)(i \gamma^{\mu} \partial_{\mu}^{(x)} + m) (D(x-y) - D(y-x)) &= \dots \\
        &= \frac{-1}{i} (\partial_{\mu}^{(x)} \partial^{\mu\,(x)} + m^2) (D(x-y) - D(y-x)) \\
        &= \frac{1}{i} \int \frac{\mathrm{d}^3 \mathbf{p}}{(2 \pi)^3} \frac{1}{2 E_{\mathbf{p}}} (p_{\mu}p^{\mu} - m^2) \left( e^{- i p \cdot (x-y)} - e^{i p \cdot (x-y)} \right) = 0,
    \end{aligned}
\]
where we have used the KG equation satisfied by the scalar propagator.