\chapter{Dirac theory}

From now on, we aim to describe particles with spin different from zero. The simplest case is spin \(\tfrac12\) fermions, which are described by the Dirac equation. The starting point will be the Lagrangian, then we will show the most general solution to the equations of motion and finally proceed to quantize the system and promote the observables to operators on the Fock Space.

The lagrangian has to be Lorentz invariant, so we recall that the Lorentz group admits a \(4D\) spinor representation, which for dirac translates into
\[
    \psi_D \to \psi_D^{\prime} = e^{-\tfrac{i}{2}\omega_{\mu \nu} \Sigma^{\mu \nu}} \psi_D,
\]
the \textbf{Dirac spinor}, where \(\Sigma_{\mu \nu} = \frac{1}{4} \gamma^{\mu \nu}\) are the \(4\times 4\) antisymmetric generators of the Lorentz group in the Dirac representation,\footnote{\(\Sigma^{\mu \nu}\) has only six independent components, corresponding to the six generators of the Lorentz group: it is antisymmetric in \(\mu\) and \(\nu\) and traceless, thus \((4 \times 4 - 4) / 2 = 6\).}\QUESTION{is it traceless?} since
\[
    \gamma^{\mu \nu} = [\gamma^{\mu},\, \gamma^{\nu}] = \gamma^{\mu} \gamma^{\nu} - \gamma^{\nu} \gamma^{\mu},
\]
respecting the clifford algebra \( \{ \gamma^{\mu}, \gamma^{\nu} \} = 2 \eta^{\mu \nu} \mathbb{I}_4\) of \(4\times 4\) complex matrices.

We adopt the \textbf{Weyl representation} (or chiral representation) of Dirac matrices, which assumes the form
\[
    \gamma^0 = \begin{pmatrix}
        0            & \mathbb{I}_2 \\
        \mathbb{I}_2 & 0
    \end{pmatrix}, \quad \gamma^i = \begin{pmatrix}
        0         & \sigma^i \\
        -\sigma^i & 0
    \end{pmatrix},
\]
where \(\sigma^i\) are the Pauli matrices
\[
    \sigma^1 = \begin{pmatrix}
        0 & 1 \\
        1 & 0
    \end{pmatrix}, \quad \sigma^2 = \begin{pmatrix}
        0 & -i \\
        i & 0
    \end{pmatrix}, \quad \sigma^3 = \begin{pmatrix}
        1 & 0  \\
        0 & -1
    \end{pmatrix}.
\]

The generators of Lorentz transformations in the Dirac representation can be written as
\[
    S^{\mu \nu} = - i \Sigma^{\mu \nu} = \frac{1}{4} \gamma^{\mu \nu}, \quad \psi_D^{\prime} = e^{\frac{1}{2} \omega_{\mu \nu} S^{\mu \nu}} \psi_D.
\]
If we introduce now the \textbf{spinorial indices}, we can write the transformation law as:
\[
    \psi_D^{\alpha}(x) \to \psi_D^{\prime \alpha}(x^{\prime}) = S^{\alpha}_{\ \beta}(\Lambda) \psi_D^{\beta}(x) = \left(e^{\frac{1}{2} \omega_{\mu \nu} S^{\mu \nu}}\right)^{\alpha}_{\ \beta} \psi_D^{\beta}(x),
\]
where \(\Lambda\) is the Lorentz transformation, which also acted on the coordinates (they got transformed along with the Dirac spinor). From now on we will drop the subscript \(D\) for Dirac spinors, since we will only deal with them.

\section{Action and Lagrangian}

Now the idea is to derive an action for the system, thus we need a Lorentz invariant Lagrangian dependent upon \(\psi\), remembering that \(\psi\) has four complex components, so eight real dof, and
\[
    \psi^{\dagger}(x) = (\psi^*)^T(x).
\]
The lagrangian must be a Lorentz scalar, so we need to find building blocks which are lorentz scalars, vectors, tensors\dots built from the spinor \(\psi\) and its derivatives. The first attempt will be to consider the \textbf{spinor bilinear}
\[
    \psi^{\dagger} \psi
\]
as a building block; it's going to be a real number, but we have to check if it is Lorentz invariant.

\subsection{Building Block for the Mass Term}

We are considering the spinor bilinear as a candidate for a Lorentz scalar, thus we have to check its transformation properties under Lorentz transformations. We know that
\[
    \psi \to  \psi^{\prime} = S \psi, \quad \psi^{\dagger} \to  \psi^{\dagger \prime} = \psi^{\dagger} S^{\dagger},
\]
hence their product
\[
    \psi^{\dagger} \psi \to \psi^{\dagger \prime} \psi^{\prime} = \psi^{\dagger} S^{\dagger} S \psi,
\]
which could be Lorentz scalar if and only if \(S^{\dagger} = S^{-1}\), which is not the case, since the Dirac representation is not unitary:
\[
    S^{\dagger} S \neq \mathbb{I}_4 \implies \psi^{\dagger \prime} \psi^{\prime} \neq \psi^{\dagger} \psi,
\]
and therefore the spinor bilinear is not a good building block for our Lagrangian. Let's check why \(S^{\dagger} \neq S^{-1}\):
\[
    S = e^{\tfrac{1}{2} \omega_{\mu \nu} S^{\mu \nu}} \quad \implies \begin{dcases}
        S^{-1}      & = e^{-\tfrac{1}{2} \omega_{\mu \nu} S^{\mu \nu}};           \\
        S^{\dagger} & = e^{\tfrac{1}{2}\omega_{\mu \nu} (S^{\mu \nu})^{\dagger}}.
    \end{dcases} \\
\]
Thus for the representation to be unitary, we are requiring the generators \(S^{\mu \nu}\) to be \textit{anti-hermitian}, i.e.
\[
    (S^{\mu \nu})^{\dagger} = - S^{\mu \nu} \implies (-i \Sigma^{\mu \nu})^{\dagger} = - (-i \Sigma^{\mu \nu});
\]
in other words we are requiring the generators \(\Sigma^{\mu \nu}\) to be hermitian
\[
    (\Sigma^{\mu \nu})^{\dagger} = \Sigma^{\mu \nu}.
\]
But we can explicitly verify that this is never the case:
\[
    (S^{\mu \nu})^{\dagger} = \frac{1}{4} \left[\gamma^{\mu},\,\gamma^{\nu}\right]^\dagger = \frac{1}{4}(\gamma^{\nu})^{\dagger} (\gamma^{\mu})^{\dagger} - \frac{1}{4}(\gamma^{\mu})^{\dagger} (\gamma^{\nu})^{\dagger} = - \frac{1}{4} \left[(\gamma^{\mu})^{\dagger},\, (\gamma^{\nu})^{\dagger}\right],
\]
which would be equal to \(- S^{\mu \nu}\) only if
\[
    (S^{\mu \nu})^{\dagger} = - \frac{1}{4} \left[(\gamma^{\mu})^{\dagger},\, (\gamma^{\nu})^{\dagger}\right] = - \frac{1}{4} \left[\gamma^{\mu},\, \gamma^{\nu}\right] = - S^{\mu \nu}.
\]
Thus \((\gamma^{\mu})^{\dagger} = \pm \gamma^{\mu}\) \(\iff\) the gamma matrices are hermitian or anti-hermitian; we can check from the Clifford algebra that this is not possible for all \(\mu\):
\[
    \{\gamma^{\mu} ,\, \gamma^{\nu}\} = 2 \eta^{\mu \nu} \mathbb{I}_4 \implies (\gamma^0)^2 = \mathbb{I}_4 \quad (\gamma^i)^2 = -\mathbb{I}_4,
\]
where we have computed the anticommutator for \(\mu = \nu = 0\) and \(\mu = \nu = i\). If we apply those squared matrices to a generic complex four-vector \(v\), we get
\[
    (\gamma^0)^2 v = \lambda_0^2 v = v, \quad (\gamma^i)^2 v = \lambda_i^2 v = -v,
\]
thus \(\gamma^0\) has real eigenvalues \(\lambda_0 = \pm 1\), while \(\gamma^i\) have imaginary eigenvalues \(\lambda_i = \pm i\); hence \(\gamma^0\) is hermitian, while \(\gamma^i\) are anti-hermitian. Therefore not all gamma matrices are hermitian or anti-hermitian, so the generators \(S^{\mu \nu}\) are not hermitian, and the Dirac representation is not unitary:
\[
    (\gamma^0)^\dagger = \gamma^0, \quad (\gamma^i)^{\dagger} = - \gamma^i \implies (S^{\mu \nu})^{\dagger} \neq - S^{\mu \nu} \implies S^{\dagger} \neq S^{-1}.
\]
We then need to find another building block for our Lagrangian.

We can start from the following consideration: \((\gamma^{\mu})^{\dagger} = \gamma^0 \gamma^{\mu} \gamma^0\) \(\forall \mu\). Lets check it:
\[
    \begin{aligned}
        \mu = 0) \quad & (\gamma^0)^\dagger = \gamma^0 \gamma^0 \gamma^0 = \gamma^0 = (\gamma^0)^\dagger,   \\
        \mu = i) \quad & (\gamma^i)^\dagger = \gamma^0 \gamma^i \gamma^0 = - \gamma^i = (\gamma^i)^\dagger,
    \end{aligned}
\]
where in the first case we used the idempotency of \(\gamma^0\) and its hermitianity, while in the second case we used the anticommutation relations among gamma matrices.\footnote{Indeed, computing it explicitly from the Clifford algebra we get \(\{\gamma^i,\,\gamma^0\} = 2 \eta^{i0} =0 \implies \gamma^i \gamma^0 = - \gamma^0 \gamma^i\).} So this relation holds for all \(\mu\). Using this we can compute \((S^{\mu \nu})^{\dagger}\) as:
\[
    \begin{aligned}
        (S^{\mu \nu})^{\dagger} & = \frac{1}{4} \left[\gamma^{\mu},\, \gamma^{\nu}\right]^{\dagger} = \frac{1}{4} (\gamma^{\nu})^{\dagger} (\gamma^{\mu})^{\dagger} - \frac{1}{4} (\gamma^{\mu})^{\dagger} (\gamma^{\nu})^{\dagger} \\
                                & = \frac{1}{4} \gamma^0 \gamma^{\nu} \gamma^0 \gamma^0 \gamma^{\mu} \gamma^0 - \frac{1}{4} \gamma^0 \gamma^{\mu} \gamma^0 \gamma^0 \gamma^{\nu} \gamma^0                                           \\
                                & = \gamma^0 \left(\frac{1}{4} [\gamma^{\nu},\, \gamma^{\mu}] \right) \gamma^0 = - \gamma^0 \left(\frac{1}{4} [\gamma^{\mu},\, \gamma^{\nu}] \right) \gamma^0 = - \gamma^0 S^{\mu \nu} \gamma^0,
    \end{aligned}
\]
where we have expanded the commutator and simplified the \((\gamma^0)^2\). So the right relation seems to be
\[
    (S^{\mu \nu})^{\dagger} = \gamma^0 S^{\mu \nu} \gamma^0.
\]
Since we are looking for the scalar building block for the lagrangian, we are lead to follow this result and compute \(S^{\dagger}\):
\[
    S^{\dagger} = e^{\frac{1}{2}\omega_{\mu \nu}(S^{\mu \nu})^{\dagger}} = e^{-\frac{1}{2}\omega_{\mu \nu} \gamma^0 S^{\mu \nu} \gamma^0}.
\]
Let's factor \(\gamma^0\) out again, Taylor expanding the exponential:
\[
    \begin{aligned}
        S^{\dagger} & \sim \mathbb{I}_{4} - \frac{1}{2}\omega_{\mu \nu} \gamma^0 S^{\mu \nu} \gamma^0  + \frac{1}{4}\omega_{\mu \nu}\omega_{\rho \sigma} \gamma^0 S^{\mu \nu} \gamma^0\gamma^0 S^{\rho \sigma} \gamma^0 - \frac{1}{8}\cdots \\
                    & = \gamma^0 \gamma^0 - \frac{1}{2}\omega_{\mu \nu} \gamma^0 S^{\mu \nu} \gamma^0  + \frac{1}{4}\omega_{\mu \nu}\omega_{\rho \sigma} \gamma^0 S^{\mu \nu} \mathbb{I}_4 S^{\rho \sigma} \gamma^0 - \frac{1}{8}\cdots     \\
                    & = \gamma^0 \left(\mathbb{I}_{4} - \frac{1}{2}\omega_{\mu \nu} S^{\mu \nu} + \frac{1}{4}\omega_{\mu \nu}\omega_{\rho \sigma} S^{\mu \nu} S^{\rho \sigma} - \frac{1}{8}\cdots \right) \gamma^0                          \\
                    & = \gamma^0 e^{\frac{1}{2}\omega_{\mu \nu}S^{\mu \nu}} \gamma^0 = \gamma^0 S^{\mu \nu} \gamma^0.
    \end{aligned}
\]
Now we have definitive proof that the right expression for \(S^{\dagger}\) is
\[
    S^{\dagger} = \gamma^0 S^{-1} \gamma^0,
\]
since we already knew that \(S^{-1} = e^{-\frac{1}{2}\omega_{\mu \nu} S^{\mu \nu}}\).

Thus finally we have insight on how to build a lorentz scalar from spinors: instead of \(\psi^{\dagger} \psi\) we can consider the following bilinear
\[
    \overline{\psi}(x)\psi(x),
\]
built using the \textbf{adjoint Dirac Spinor} \(\overline{\psi}(x) = \psi^{\dagger} \gamma^0\). Let's check its transformation properties:
\[
    \begin{aligned}
        \overline{\psi}(x) \psi(x) & \to \overline{\psi}^{\prime}(x^{\prime}) \psi^{\prime}(x^{\prime}) = \psi^{\dagger \prime}(x^{\prime}) \gamma^0 \psi^{\prime}(x^{\prime}) = \psi^{\dagger}(x) S^{\dagger} \gamma^0 S \psi(x) \\
                                   & = \psi^{\dagger}(x) \gamma^0 S^{-1} \gamma^0 \gamma^0 S \psi(x) = \psi^{\dagger}(x) \gamma^0 S^{-1} S \psi(x)                                                                                \\
                                   & = \psi^{\dagger}(x) \gamma^0 \psi(x) = \overline{\psi}(x) \psi(x).
    \end{aligned}
\]
it is indeed lorentz invariant. Thus \(\overline{\psi} \psi\) is the lorentz scalar we will use to build our lagrangian, in particular for the mass term, where there is no derivative involved:
\[
    \mathcal{L}_{mass} = - m \overline{\psi}(x) \psi(x).
\]

Recalling the expression for the KG Lagrangian
\[
    \mathcal{L}_{KG} = \frac{1}{2} \left(\partial_{\mu} \phi \partial^{\mu} \phi - m^2 \phi^2\right),
\]
we have indeed found the analog of the last term: \(m \overline{\psi}(x)\psi(x)\); the first one instead, where the indices has to be contracted into a Lorentz scalar, involves derivatives. Thus we have to look for another spinor bilinear for the kinetic term, preserving Lorentz invariance with derivatives.

\subsection{Building Block for the Kinetic Term}

Considering the following spinor bilinear
\[
    \overline{\psi}(x) \gamma^{\mu} \psi(x)
\]
which is a vector of four components, we have to understand if that is a Lorentz vector or not, So that
\[
    \overline{\psi}^{\prime} \gamma^{\mu} \psi^{\prime} = \Lambda^{\mu}_{\ \nu} \overline{\psi}\gamma^{\nu} \psi.
\]
If this is true, we can contract that \(\mu\) indices in order to obtain a lorentz scalar (invariant)
\[
    \overline{\psi} \gamma^{\mu}\partial_{\mu}\psi.
\]
So that if we verify that \(\overline{\psi} \gamma^{\mu} \psi\) transforms as a lorentz vector, we can use it to build the kinetic term of the lagrangian. Let's check how it transforms:
\[
    \overline{\psi} \gamma^{\mu} \psi \to \overline{\psi}^{\prime} \gamma^{\mu} \psi^{\prime} = \overline{\psi} S^{-1}\gamma^{\mu} S \psi,
\]
where we have used the results of the previous section to write \(\overline{\psi}^{\prime} = (S \psi)^{\dagger} \gamma^0 = \overline{\psi} S^{-1}\). Thus if we now use the infinitesimal form of the lorentz transformation in the foundamental representation of the Lorentz group \(\mathrm{SO}(1,3)\)
\[
    \Lambda = e^{-\frac{i}{2} \omega_{\mu \nu} M^{\mu \nu}},
\]
and in parallel
\[
    S = e^{-\frac{i}{2} \omega_{\mu \nu} \Sigma^{\mu \nu}},
\]
we are studying the \(4\times 4\) matrices realizing the Lorentz transformation, the first from the foundamental representation \(\mathrm{SO}(1,3)\), the second from the Dirac representation; also \(\Sigma^{\mu \nu}\) and \(M^{\mu \nu}\) are generators of lorentz transformation in the Dirac and foundamental representation.

We can even make the parallel substitution of \(S^{\mu \nu} = i \Sigma^{\mu \nu}\) in the foundamental representation, so that
\[
    \mathcal{M}^{\mu \nu} = - i M^{\mu \nu},
\]
so that now it is possible to write both infinitesimal transformations in the same form:
\[
    \begin{aligned}
        \Lambda & = e^{\frac{1}{2} \omega_{\mu \nu}\mathcal{M}^{\mu \nu}} \sim 1 + \frac{1}{2} \omega_{\mu \nu} \mathcal{M}^{\mu \nu} + O(\omega_{\mu \nu}^2), \\
        S       & = e^{\frac{1}{2} \omega_{\mu \nu} S^{\mu \nu}} \sim 1 + \frac{1}{2} \omega_{\mu \nu} S^{\mu \nu} + O(\omega_{\mu \nu}^2).
    \end{aligned}
\]

Now if we stick to the foundamental representation, we have that
\[
    (\mathcal{M}^{\rho \sigma})^{\mu}_{\ \nu} \gamma^{  \nu} = (\eta^{\rho \mu} \delta^{\sigma}_{\ \nu} - \eta^{\sigma \mu} \delta^{\rho}_{\ \nu}) \gamma^{\nu} = \eta^{\rho \mu} \gamma^{\sigma} - \eta^{\sigma \mu} \gamma^{\rho}.
\]
Now we want some kind of relation among the two representations, since this last expression is practically the corresponding of \(\Lambda^{\mu}_{\ \nu} \overline{\psi} \gamma^{\nu} \psi\). The idea is to expand the expression for \(S^{\rho \sigma}\) in order to find a way to relate the previous expression to a commutator with gamma matrices. We have
\[
    \begin{aligned}
        S^{\rho \sigma} & = \frac{1}{4} [\gamma^{\rho},\, \gamma^{\sigma}] = \frac{1}{4} (\gamma^{\rho} \gamma^{\sigma} - \gamma^{\sigma} \gamma^{\rho})                              \\
                        & = \frac{1}{4} ( 2 \gamma^{\rho} \gamma^{\sigma} - \{\gamma^{\rho},\, \gamma^{\sigma}\}) = \frac{1}{2} (\gamma^{\rho} \gamma^{\sigma} - \eta^{\rho \sigma}),
    \end{aligned}
\]
where we have used the clifford algebra to rewrite the commutator in terms of the anticommutator. Now if we compute the following commutator among gamma matrices and the generators in the Dirac representation
\[
    \begin{aligned}
        \left[S^{\rho \sigma},\,\gamma^{\mu}\right] & = \frac{1}{2} \left[ \gamma^{\rho} \gamma^{\sigma},\, \gamma^{\mu} \right] -\frac{1}{2} \eta^{\rho \sigma}\left[\mathbb{I},\, \gamma^{\mu} \right] \\
                                                    & = \frac{1}{2} (\gamma^{\rho} \gamma^{\sigma} \gamma^{\mu} - \gamma^{\mu} \gamma^{\rho} \gamma^{\sigma}) - 0,
    \end{aligned}
\]
and using again the Clifford algebra to compute \(\gamma^{\sigma} \gamma^{\rho} = 2\eta^{\rho \sigma} - \gamma^{\rho} \gamma^{\sigma}\), we get to
\[
    \begin{aligned}
        [S^{\rho \sigma},\,\gamma^{\mu}] & = \frac{1}{2} \gamma^{\rho} \left(2 \eta^{\mu \sigma} - \gamma^{\mu}\gamma^{\sigma}\right) - \frac{1}{2} \left(2 \eta^{\rho \mu} - \gamma^{\rho} \gamma^{\mu} \right) \gamma^{\sigma} \\
                                         & = \eta^{\mu \sigma} \gamma^{\rho} - \eta^{\rho \mu} \gamma^{\sigma} = - (\eta^{\rho \mu} \gamma^{\sigma} - \eta^{\sigma \mu} \gamma^{\rho}).
    \end{aligned}
\]
Recalling in the end the previous computation, it is clear that\footnote{Since the Minkowski metric \(\eta^{\mu \sigma} = \eta^{\sigma \mu}\) is symmetric, we can exchange the indices in the last term.}
\[
    (\mathcal{M}^{\rho \sigma})^{\mu}_{\ \nu} \gamma^{\nu} = - [S^{\rho \sigma},\,\gamma^{\mu}].
\]
And now finally we can compare the two infinitesimal transformations:
\[
    \Lambda^{\mu}_{\ \nu} \gamma^{\nu} = \left[\delta^{\mu}_{\ \nu} + \frac{1}{2} \omega_{\rho \sigma} (\mathcal{M}^{\rho \sigma})^{\mu}_{\ \nu} + O(\omega_{\rho \sigma}^2)\right]\gamma^{\nu} = \gamma^{\mu} -\frac{1}{2} \omega_{\rho \sigma}[S^{\rho \sigma},\,\gamma^{\mu}] + O(\omega_{\rho \sigma}^2),
\]
which is identical to
\[
    \begin{aligned}
        S^{-1} \gamma^{\mu} S & = \left( 1 - \frac{1}{2} \omega_{\rho \sigma} S^{\rho \sigma} + O(\omega_{\rho \sigma}^2) \right) \gamma^{\nu} \left(1 + \frac{1}{2} \omega_{\rho \sigma} S^{\rho \sigma} + O(\omega_{\rho \sigma}^2) \right) \\
                              & = \gamma^{\mu} -\frac{1}{2}\omega_{\rho \sigma} S^{\rho \sigma}\gamma^{\mu} + \frac{1}{2}\omega_{\rho \sigma} \gamma^{\mu} S^{\rho \sigma} + O(\omega_{\rho \sigma}^2)                                        \\
                              & = \gamma^{\mu} - \frac{1}{2} \omega_{\rho \sigma} [S^{\rho \sigma},\, \gamma^{\mu}] + O(\omega_{\rho \sigma}^2).
    \end{aligned}
\]
So we finally proved the identity
\[
    S^{-1} \gamma^{\mu} S = \Lambda^{\mu}_{\ \nu} \gamma^{\nu},
\]
which implies that the Dirac bilinear \(\overline{\psi} \gamma^\mu \psi\) transforms as a Lorentz four vector, since
\[
    \overline{\psi} \gamma^{\mu} \psi \to \overline{\psi}^{\prime} \gamma^{\mu} \psi^{\prime} = \overline{\psi} S^{-1} \gamma^{\mu} S \psi = \Lambda^{\mu}_{\ \nu} \overline{\psi} \gamma^{\mu} \psi.
\]

We can now obtain other Lorentz scalars, vectors or tensors by contracting this four vector with other objects transforming in the vectorial representation of the Lorentz group, for example \(\partial_{\mu}\) or a gauge field \(A_{\mu}\):
\begin{itemize}
    \item \(\overline{\psi} \gamma^{\mu} \partial_\mu \psi\), which is a contraction with the derivative operator, this term represents the kinetic term of the lagrangian and it is a Lorentz scalar;
    \item \(\overline{\psi} \gamma^{\mu}A_\mu \psi\), which is a contraction with a field in vectorial representation, this term represents interaction among spin \(\tfrac12\) particles and spin 1 gauge bosons (e.g. photons);
    \item \(\overline{\psi} \gamma^{\mu \nu} \psi\), where there is a Lorentz tensor, i.e. \(\overline{\psi}^{\prime} \gamma^{\mu \nu} \psi^{\prime} = \Lambda^{\mu}_{\ \rho} \Lambda^{\nu}_{\sigma} \overline{\psi} \gamma^{\rho \sigma} \psi\) (every index transforming accordingly with a Lorentz transformation).
\end{itemize}
We can introduce the \textbf{slash notation} for indicating an object contracted with a gamma matrix:
\[
    \slashed{A} = \gamma^{\mu} A_{\mu}.
\]

We can finally write an expression for the manifestly Lorentz invariant Dirac Lagrangian:
\begin{equation}
    \mathcal{L} = \overline{\psi} i \gamma^{\mu} \partial_{\mu} \psi - m \overline{\psi} \psi = \overline{\psi} \left(i \gamma^{\mu} \partial_{\mu} - m\right) \psi.
    \label{eq:Lorentz_invariant_Dirac_Lagrangian}
\end{equation}

It is worth to notice some features of this lagrangian:
\begin{itemize}
    \item the \(i\) factor ensures the lagrangian to be real: we have to check the realness of the two terms separately. For the mass term we have
          \[
              (\overline{\psi} \psi)^{\dagger} = \psi^{\dagger} (\psi^{\dagger} \gamma^0)^{\dagger}= \psi^{\dagger} \gamma^0 \psi = \overline{\psi} \psi,
          \]
          for hermitianity of \(\gamma^0\). Instead for the kinetic term we have
          \[
              \begin{aligned}
                  (\overline{\psi} \gamma^{\mu} \partial_{\mu} \psi)^{\dagger} & = (\partial_{\mu} \psi)^{\dagger} (\gamma^{\mu})^{\dagger} \overline{\psi}^{\dagger}                                                 \\
                                                                               & = (\partial_{\mu} \psi)^{\dagger} (\gamma^{\mu})^{\dagger}\gamma^0 \psi = (\partial_{\mu} \psi)^{\dagger} \gamma^0 \gamma^{\mu} \psi \\
                                                                               & = \partial_{\mu} (\psi^{\dagger} \gamma^0 \gamma^{\mu} \psi) - \psi^{\dagger} \gamma^0 \gamma^{\mu} \partial_{\mu} \psi              \\
                                                                               & = - \psi^{\dagger} \gamma^0 \gamma^{\mu} \partial_{\mu} \psi= - \overline{\psi} \gamma^{\mu} \partial_{\mu} \psi.
              \end{aligned}
          \]
          where we have used the usual trick of integrating by parts and then neglecting the boundary term computed at infinite times and distances, since this lagrangian is integrated to obtain the action. As we can see the kinetic term is anti-hermitian, so multiplying it by \(i\) we get a hermitian expression, ensuring the lagrangian to be real.
    \item Since the action is dimensionless, we can deduce the mass dimension of the field \(\psi\):
          \[
              \begin{aligned}
                  [S] = 0, \quad [\mathrm{d}^4 x]=-4 \quad \implies [\mathcal{L}] = 4, \\
                  [\partial_\mu] = 1,\quad [m]= 1 \quad \implies [\psi]=[\overline{\psi}] = \frac{3}{2}.
              \end{aligned}
          \]
          So this field has a different mass dimention from KG scalar field, which had: \([\psi]=1\).
    \item KG Lagrangian contains two derivatives \(\partial_\mu \psi \partial^{\mu} \psi\) in the kinetic term, while Dirac's only one \(\overline{\psi} i \gamma^{\mu} \partial_{\mu} \psi\): The KG lagrangian was of the second order, while Dirac's of the first; it changes the order of the equations of motion.
    \item Upon quantization the Dirac theory will describe particles/antiparticles (for whose we needed a complex field for the charge dof, which is our case) with spin \(\tfrac12\) and mass \(m\); in principle we have four complex dof, which are LH/RH and spin up/down (doubled because we need 4 real dof for the particle description and 4 real dof for the antiparticle one).
\end{itemize}

\section{Dirac Equation}

The Dirac Lagrangian is
\[
    \dots
\]
where \(\overline{\psi} = \psi^{\dagger} \gamma^0\)  is the Dirac conjugate. We can find the equations of motion using the Euler-Lagrange equations for fields:
\[
    \dots
\]
notice that \(\frac{\partial}{\partial \overline{\psi}} \mathcal{L} = 0\), in fact \(\dots \) and so \(\dots \) too. Meanwhile the derivatives with respect to \(\psi\) are
\[
    \dots
\]
Hence we get the Dirac equation
\[
    \dots
\]
where the arrow on the derivatives tells us it is to be applied to the left. This is a first order partial differential equation for the spinor field \(\psi(x)\).

We can build such an equation of motion just thatks to the presence of \(\gamma^{\mu}\) which grants lorentz invariance. Dirac equation si a first order differential equation, while for KG we could only get a second order one. Furthermore KG is a scalar equation, while Dirac is a spinor equation (vectorial in spinor space, with 4 components).

Notice that dirac equation mixes components of the spinor, since \(\gamma^{\mu}\) are \(4\times 4\) matrices:
\[
    \psi = \begin{pmatrix}
        \psi_1 \\
        \psi_2 \\
        \psi_3 \\
        \psi_4
    \end{pmatrix}, \quad \gamma^0 = \begin{pmatrix}
        0            & \mathbb{I}_2 \\
        \mathbb{I}_2 & 0
    \end{pmatrix}, \quad \gamma^i = \begin{pmatrix}
        0         & \sigma^i \\
        -\sigma^i & 0
    \end{pmatrix}.
\]
So in components in the spinor space, Dirac equations reads:
\[
    i \begin{pmatrix}
        \partial_t \psi_3 \\
        \partial_t \psi_4 \\
        \partial_t \psi_1 \\
        \partial_t \psi_2
    \end{pmatrix} + i \begin{pmatrix}
        \partial_x \psi_4  \\
        \partial_x \psi_3  \\
        -\partial_x \psi_2 \\
        -\partial_x \psi_1
    \end{pmatrix} \dots
\]
in general it becomes a system of four coupled first order differential equations, mixing the four components of the spinor.

However each component \(\psi_{\alpha}(x)\) satosfy the KG equation, since in effect each components describe a dof of a relativistic particle with mass \(m\): if we multiply Dirac equation by \((i \gamma^{\mu} \partial_{\mu} + m)\) from the left (it is still valid since zero multiplied by anything remains xero), we get
\[
    \begin{aligned}
        (i \gamma^{\mu} \partial_{\mu} + m)(i \gamma^{\nu} \partial_{\nu} - m) \psi(x) = 0                                \\
        \implies \left(-\gamma^{\mu} \gamma^{\nu} \partial_{\mu} \partial_{\nu} - m^2\right) \psi(x) = 0                  \\
        \implies \left(-\frac{1}{2}\{\gamma^{\mu},\,\gamma^{\nu}\} \partial_{\mu} \partial_{\nu} - m^2\right) \psi(x) = 0 \\
        \implies \left(-\eta^{\mu \nu} \partial_{\mu} \partial_{\nu} - m^2\right) \psi(x) = 0                             \\
        \implies (\Box + m^2) \psi(x) = 0.
    \end{aligned}
\]
In terms of matrices we have used the clifford algebra to simplify the product of gamma matrices. So each component of the dirac spinor satisfies the KG equation, thus \((\Box + m^2) \psi_{\alpha} (x) = 0\) \(\forall \alpha = 1,\,2,\,3,\,4\).

\subsection{Chiral Spinors}

Chirality means that Dirac representation \((\tfrac12,\,0) \oplus (0,\,\tfrac12)\) can be decomposed into two irreducible representations, \textbf{Weyl representation}, of the Lorentz group. Weyl or Chiral spinors are two-component objects (complex dof) with different transformation properties:
\begin{itemize}
    \item \textbf{Left-Handed} weyl spinors: \(\psi_L \sim (\tfrac12,\,0)\), transform under \(S_L\) only;
    \item \textbf{Right-Handed} weyl spinors: \(\psi_R \sim (0,\,\tfrac12)\), transform under \(S_R\) only.
\end{itemize}

We can write the Dirac spinor as a combination of two weyl spinors:
\[
    \psi_D = \begin{pmatrix}
        \psi_L^{(w)} \\
        \psi_R^{(w)}
    \end{pmatrix}, \quad \psi_L^{(w)} \xleftrightarrow{\text{Parity}} \psi_R^{(w)}.
\]

\paragraph{Chirality operator.} In order to project out the two chiral components from a Dirac spinor we can introduce the chirality operator
\[
    \gamma^5 = \dots
\]
using which we can define projectors able to select the desired chiral component from the Dirac spinor:
\[
    \begin{aligned}
        \psi_L = \\
        \psi_R =.
    \end{aligned}
\]
Thus the dirac spinor is the sum of the two chiral components \(\psi_D = \psi_L + \psi_R\) and the projectors satisfy the usual properties:
\[
    \begin{dcases}
        P_L^2 = P_L \\
        P_L^{\dagger} = P_L
    \end{dcases} \quad \dots
\]
Note that the eigenvalues of \(\gamma^5\) are \(\pm 1\), so the chirality operator measures the chirality of a spinor:
\[
    \dots
\]
We will see how chirality is related to helicity in the massless limit.

\paragraph{Lagrangian and Chirality.}To understand better, let's write the dirac lagrangian in terms of chiral components (which is very useful for statistical field theory and the standard model, also for understanding the massless limit):
\begin{enumerate}
    \item aa
          \[
              \begin{aligned}
                  \overline{\psi} \gamma^{\mu} \psi = \psi^{\dagger} \gamma^0 \gamma^{\mu} \left(\psi_L + \psi_R\right) = \dots
              \end{aligned}
          \]

          Now exploiting hermitianity of the gamma matrices we can write
          \[
              \overline{\psi} \gamma^{\mu} \psi = \overline{\psi}_R \gamma^{\mu} \psi_R + \overline{\psi}_L \gamma^{\mu} \psi_L.
          \]
          This found vector current does not mix chiral components, its parity invariant; this is not good for electroweak interactions and interpretation \(\mathrm{SU}(2)_L\).

    \item aa
          \[
              \overline{\psi} \gamma^{\mu} \gamma^{5} \psi = \dots
          \]
          same steps as before, but now we find another current, called \textbf{axial vector current}: it changes sign under parity and not good again for EW interpretation.

    \item aa
          \[
              \frac{1}{2} (V-A) = \overline{\psi} \dots
          \]
          V-A current violates parity, and only LH components enter the interarction: it is a good candidate for EW interactions.

    \item mass term
          \[
              \overline{\psi} \psi 0 = \psi^{\dagger} \gamma^0 (P_R^2 + P_L^2) \psi = \dots =
          \]
          with respect to before we dont have \(\gamma^{\mu}\), so we have to put a minus while writing in terms of chiral components: mass term mixes LH and RH components:
          \[
              \dots
          \]
\end{enumerate}

Now we have everything to write the dirac lagrangian in terms of chiral components:
\[
    \mathcal{L} = \overline{\psi}_L i \gamma^{\mu} \partial_{\mu} \psi_L + \overline{\psi}_R i \gamma^{\mu} \partial_{\mu} \psi_R - m \left(\overline{\psi}_L \psi_R + \overline{\psi}_R \psi_L\right).
\]
We can see that in the massless limit the two chiral components decouple, and we get two independent Weyl equations for each chiral component: this is important, since kinetic terms evolves independently the two chiral components, while the mass term allow us to perform a boost and mix the two chiralities: since the particle is massive there will be frames where the particle is RH and others where it is seen as LH.

If you take an electron whith both the chiral components, only the LH component will interact weakly, while the RH will not; but since the electron is massive you can always boost to a frame where the electron appears as RH, so both components are needed to describe a massive fermion. When we see only the RH component, the electron will not seem to interact weakly.

\paragraph{Dirac in Weyl components.}
If we write the Dirac equation in terms of chiral components we get:
\[
    \dots
\]
which is a system of two coupled first order coupled differential equations
\[
    \begin{dcases}
        \dots \\
        \dots \\
    \end{dcases}
\]
In the massless limit the two equations decouple and we get two Weyl equations:
\[
    \begin{dcases}
        \dots \\
        \dots \\
    \end{dcases}
\]
and we call this last set of equations the \textbf{Weyl equations} for massless fermions.
Now it is clearer the meaning of Left and Right-Handed Weyl spinors: in terms of operators we have
\[
    i \partial_t = \hat{H},\quad -i \nabla = \hat{\mathbf{p}}, \quad \bs{\sigma} = \hat{\mathbf{S}},
\]
so that we can see the last two equations as
\[
    \dots
\]
Now, dropping the hat notation for operators (so that we can use it to indicate versors), we find clearly that the weyl components are eigenstate of the \textbf{helicity}:
\[
    \begin{dcases}
        \dots \\
        \dots
    \end{dcases}
\]
where helicity operator is defined as \(\mathbf{S} \cdot \mathbf{p}\); since the eigenvalues are \(\pm 1\), we can affirm that in the massless limit chirality and helicity coincide: they have same eigenstates and eigenvalues.

If the neutrinos were massless, only the LH component would exist, since only that interacts weakly; but since neutrinos have a small mass, both chiral components exist, even if the RH component has never been observed (it interacts only gravitationally, so it is very difficult to detect it).

We now want to prove that under parity we can pass from one weyl component to the other:
\[
    \psi_L^{(w)} \xleftrightarrow{\text{Parity}} \psi_R^{(w)}.
\]
Let's start from the Dirac equation:
\[
    \begin{aligned}
        \dots
    \end{aligned}
\]
which, given the action of the parity:
\[
    \dots
\]
then transforms the Dirac equation as
\[
    \dots
\]
But how does \(\psi\) transform under parity?
\[
    \dots
\]
but gamma matrices satisfy
\[
    \dots
\]
so we gave
\[
    \dots
\]
[...] sofi [...]

and we find finally that
\[
    \psi^{\prime} = \begin{pmatrix}
        \dots
    \end{pmatrix} = \begin{pmatrix}
        0            & \mathbb{I}_2 \\
        \mathbb{I}_2 & 0
    \end{pmatrix} \begin{pmatrix}
        \psi_L^{(w)} \\
        \psi_R^{(w)}
    \end{pmatrix},
\]
so that
\[
    \begin{dcases}
        \psi_L^{\prime (w)} = \psi_R^{(w)}, \\
        \psi_R^{\prime (w)} = \psi_L^{(w)}.
    \end{dcases}
\]

------

\subsection{Solutions of the Dirac Equation}

Each component of the dirac spinor \(\psi(x)\) satisfies the KG equation, so we can look for plane wave solutions of the form
\[
    \psi_{\alpha}(x) = u_{\alpha}(\mathbf{p}) e^{-i p_\mu x^{\mu}} = u_{\alpha}(\mathbf{p}) e^{-i (E_{\mathbf{p}} t - \mathbf{p} \cdot \mathbf{x})},
\]
where \(\alpha = 1,\,2,\,3,\,4\) and \(u_{\alpha}(\mathbf{p})\) are complex coefficients depending on the momentum \(\mathbf{p}\) (it's a 4-component vector). Plugging this ansatz into the Dirac equation we get
\[
    (i \gamma^{\mu} \partial_{\mu} - m) \psi(x) = 0, \quad \implies \quad (\gamma^{\mu} p_{\mu} - m) u(\mathbf{p}) = 0.
\]
which in spinorial representation reads
\[
    \left[ \begin{pmatrix}
            0 & 1 \\
            1 & 0
        \end{pmatrix} p_0 + \begin{pmatrix}
            0         & \sigma^i \\
            -\sigma^i & 0
        \end{pmatrix} p_i - m \begin{pmatrix}
            1 & 0 \\
            0 & 1
        \end{pmatrix}\right] u(\mathbf{p}) = \begin{pmatrix}
        -m                         & p^{\mu} \sigma_{\mu} \\
        p^{\mu} \bar{\sigma}_{\mu} & -m
    \end{pmatrix} u(\mathbf{p}) = 0,
\]
where we have defined \(\sigma^{\mu} = (1,\,\sigma^i)\) and \(\bar{\sigma}^{\mu} = (1,\,-\sigma^i)\). Writing the spinor \(u(\mathbf{p})\) in terms of its two-component chiral parts
\[
    u(\mathbf{p}) = \begin{pmatrix}
        u_L(\mathbf{p}) \\
        u_R(\mathbf{p})
    \end{pmatrix},
\]
we get the system of equations
\[
    \begin{dcases}
        (p^{\mu} \sigma_{\mu}) u_R(\mathbf{p}) = m u_L(\mathbf{p}), \\
        (p^{\mu} \bar{\sigma}_{\mu}) u_L(\mathbf{p}) = m u_R(\mathbf{p}) .
    \end{dcases}
\]
It's a system of two coupled equations for the two chiral components of the spinor. We can solve for one component in terms of the other; for example, solving for \(u_R(\mathbf{p})\) from the second equation and plugging it into the first, we get
\[
    (p^{\mu} \sigma_{\mu})(p^{\nu} \bar{\sigma}_{\nu}) = (p_0 + p_i \sigma^i)(p_0 - p_j \sigma^j) = p_0^2 - - p_i p_j \sigma^i \sigma^j = p_0^2 - \mathbf{p}^2 = m^2,
\]
where we have used the clifford algebra of the Pauli matrices \(\sigma^i \sigma^j = \delta^{ij} + i \epsilon^{ijk} \sigma^k\) and the relativistic dispersion relation \(p_0^2 - \mathbf{p}^2 = m^2\). Thus we can write
\[
    \begin{aligned}
        u_L(\mathbf{p})                             & = A (p^{\mu} \sigma_{\mu}) \chi,                             \\
        (p^{\mu} \bar{\sigma}_{\mu}) u_L(\mathbf{p} & = (p^{\mu} \bar{\sigma}_{\mu}) (p^{\mu} \sigma_{\mu}) A \chi \\
                                                    & = m^2 A \chi = m u_R(\mathbf{p})                             \\
        \implies u_R(\mathbf{p})                    & = m A \chi,
    \end{aligned}
\]
and similarly for \(u_L(\mathbf{p})\):
\[
    \begin{aligned}
        u_R(\mathbf{p})                       & = B (p^{\mu} \bar{\sigma}_{\mu}) \xi,                       \\
        (p^{\mu} \sigma_{\mu}) u_R(\mathbf{p} & = (p^{\mu} \sigma_{\mu}) (p^{\mu} \bar{\sigma}_{\mu}) B \xi \\
                                              & = m^2 B \xi = m u_L(\mathbf{p})                             \\
        \implies u_L(\mathbf{p})              & = m B \xi,
    \end{aligned}
\]
where \(\chi\) and \(\xi\) are arbitrary two-component spinors and \(A\), \(B\) are normalization constants. Thus the general solution for \(u(\mathbf{p})\) can be written as
\[
    u(\mathbf{p}) = A \begin{pmatrix}
        (p^{\mu} \sigma_{\mu}) \chi \\
        m \chi
    \end{pmatrix},
\]
where now we want to symmetrize the solution by choosing \(A = \frac{1}{m}\) and \(\chi = \sqrt{p^{\mu} \bar{\sigma}_{\mu}} \xi\), where \(\xi\) is a constant two-component spinor respecting
\[
    \xi^{\dagger} \xi = 1.
\]
Thus we get the final expression for the \textbf{positive frequency solution} of the Dirac equation:
\[
    u_L(\mathbf{p}) = \dots
\]
\[
    u_R (\mathbf{p}) = \dots
\]
\[
    \psi(x) = \dots
\]

We can also find \textbf{negative frequency solutions} of the Dirac equation by considering
\[
    \psi(x) = v(\mathbf{p}) e^{i p_{\mu} x^{\mu}},
\]
where \(v(\mathbf{p})\) satisfies
\[
    (\gamma^{\mu} p_{\mu} + m) v(\mathbf{p}) = \begin{pmatrix}
        m                          & p^{\mu} \sigma_{\mu} \\
        p^{\mu} \bar{\sigma}_{\mu} & m
    \end{pmatrix} v(\mathbf{p}) = 0,
\]
and following the same steps as before we get
\[
    v(\mathbf{p}) = C \begin{pmatrix}
        \sqrt{p^{\mu} \sigma_{\mu}} \eta \\
        -\sqrt{p^{\mu} \bar{\sigma}_{\mu}} \eta
    \end{pmatrix}.
\]

If we were to apply the Hamiltonian on these ansatzs we would get
\[
    \hat{H} \psi(x) = i \partial_t (u(\mathbf{p}) e^{-i p_{\mu} x^{\mu}}) =,
\]
\[
    \hat{H} \psi(x) = i \partial_t (v(\mathbf{p}) e^{i p_{\mu} x^{\mu}}) =.
\]
Thus \(u(\mathbf{p})\) are positive energy solutions while \(v(\mathbf{p})\) are negative energy solutions of the Dirac equation.

\begin{example}
    In the rest frame of a massive particle
    \[
        \mathbf{p} = 0, \quad E_{\mathbf{p}} = p^0 = m.
    \]
    Thus the positive frequency solutions read
    \[
        u(\mathbf{0}) = \begin{pmatrix}
            \sqrt{p^{\mu} \sigma_\mu} \xi \\
            \sqrt{p^{\mu} \bar{\sigma}_\mu} \xi
        \end{pmatrix} = \sqrt{m} \begin{pmatrix}
            \xi \\
            \xi
        \end{pmatrix},
    \],
    so that the plane wave solution is
    \[
        \psi(x) = \sqrt{m} \begin{pmatrix}
            \xi \\
            \xi
        \end{pmatrix} e^{-i m t}.
    \]

    Now recalling the Lorentz transformation for spinors
    \[
        \psi^{\prime \, \alpha}(x^{\prime}) = S(\Lambda)^{\alpha}_{\beta} \psi^{\beta}(x), \quad S(\Lambda)^{\alpha}_{\beta} = (e^{-\frac{i}{2} \omega_{\mu \nu} S^{\mu \nu}})^{\alpha}_{\beta},
    \]
    where the algebra generators of the Lorentz transformation are
    \[
        S^{\mu \nu} = \frac{i}{4} [\gamma^{\mu},\, \gamma^{\nu}]
    \]
    with parameters \(\omega_{\mu \nu}\) depending on the boost/rotation we are performing. For a pure spatial rotation we have
    \[
        S^{ij} = -\frac{i}{2} \begin{pmatrix}
            \sigma^k & 0        \\
            0        & \sigma^k
        \end{pmatrix}, \quad (i \neq j)
    \]
    where \(k\) is the axis of rotation, and the three parameters
    \[
        \omega_{ij} = - \epsilon_{ijk} \theta^k, \quad (i \neq j)
    \]
    are related to the rotation angles around the three spatial axes. Thus the spinor transformation under a spatial rotation reads
    \[
        e^{\frac{1}{2} \omega_{ij} S^{ij}} = \begin{pmatrix}
            e^{i \frac{\theta^k}{2} \sigma^k} & 0                                 \\
            0                                 & e^{i \frac{\theta^k}{2} \sigma^k}
        \end{pmatrix},
    \]
    and if we apply this transformation to the rest-frame spinor we get
    \[
        \psi(x) = \sqrt{m} \begin{pmatrix}
            \xi \\
            \xi
        \end{pmatrix} e^{-i m t} \to  \psi^{\prime}(x^{\prime}) = \begin{pmatrix}
            e^{i \frac{\theta^k}{2} \sigma^k} & 0                                 \\
            0                                 & e^{i \frac{\theta^k}{2} \sigma^k}
        \end{pmatrix} \psi(x) = \sqrt{m} \begin{pmatrix}
            e^{i \frac{\theta^k}{2} \sigma^k} \xi \\
            e^{i \frac{\theta^k}{2} \sigma^k} \xi
        \end{pmatrix} e^{-i m t},
    \]
    so that both chiral components transform in the same way under spatial rotations, as expected since in the rest frame chirality and helicity are not defined
    \[
        \xi \to \xi^{\prime} = e^{i \frac{\theta^k}{2} \sigma^k} \xi.
    \]
    This is the representation of the standard \(\mathrm{SO}(2)\) transformation for spin \(\tfrac12\) objects
    \[
        \mathbf{S} = \frac{1}{2} \hbar \bs{\sigma}.
    \]
    We are considering particles at rest with spin \(\tfrac12\), so we can choose the basis where the spin is aligned along the \(z\)-axis:
    \[
        \xi_+ = \begin{pmatrix}
            1 \\
            0
        \end{pmatrix}, \quad \xi_- = \begin{pmatrix}
            0 \\
            1
        \end{pmatrix},
    \]
    which are eigenstates of the spin operator \(S_z = \frac{\hbar}{2} \sigma^3\) with eigenvalues \(\pm \frac{\hbar}{2}\).
\end{example}

\begin{example}
    Now consider a particle with spin up along the \(z\)-axis moving with momentum \(\mathbf{p} = (0,\,0,\,p_z)= (0,\,0,\,p)\) along the \(z\)-axis
    \[
        p^{\mu} = (E_{\mathbf{p}},\,0,\,0,\,p_z), \quad E_{\mathbf{p}} = \sqrt{p_z^2 + m^2} = E.
    \]
    We have the positiv frequency solution as
    \[
        \psi(x) = \begin{pmatrix}
            \sqrt{p^{\mu} \sigma_{\mu}} \xi_+ \\
            \sqrt{p^{\mu} \bar{\sigma}_{\mu}} \xi_+
        \end{pmatrix}e^{-i p_{\mu} x^{\mu}},
    \]
    where computing the square roots we get
    \[
        \psi(x) = \begin{pmatrix}
            \begin{pmatrix}
                \sqrt{E + p} & 0            \\
                0            & \sqrt{E - p}
            \end{pmatrix}\begin{pmatrix}
                             1 \\
                             0
                         \end{pmatrix} \\
            \begin{pmatrix}
                \sqrt{E - p} & 0            \\
                0            & \sqrt{E + p}
            \end{pmatrix}\begin{pmatrix}
                             1 \\
                             0
                         \end{pmatrix}
        \end{pmatrix} e^{-i (E t - p z)} = \begin{pmatrix}
            \sqrt{E + p} \\
            0            \\
            \sqrt{E - p} \\
            0
        \end{pmatrix} e^{-i (E t - p z)}.
    \]
    where we have a right handed helicity state since both momentum and spin are aligned along the \(z\)-axis, but both chiral components since for massive particles chirality and helicity do not coincide. Now we can consider the massless limit \(m \to 0\), so that \(E = p\) and we get that the left chiral component vanishes
    \[
        \psi(x) = \begin{pmatrix}
            \sqrt{2 p} \\
            0          \\
            0          \\
            0
        \end{pmatrix} e^{-i p (t - z)},
    \]
    which is a purely right-handed spinor, as expected since in the massless limit chirality and helicity coincide:
    \[
        \begin{aligned}
            \psi_L (x) & = 0,                        \\
            \psi_R (x) & = \sqrt{2 p} \begin{pmatrix}
                                          1 \\
                                          0
                                      \end{pmatrix}.
        \end{aligned}
    \]
\end{example}

\subsection{Useful Formulae}

We want to introduce some useful formulae for manipulating Dirac spinors and gamma matrices, in order to comprehend better the structure of the quantum theory.

\paragraph{Inner product.}
We have to define a basis \(\xi^n\) and \(\eta^n\) for the two-component spinors used to build the positive and negative frequency solutions of the Dirac equation (with \(n = 1,\,2\)):
\[
    \xi^{n \, \dagger} \xi^m = \delta^{nm}, \quad \eta^{n \, \dagger} \eta^m = \delta^{nm},
\]
with trivial example
\[
    \xi^1 = \begin{pmatrix}
        1 \\
        0
    \end{pmatrix}, \quad \xi^2 = \begin{pmatrix}
        0 \\
        1
    \end{pmatrix}.
\]
Thus solutions of the Dirac equation can be labeled with this index and distinguished into different spin states:
\[
    u^{n\,\dagger}(\mathbf{p}) u^m(\mathbf{p}) = \begin{pmatrix}
        \xi^{n\, \dagger} \sqrt{p^{\mu} \sigma_\mu} & \xi^{n\, \dagger} \sqrt{p^{\mu} \bar{\sigma}_\mu}
    \end{pmatrix} \begin{pmatrix}
        \sqrt{p^{\mu} \sigma_\mu} \xi^m \\
        \sqrt{p^{\mu} \bar{\sigma}_\mu} \xi^m
    \end{pmatrix}
\]
which can be computed as
\[
    \begin{aligned}
        u^{n\,\dagger}(\mathbf{p}) u^m(\mathbf{p}) & = \xi^{n\, \dagger} (p^{\mu} \sigma_\mu) \xi^m + \xi^{n\, \dagger} (p^{\mu} \bar{\sigma}_\mu) \xi^m \\
                                                   & = \dots = 2 p_0 \xi^{n\, \dagger} \xi^m = 2 E_{\mathbf{p}} \delta^{nm}.
    \end{aligned}
\]
Instead if we have
\[
    \overline{u}^{n}(\mathbf{p}) u^m(\mathbf{p}) = \begin{pmatrix}
        \xi^{n\, \dagger} \sqrt{p^{\mu} \sigma_\mu} & \xi^{n\, \dagger} \sqrt{p^{\mu} \bar{\sigma}_\mu}
    \end{pmatrix} \begin{pmatrix}
        0 & 1 \\
        1 & 0
    \end{pmatrix} \begin{pmatrix}
        \sqrt{p^{\mu} \sigma_\mu} \xi^m \\
        \sqrt{p^{\mu} \bar{\sigma}_\mu} \xi^m
    \end{pmatrix},
\]
and we get (\(\overline{u}^n = u^{n\,\dagger} \gamma^0\))
\[
    \begin{aligned}
        u^{n\,\dagger}(\mathbf{p}) \gamma^0 u^m(\mathbf{p}) & = \xi^{n\, \dagger} (p^{\mu} \sigma_\mu) \xi^m + \xi^{n\, \dagger} (p^{\mu} \bar{\sigma}_\mu) \xi^m \\
                                                            & = \dots = 2 m \xi^{n\, \dagger} \xi^m = 2 m \delta^{nm}.
    \end{aligned}
\]
Now we can summarize these two results as
\[
    \begin{dcases}
        u^{n\,\dagger}(\mathbf{p}) u^m(\mathbf{p}) = 2 E_{\mathbf{p}} \delta^{nm}, \\
        \overline{u}^n (\mathbf{p}) u^m(\mathbf{p}) = 2 m \delta^{nm}.
    \end{dcases}
\]
Similarly for the negative frequency solutions we have\TODO{Compute these explicitly.}
\[
    \begin{dcases}
        v^{n\,\dagger}(\mathbf{p}) v^m(\mathbf{p}) = 2 E_{\mathbf{p}} \delta^{nm}, \\
        \overline{v}^n (\mathbf{p}) v^m(\mathbf{p}) = -2 m \delta^{nm}.
    \end{dcases}
\]
Now we can compute the mixed products
\[
    \overline{u}^n (\mathbf{p}) v^m(\mathbf{p}) = \dots = 0,
\]
and similarly
\[
    \overline{v}^n (\mathbf{p}) u^m(\mathbf{p}) = \dots = 0.
\]
Thus we have the orthogonality relations, which do not mix positive and negative frequency solutions.
We have also
\[
    u^{n\,\dagger}(\mathbf{p}) v^m(-\mathbf{p}) = \dots = 0,
\]
and similarly
\[
    v^{n\,\dagger}(\mathbf{p}) u^m(-\mathbf{p}) = \dots = 0.
\]

\paragraph{Outer products.}
We can compute the outer products of the spinors, starting from the positive frequency solutions:
\[
    \sum_{n} u^n(\mathbf{p}) \overline{u}^n(\mathbf{p}) = \sum_n \begin{pmatrix}
        \sqrt{p^{\mu} \sigma_\mu} \xi^n \\
        \sqrt{p^{\mu} \bar{\sigma}_\mu} \xi^n
    \end{pmatrix} \begin{pmatrix}
        \xi^{n\, \dagger} \sqrt{p^{\mu} \sigma_\mu} & \xi^{n\, \dagger} \sqrt{p^{\mu} \bar{\sigma}_\mu}
    \end{pmatrix} \begin{pmatrix}
        0 & 1 \\
        1 & 0
    \end{pmatrix},
\]
where we have used the definition of \(\overline{u}^n\). Computing the sum over the basis we get
\[
    \sum_{n} u^n(\mathbf{p}) \overline{u}^n(\mathbf{p}) = \sum_{n} \begin{pmatrix}
        \sqrt{p^{\mu} \sigma_\mu} \xi^n \xi^{n\, \dagger} \sqrt{p^{\mu} \overline{\sigma}_\mu}            & \sqrt{p^{\mu} \sigma_\mu} \xi^n \xi^{n\, \dagger} \sqrt{p^{\mu} \sigma_\mu}            \\
        \sqrt{p^{\mu} \overline{\sigma}_\mu} \xi^n \xi^{n\, \dagger} \sqrt{p^{\mu} \overline{\sigma}_\mu} & \sqrt{p^{\mu} \overline{\sigma}_\mu} \xi^n \xi^{n\, \dagger} \sqrt{p^{\mu} \sigma_\mu}
    \end{pmatrix} = (\gamma^{\mu} p_{\mu} + m).
\]
Now if we use the completeness relation for the basis
\[
    \sum_{n} \xi^n \xi^{n\, \dagger} = \sum_{n} \begin{pmatrix}
        1 \\
        0
    \end{pmatrix} \begin{pmatrix}
        1 & 0
    \end{pmatrix} = \begin{pmatrix}
        1 & 0 \\
        0 & 1
    \end{pmatrix} = \mathbb{I}_2,
\]
then we find
\[
    \sum_{n} u^n(\mathbf{p}) \overline{u}^n(\mathbf{p}) = \begin{pmatrix}
        \sqrt{(p \sigma)(p \overline{\sigma})} & (p \sigma)                             \\
        (p \overline{\sigma})                  & \sqrt{(p \overline{\sigma})(p \sigma)}
    \end{pmatrix} = \begin{pmatrix}
        m                          & p^{\mu} \sigma_{\mu} \\
        p^{\mu} \bar{\sigma}_{\mu} & m
    \end{pmatrix} = \gamma^{\mu} p_{\mu} + m \mathbb{I}_4.
\]
Similarly for the negative frequency solutions we have
\[
    \sum_{n} v^n(\mathbf{p}) \overline{v}^n(\mathbf{p}) = \dots = \gamma^{\mu} p_{\mu} - m \mathbb{I}_4.
\]