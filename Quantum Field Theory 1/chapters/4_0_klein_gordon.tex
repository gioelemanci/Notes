\chapter{Klein Gordon Theory}

We have now developed the Hamiltonian formulation of \textbf{Classical Field Theory}, where the dynamical degrees of freedom are represented by the fields \(\phi_a(\mathbf{x}, t)\) and their conjugate momenta \(\pi_a(\mathbf{x}, t)\), which satisfy canonical Poisson brackets.
The next step is to promote this classical description to a \textbf{Quantum Field Theory (QFT)}, in which fields become operators acting on a Hilbert space, and classical observables become non-commuting operators.

\paragraph{From Classical Mechanics to Quantum Mechanics.}
The passage from classical to quantum mechanics provides a useful template.
In \textbf{Classical Mechanics (CM)}, a system with \(n\) degrees of freedom is described by canonical variables \((q_i, p_i)\), whose dynamics is determined by Hamilton’s equations:
\[
    \dot{q}_i = \frac{\partial H}{\partial p_i}, \qquad
    \dot{p}_i = -\frac{\partial H}{\partial q_i},
\]
and whose Poisson brackets encode the symplectic structure:
\[
    \{ q_i, p_j \} = \delta_{ij}.
\]

In \textbf{Quantum Mechanics (QM)}, the canonical variables are promoted to Hermitian operators \((\hat{q}_i, \hat{p}_i)\) on a Hilbert space, with the replacement
\[
    \{ \cdot, \cdot \} \;\longrightarrow\; \frac{1}{i\hbar} [\cdot, \cdot],
\]
leading to the canonical commutation relations
\[
    [\hat{q}_i, \hat{p}_j] = i\hbar \, \delta_{ij}.
\]
Observables evolve according to the Heisenberg equation
\[
    i\hbar \frac{d\hat{A}}{dt} = [\hat{A}, \hat{H}].
\]

\paragraph{From Classical Field Theory to Quantum Field Theory.}
The same canonical quantization idea extends to continuous systems.
Here we adopt the \textbf{Schr\"odinger picture}, in which operators are constant in time while states evolve.

In Classical Field Theory, the generalized coordinates are fields \(\phi_a(\mathbf{x}, t)\), and the conjugate momenta are
\[
    \pi_a(\mathbf{x}, t) = \frac{\partial \mathcal{L}}{\partial \dot{\phi}_a(\mathbf{x}, t)},
\]
with canonical Poisson brackets
\[
    \{ \phi_a(\mathbf{x}, t), \pi_b(\mathbf{y}, t) \} = \delta_{ab} \, \delta^{(3)}(\mathbf{x} - \mathbf{y}).
\]
The Hamiltonian functional
\[
    H[\phi, \pi] = \int d^3x \, \mathcal{H}(\phi_a, \pi_a, \nabla\phi_a)
\]
generates the time evolution of the fields.

In \textbf{Quantum Field Theory}, the fields become operators on a Hilbert (or Fock) space:
\[
    \phi_a(\mathbf{x}, t) \;\longrightarrow\; \hat{\phi}_a(\mathbf{x}),
    \qquad
    \pi_a(\mathbf{x}, t) \;\longrightarrow\; \hat{\pi}_a(\mathbf{x}),
\]
and the Poisson brackets are replaced by equal-time commutation relations:
\[
    [\hat{\phi}_a(\mathbf{x}), \hat{\pi}_b(\mathbf{y})] = i\hbar \, \delta_{ab}\, \delta^{(3)}(\mathbf{x}-\mathbf{y}), \qquad
[\hat{\phi}_a(\mathbf{x}), \hat{\phi}_b(\mathbf{y})] = [\hat{\pi}_a(\mathbf{x}), \hat{\pi}_b(\mathbf{y})] = 0,
\]
defining operators acting on the Fock space which can be written in terms of \textit{ladder operators} (annihilation and creation operators) acting on the vacuum.

In the Schr\"odinger picture, the states \(\ket{\psi(t)}\) evolve according to the Schr\"odinger equation:
\[
    i\hbar \frac{d}{dt} \ket{\psi(t)} = \hat{H} \ket{\psi(t)},
\]
where the wave functional \(\psi[\phi, t]\) encodes the probability amplitude for the field configuration \(\phi(\mathbf{x})\) at time \(t\).

Solving the theory requires diagonalizing the Hamiltonian \(\hat{H}\), which is generally difficult due to the infinite number of degrees of freedom and possible discrete or continuous field labels.
An important exception is \textbf{free theories} with quadratic Lagrangians, where the equations of motion are linear and can be solved exactly (e.g., the continuum limit of an elastic string).

Thus, canonical quantization provides a direct bridge from the Hamiltonian structure of classical fields to the operator formalism of quantum theory, where particles naturally arise as quantized excitations of the underlying fields.

\section{The Klein-Gordon Field as a Set of Harmonic Oscillators}

The simplest relativistic free field theory is the Klein-Gordon theory for a real scalar field \(\psi(\mathbf{x},t)\). Its \textbf{Lagrangian density} is
\begin{equation}
    \mathcal{L} = \frac{1}{2} \partial_\mu \psi \, \partial^{\mu} \psi - \frac{1}{2} m^2 \psi^2,
    \label{eq:KG_Lagrangian}
\end{equation}
and the corresponding \textbf{classical equations of motion}, obtained via the Euler-Lagrange procedure, read
\begin{equation}
    (\Box + m^2)\psi(x) = 0, \qquad \Box = \partial_\mu \partial^{\mu}.
    \label{eq:KG_motion_equations}
\end{equation}

\begin{remark}
    Although the Klein-Gordon equation appears as a single field equation in coordinate space, it is effectively a \emph{coupled system} of infinitely many degrees of freedom: the Laplacian term \(-\nabla^2 \psi\) couples the field at different spatial points.
    By performing a spatial Fourier transform, one can \emph{decouple} the system, obtaining independent harmonic oscillators for each momentum mode \(\mathbf{k}\).
    In momentum space, each mode evolves independently and can be quantized separately.
\end{remark}

We perform a spatial Fourier transform of the field:
\[
    \psi(t,\mathbf{x}) = \int \frac{\mathrm{d}^3 \mathbf{p}}{(2\pi)^3} \, e^{i \mathbf{p}\cdot \mathbf{x}} \tilde{\psi}(t,\mathbf{p}).
\]
The Klein-Gordon equation in momentum space becomes
\[
    \left[\frac{\partial^2}{\partial t^2} + \omega_{\mathbf{p}}^2 \right] \tilde{\psi}(t, \mathbf{p}) = 0,
    \qquad \omega_{\mathbf{p}} = \sqrt{|\mathbf{p}|^2 + m^2}.
\]
This shows explicitly that the system reduces to an infinite set of \emph{decoupled harmonic oscillators}, one for each momentum mode \(\mathbf{p}\).

\subsection{Canonical Quantization in Momentum Space}

Following the analogy with the quantum harmonic oscillator in quantum mechanics, each momentum mode can be quantized independently.
Recall that for a single harmonic oscillator with mass \(m=1\) and frequency \(\omega\), the canonical operators satisfy
\[
    \hat{q} = \frac{1}{\sqrt{2\omega}} (\hat{a} + \hat{a}^{\dagger}), \qquad
    \hat{p} = -i \sqrt{\frac{\omega}{2}} (\hat{a} - \hat{a}^{\dagger}),
\]
where \(\hat{a}, \hat{a}^\dagger\) are the annihilation and creation operators, obeying \([\,\hat{a}, \hat{a}^\dagger\,] = 1\).

We can generalize this construction to the field operators. A general solution to the Klein-Gordon equation (satisfying reality of the field) can be written as
\[
    \hat{\psi}(\mathbf{x}) = \int \frac{\mathrm{d}^3 \mathbf{p}}{(2\pi)^3} \frac{1}{\sqrt{2 \omega_{\mathbf{p}}}} \Big[ \hat{a}_{\mathbf{p}} e^{i \mathbf{p}\cdot \mathbf{x}} + \hat{a}_{\mathbf{p}}^\dagger e^{-i \mathbf{p}\cdot \mathbf{x}} \Big],
\]
with conjugate momentum operator
\[
    \hat{\pi}(\mathbf{x}) = \int \frac{\mathrm{d}^3 \mathbf{p}}{(2\pi)^3} \, (-i) \sqrt{\frac{\omega_{\mathbf{p}}}{2}} \Big[ \hat{a}_{\mathbf{p}} e^{i \mathbf{p}\cdot \mathbf{x}} - \hat{a}_{\mathbf{p}}^\dagger e^{-i \mathbf{p}\cdot \mathbf{x}} \Big],
\]
where the ladder operators acts on the Fock space creating and destroying particles of momentum \(\mathbf{p}\) (or \(\mathbf{q}\)).
Furthermore, they satisfy the canonical commutation relations:
\[
    [\,\hat{a}_{\mathbf{p}}, \hat{a}_{\mathbf{q}}\,] = [\,\hat{a}_{\mathbf{p}}^\dagger, \hat{a}_{\mathbf{q}}^\dagger\,] = 0, \qquad
    [\,\hat{a}_{\mathbf{p}}, \hat{a}_{\mathbf{q}}^\dagger\,] = (2\pi)^3 \delta^3(\mathbf{p} - \mathbf{q}).
\]

Our solutions now have to respect similar canonical commutation relations. In fact we know that:
\[
    \begin{aligned}
        [\,\hat{\psi}(\mathbf{x}),\hat{\pi}(\mathbf{y})\,]  & = i \delta^3 (\mathbf{x} - \mathbf{y}), \text{ while}    \\
        [\,\hat{\psi}(\mathbf{x}),\hat{\psi}(\mathbf{y})\,] & = [\,\hat{\pi}(\mathbf{x}),\hat{\pi}(\mathbf{y})\,] = 0.
    \end{aligned}
\]
Let us show this results:
\[
    \begin{aligned}
        [\,\hat{\psi}(\mathbf{x}),\hat{\pi}(\mathbf{y})\,] & = \int \frac{\mathrm{d}^3 \mathbf{p}\mathrm{d}^3 \mathbf{q}}{(2\pi)^6} \, \frac{(-i)}{2} \sqrt{\frac{\omega_{\mathbf{q}}}{\omega_{\mathbf{p}}}} \left[ \,\hat{a}_{\mathbf{p}} e^{i \mathbf{p}\cdot \mathbf{x}} + \hat{a}_{\mathbf{p}}^\dagger e^{-i \mathbf{p}\cdot \mathbf{x}}\, , \,\hat{a}_{\mathbf{q}} e^{i \mathbf{q}\cdot \mathbf{y}} - \hat{a}_{\mathbf{q}}^\dagger e^{-i \mathbf{q}\cdot \mathbf{y}} \, \right]  \\
                                                           & = \int \frac{\mathrm{d}^3 \mathbf{p}\mathrm{d}^3 \mathbf{q}}{(2\pi)^6} \, \frac{(-i)}{2} \sqrt{\frac{\omega_{\mathbf{q}}}{\omega_{\mathbf{p}}}} \left[ - [\,\hat{a}_{\mathbf{p}}, \hat{a}_{\mathbf{q}}^\dagger\,] e^{i (\mathbf{p} \cdot \mathbf{x} - \mathbf{q} \cdot \mathbf{y})} + [\,\hat{a}^\dagger_{\mathbf{p}}, \hat{a}_{\mathbf{q}}\,] e^{-i (\mathbf{p} \cdot \mathbf{x} - \mathbf{q} \cdot \mathbf{y})}\right] \\
                                                           & = \int \frac{\mathrm{d}^3 \mathbf{p}\mathrm{d}^3 \mathbf{q}}{(2\pi)^6} \, \frac{(-i)}{2} \sqrt{\frac{\omega_{\mathbf{q}}}{\omega_{\mathbf{p}}}} (2\pi)^3\left[ - \delta^3(\mathbf{p} - \mathbf{q})e^{i (\mathbf{p} \cdot \mathbf{x} - \mathbf{q} \cdot \mathbf{y})} + (- \delta^3(\mathbf{q} - \mathbf{p}))e^{-i (\mathbf{p} \cdot \mathbf{x} - \mathbf{q} \cdot \mathbf{y})}\right]                                     \\
                                                           & = \int \frac{\mathrm{d}^3 \mathbf{p}}{(2\pi)^3} \, \frac{i}{2} \sqrt{\frac{\omega_{\mathbf{q}}}{\omega_{\mathbf{p}}}} \left[ \delta^3(\mathbf{p} - \mathbf{q})\left(e^{i (\mathbf{p} \cdot \mathbf{x} - \mathbf{q} \cdot \mathbf{y})} + e^{-i (\mathbf{p} \cdot \mathbf{x} - \mathbf{q} \cdot \mathbf{y})}\right) \right]                                                                                                \\
                                                           & = \frac{i}{2} \int \frac{\mathrm{d}^3 \mathbf{p}}{(2\pi)^3} \, \sqrt{\frac{\omega_{\mathbf{p}}}{\omega_{\mathbf{p}}}} \left( e^{i \mathbf{p} \cdot (\mathbf{x} - \mathbf{y})} + e^{-i \mathbf{p} \cdot (\mathbf{x} - \mathbf{y})}\right)                                                                                                                                                                                 \\
                                                           & =  \frac{i}{2} \int \frac{\mathrm{d}^3 \mathbf{p}}{(2\pi)^3} \,  e^{i \mathbf{p} \cdot (\mathbf{x} - \mathbf{y})} + \frac{i}{2} \int \frac{\mathrm{d}^3 \mathbf{p}}{(2\pi)^3} \, e^{i \mathbf{p} \cdot (\mathbf{y} - \mathbf{x})}                                                                                                                                                                                        \\
                                                           & = \frac{i}{2} \delta^3(\mathbf{x} - \mathbf{y}) + \frac{i}{2} \delta^3(\mathbf{y} - \mathbf{x}) = i \delta^3(\mathbf{x} - \mathbf{y}),
    \end{aligned}
\]
where in the first step we simplified the null commutators of the ladder operators, while in the last steps we recognized the integral representation of the Dirac delta function
\[
    \delta^3(\mathbf{x}) = \int \frac{\mathrm{d}^3 \mathbf{p}}{(2\pi)^3} \,  e^{\pm i \mathbf{p}\cdot\mathbf{x}}
\]
which is by definition the Fourier transform of a constant and it respects the property
\[
    \delta(\alpha \mathbf{x}) = \frac{\delta(\mathbf{x})}{\vert \alpha \vert}, \quad \to \quad \delta(-\mathbf{x}) = \delta(\mathbf{x}).
\]
\TODO{compute the other two commutators.}



\section{Two Real Klein Gordon Fields}

 [...]
\[
    \hat{Q} = \dots
\]
Note that there is an ambiguity: \(\hat{Q}\) is an hermitian operator (good, physical quantity), but if it is conserved and I take \(c_1 \hat{Q} + c_2\) is also conserved:
\[
    \ddt c_1 \hat{Q} + c_2 = c_2 \ddt \hat{Q} = 0,
\]
so \(c_1\) controls \textit{units} in which \(Q\) is measured; \(c_2\) instead is just a constant which can be removed by normal ordering (similar to what we did for the Hamiltonian and the vacuum energy).

\[
    \begin{aligned}
        \hat{Q}\ket{0}         & = \int \frac{\mathrm{d}^3 \mathbf{p}}{(2\pi)^3} \left[ \dots \right] = 0,   \\
        \hat{\tilde{Q}}\ket{0} & = c_2, \iff \bra{0}\hat{\tilde{Q}} \ket{0} = c_2 \langle 0|0 \rangle = c_2, \\
        :\hat{\tilde{Q}}:      & = \hat{\tilde{Q}} - \bra{0}\hat{\tilde{Q}} \ket{0}.
    \end{aligned}
\]

Let us now determine the spectrum of the theory: let us define:
\[
    \begin{aligned}
        \hat{a}_{\pm,\,\mathbf{p}}           & = \\
        \hat{a}_{\pm,\,\mathbf{p}}^{\dagger} & = \\
    \end{aligned}
\]

\textbf{Exercise:} show the commutators:
\[
    fi l l l l l
\]

Let us now consider the state \(\ket{s}\) with charge \(q_s \longrightarrow \hat{\tilde{Q}}\ket{s} = q_s \ket{s}\), then:
\[
    \hat{\tilde{Q}}(\hat{a}_{\pm,\,\mathbf{p}}^{\dagger}\ket{s}) = \dots,
\]
thus we recognize \(\hat{a}_{\pm,\,\mathbf{p}}^{\dagger}\) to be a \textit{ladder operator}\footnote{Operators that let you go up and down in states referring to a particular eigenvalue.} for \(\hat{\tilde{Q}}\); they are ladder operators even for \(\hat{H}\) and \(\hat{P}\) (linear combinations of \(\hat{a}_{(1),\,\mathbf{p}}^{\dagger}\) and \(\hat{a}_{(2),\,\mathbf{p}}^{\dagger}\) are ladder operators for \(\hat{H}\) and \(\hat{P}\)).

Our goal is now to find \textit{common eigenstates} of \(\hat{H}\) and \(\hat{P}\) and \(\hat{\tilde{Q}}\):
\[
    \begin{aligned}
        \hat{\tilde{Q}} \ket{0} = 0, \\
        \ket{s_n^{\pm}} = \prod_{i=1}^n  \hat{a}_{\pm,\,\mathbf{p}}^{\dagger} \ket{0}
    \end{aligned}
\]
so we have \(n\) particle states with positive (\(\ket{s_n^{+}}\)) or negative charge (\(\ket{s_n^{-}}\)) \(\pm nq\):
\[
    \begin{aligned}
        \hat{H} \ket{s_n^{\pm}}         & =   \\
        \hat{P} \ket{s_n^{\pm}}         & =   \\
        \hat{N} \ket{s_n^{\pm}}         & =   \\
        \hat{\tilde{Q}} \ket{s_n^{\pm}} & = .
    \end{aligned}
\]
This symmetry exists only for \(m_1 = m_2 = m\), otherwise we have no internal symmetries and cannot find a Noether's charge.

\(\hat{\tilde{Q}}\) is the \textbf{electric charge} and it can describe \textbf{particle/antiparticle} with positive energy: \(\ket{s_n^{\pm}}\) describes the sign of the charge: \(\ket{s_n^{+}}\) for particles and \(\ket{s_n^{-}}\) for antiparticles.

Note that a real KG field can only describe a \(Q=0\) particle
\[
    Q = \int \dots
\]
if \(\psi_1=\psi_2=\psi\), we need at least 2 degrees of freedom if we wnat to have a more complete description of a picture with particles and antiparticles with nonzero electric charge: \textbf{complex KG field}.

The Lagrangian can be rewritten also as follows
\[
    \mathcal{L} =\partial_\mu \psi^* \partial^{\mu} \psi\dots
\]
with
\[
    \psi(x) = \frac{1}{\sqrt{2}}(\phi_1(x)+i \phi_2(x)).
\]
Thus we have a transformation of the form:
\[
    \psi(x) \xrightarrow[\mathrm{U}(1)]{\text{global}} \psi^{\prime}(x) = e^{i \theta} \psi(x)
\]
where it's important that \(\theta \neq \theta(x)\) so that
\[
    paper1
\]
hence \(\mathcal{L} \to \mathcal{L} ^{\prime} =\mathcal{L}\) and for infinitesimal transformations we have
\[
    \begin{aligned}
        paper2
    \end{aligned}
\]
and the total charge (which is conserved) can be computed as\TODO{compute it}
\[
    paper3.
\]