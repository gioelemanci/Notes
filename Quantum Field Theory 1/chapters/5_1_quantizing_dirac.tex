\section{Quantizing Dirac Theory}
We aim to quantize the theory as we did for the scalar KG field, promoting the classical fields to operators acting on a suitable Hilbert space. Starting from Lagrangian formalism, we aim to compute the conjugate momenta, Hamiltonian density, and Hamiltonian operator for the Dirac field.

We know \(\pi = \frac{\partial \mathcal{L}}{\partial \dot{\psi}}\), with the Dirac lagrangian density
\[
    \mathcal{L} = \overline{\psi} (i \gamma^{\mu} \partial_{\mu} - m) \psi,
\]
we have
\[
    \pi =\overline{\psi} i \gamma^0 = i \psi^{\dagger}.
\]

Now, exactely as for the scalar field, we have to impose canonical commutation relations between the field and its conjugate momentum at equal times, only now we have to take into account that the Dirac field is a spinor, so we have to consider each component separately; so in Schrödinger picture we impose
\[
    \begin{aligned}
        [\psi_{\alpha}(\mathbf{x}),\, \pi_{\beta}(\mathbf{y})]  & = i \delta_{\alpha \beta} \delta^{(3)}(\mathbf{x} - \mathbf{y}), \\
        [\psi_{\alpha}(\mathbf{x}),\, \psi_{\beta}(\mathbf{y})] & = [\pi_{\alpha}(\mathbf{x}),\, \pi_{\beta}(\mathbf{y})] = 0.
    \end{aligned}
\]
We will see how this choice leads to inconsistencies, so we will have to modify it. In fact we will obtain either negative energy states or negative norm states, both unphysical; thus we will have to impose anticommutation relations instead of commutation relations, in order to obtain a consistent quantum theory for spin \(\tfrac12\) particles. This is related to the spin-statistics theorem, which states that particles with half-integer spin are fermions and must obey Fermi-Dirac statistics, while particles with integer spin are bosons and obey Bose-Einstein statistics.

\subsection{Quantizing with canonical commutation relations}

Let us write the general solution of the Dirac equation in terms of linear combinations of positive and negative frequency solutions:
\[
    \psi(\mathbf{x}) = \sum_{s=1}^2 \int \frac{d^3 p}{(2 \pi)^3} \frac{1}{\sqrt{2 E_{\mathbf{p}}}} \left( \hat{b}_{\mathbf{p}}^s u^s(\mathbf{p}) e^{i \mathbf{p} \cdot \mathbf{x}} + \hat{c}^{s\, \dagger}_{\mathbf{p}} v^s(\mathbf{p}) e^{-i \mathbf{p} \cdot \mathbf{x}} \right),
\]
and its adjoint
\[
    \psi^{\dagger}(\mathbf{x}) = \sum_{s=1}^2 \int \frac{d^3 p}{(2 \pi)^3} \frac{1}{\sqrt{2 E_{\mathbf{p}}}} \left( \hat{b}_{\mathbf{p}}^{s\, \dagger} u^{s\, \dagger}(\mathbf{p}) e^{-i \mathbf{p} \cdot \mathbf{x}} + \hat{c}^s_{\mathbf{p}} v^{s\, \dagger}(\mathbf{p}) e^{i \mathbf{p} \cdot \mathbf{x}} \right).
\]
We understand that the operators \(\hat{b}_{\mathbf{p}}^s\) and \(\hat{b}_{\mathbf{p}}^{s\, \dagger}\) are annihilation and creation operators for particles with momentum \(\mathbf{p}\) and spin \(s\), while \(\hat{c}_{\mathbf{p}}^s\) and \(\hat{c}_{\mathbf{p}}^{s\, \dagger}\) are annihilation and creation operators for antiparticles with momentum \(\mathbf{p}\) and spin \(s\). So we can interpret the negative frequency solutions as antiparticles, as we did for the scalar field.\footnote{Of course this field can consider the charge degrees of freedom, since it is a complex field.}

If we now impose the canonical commutation relations on the field operators, we find that the commutation relations for the creation and annihilation operators are
\[
    \begin{aligned}
        [\hat{b}_{\mathbf{p}}^s,\, \hat{b}_{\mathbf{q}^{r}\, \dagger}] & = (2 \pi)^3 \delta^{(3)}(\mathbf{p} - \mathbf{q}) \delta^{s r},  \\
        [\hat{c}_{\mathbf{p}}^s,\, \hat{c}_{\mathbf{q}}^{r\, \dagger}] & = -(2 \pi)^3 \delta^{(3)}(\mathbf{p} - \mathbf{q}) \delta^{s r},
    \end{aligned}
\]
with all other commutators vanishing.
The minus sign in the second commutation relation is problematic, since it leads to negative norm states when we compute the norm of single antiparticle states, as we will see.

It is easy to compute that from these commutation relations we can derive the first relations presented in this section \(\left[\hat{\psi}_{\alpha}(\mathbf{x}),\, \hat{\psi}_{\beta}^\dagger(\mathbf{y})\right] = \delta_{\alpha \beta}\delta^{(3)}(\mathbf{x} - \mathbf{y})\):\footnote{The second term being proportional to \(\hat{\pi} \) as we have seen.}
\[
    \begin{aligned}
        \left[\hat{\psi}(\mathbf{x}),\, \hat{\psi}^\dagger(\mathbf{y})\right] & = \sum_{r,s} \int \frac{\mathrm{d}^3 p \mathrm{d}^3 q}{(2 \pi)^6} \frac{1}{2 \sqrt{E_{\mathbf{p}} E_{\mathbf{q}}} } \left( [\hat{b}_{\mathbf{p}}^s,\, \hat{b}_{\mathbf{q}}^{r\, \dagger}] u^s(\mathbf{p}) u^{r\, \dagger}(\mathbf{q}) e^{i (\mathbf{p} \cdot \mathbf{x} - \mathbf{q} \cdot \mathbf{y})} + [\hat{c}_{\mathbf{p}}^{s\, \dagger},\, \hat{c}_{\mathbf{q}}^r] v^s(\mathbf{p}) v^{r\, \dagger}(\mathbf{q}) e^{-i (\mathbf{p} \cdot \mathbf{x} - \mathbf{q} \cdot \mathbf{y})} \right) \\
                                                                              & = \sum_{s} \int \frac{\mathrm{d}^3 p}{(2 \pi)^3} \frac{1}{2 E_{\mathbf{p}}} \left( u^s(\mathbf{p}) u^{s\, \dagger}(\mathbf{p}) e^{i \mathbf{p} \cdot (\mathbf{x} - \mathbf{y})} - v^s(\mathbf{p}) v^{s\, \dagger}(\mathbf{p}) e^{-i \mathbf{p} \cdot (\mathbf{x} - \mathbf{y})} \right)                                                                                                                                                                                                         \\
                                                                              & = \dots                                                                                                                                                                                                                                                                                                                                                                                                                                                                                         \\
    \end{aligned}
\]
where we can use the outer product formulae for the spinors
\[
    \begin{dcases}
        \sum_{s=1}^2 u^s(\mathbf{p}) \overline{u}^s(\mathbf{p}) = \gamma^{\mu} p_{\mu} + m, \\
        \sum_{s=1}^2 v^s(\mathbf{p}) \overline{v}^s(\mathbf{p}) = \gamma^{\mu} p_{\mu} - m,
    \end{dcases}
\]
to obtain
\[
    \dots = \int \frac{\mathrm{d}^3 p}{(2 \pi)^3} \left[ \left(p_0 \gamma^0 + p_i \gamma^i + m\right) e^{i \mathbf{p} \cdot (\mathbf{x} - \mathbf{y})} + \left(p_0 \gamma^0 + p_i \gamma^i - m\right) e^{-i \mathbf{p} \cdot (\mathbf{x} - \mathbf{y})} \right],
\]
where now we can perform a change of variable on the second term \(\mathbf{p} \to -\mathbf{p}\), so that \(p_i \to -p_i\) and \(p_0 \to p_0\), and we get
\[
    \dots = \int \frac{\mathrm{d}^3 p}{(2 \pi)^3} \left[ 2 p_0 \gamma^0 e^{i \mathbf{p} \cdot (\mathbf{x} - \mathbf{y})} \right] = \gamma^0 \delta^{(3)}(\mathbf{x} - \mathbf{y}),
\]\TODO{there was a gamma0 missing from the last step of aligned which simplifies in the end.}
which is exactly what we wanted to prove.

If we consider the vacuum state \(\ket{0}\) such that
\[
    \hat{c}^r_{\mathbf{p}} \ket{0} = 0, \quad \forall \mathbf{p},\, r,
\]
and
\[
    \hat{c}^{s\, \dagger}_{\mathbf{p}} \ket{0} = \ket{\mathbf{p},\, s},
\]
but if we now compute the norm of this one-particle state we get
\[
    \bra{\mathbf{p},\,s}\ket{\mathbf{p},\, s} = \bra{0} \hat{c}^s_{\mathbf{p}} \hat{c}^{s\, \dagger}_{\mathbf{p}} \ket{0} = \bra{0} [\hat{c}^s_{\mathbf{p}},\, \hat{c}^{s\, \dagger}_{\mathbf{p}}] \ket{0} = - (2 \pi)^3 \delta^{(3)}(0) \bra{0}\ket{0} < 0,
\]
which is negative, leading to negative norm states, which are unphysical. Thus we have to modify our initial assumption, since maybe we have made a wrong choice in imposing the dagger on the creation operator; maybe the creation operator should be defined without the dagger, so that
\[
    \hat{c}^r_{\mathbf{p}} \ket{0} \neq 0 \to \hat{c}^r_{\mathbf{p}} \ket{0} = \ket{\mathbf{p},\, r},
\]
and if we swap the dagger in the commutation relations we get the opposite sign indeed:
\[
    \left[\hat{c}^{r,\,\dagger}_{\mathbf{p}},\, \hat{c}^s_{\mathbf{q}} \right] = - (2 \pi)^3 \delta^{(3)}(\mathbf{p} - \mathbf{q}) \delta^{r s}.
\]
But this interpretation is not consistent, since when we consider the energy solutions they will be negative. As we will see, the solution to this problem is to impose anticommutation relations instead of commutation relations, so that we can have positive norm states for both particles and antiparticles.

Let us ignore the probability interpretation for now and compute the Hamiltonian operator for the Dirac field, to show that this is not the solution of the problem. Starting from the Hamiltonian density
\[
    \begin{aligned}
        \mathcal{H} & = \pi \dot{\psi} - \mathcal{L} = i \psi^{\dagger} \partial_t \psi - \overline{\psi} (i \gamma^{\mu} \partial_{\mu} - m) \psi = \dots \\
                    & = \overline{\psi} ( - i \gamma^i \partial_i + m) \psi,
    \end{aligned}
\]
where we have used the Dirac equation to simplify the expression. Now we can write the Hamiltonian operator by promoting the fields to operators and using the
\[
    \begin{aligned}
        ( - i \gamma^i \partial_i + m) \psi(x) & = \sum_{s} \int \frac{\mathrm{d}^3 p}{(2 \pi)^3} \frac{1}{\sqrt{2 E_{\mathbf{p}}}} \left( \hat{b}_{\mathbf{p}}^s ( - i \gamma^i \partial_i + m) u^s(\mathbf{p}) e^{i \mathbf{p} \cdot \mathbf{x}} + \hat{c}_{\mathbf{p}}^{s\, \dagger} ( - i \gamma^i \partial_i + m) v^s(\mathbf{p}) e^{-i \mathbf{p} \cdot \mathbf{x}} \right) \\
                                               & = \sum_{s} \int \frac{\mathrm{d}^3 p}{(2 \pi)^3} \frac{1}{\sqrt{2 E_{\mathbf{p}}}} \left( \hat{b}_{\mathbf{p}}^s (-\gamma^{i} p_{i} + m) u^s(\mathbf{p}) e^{i \mathbf{p} \cdot \mathbf{x}} + \hat{c}_{\mathbf{p}}^{s\, \dagger} (\gamma^{i} p_{i} + m) v^s(\mathbf{p}) e^{-i \mathbf{p} \cdot \mathbf{x}} \right),
    \end{aligned}
\]
where we can now use the dirac equations for the spinors
\[
    (\gamma^{\mu} p_{\mu} - m) u^s(\mathbf{p}) = 0, \quad (\gamma^{\mu} p_{\mu} + m) v^s(\mathbf{p}) = 0,
\]
to get
\[
    \begin{aligned}
        (-\gamma^{i} p_{i} + m) u^s(\mathbf{p}) & = \gamma^{0} p_{0} u^s(\mathbf{p}) = E_{\mathbf{p}} \gamma^{0} u^s(\mathbf{p}),    \\
        (\gamma^{i} p_{i} + m) v^s(\mathbf{p})  & = -\gamma^{0} p_{0} v^s(\mathbf{p}) = - E_{\mathbf{p}} \gamma^{0} v^s(\mathbf{p}).
    \end{aligned}
\]
Thus if we insert these results back into the expression for \(( - i \gamma^i \partial_i + m) \psi(x)\) we get
\[
    \dots = \sum_{s} \int \frac{\mathrm{d}^3 p}{(2 \pi)^3} \sqrt{\frac{E_{\mathbf{p}}}{2}} \gamma^{0} \left( \hat{b}_{\mathbf{p}}^s u^s(\mathbf{p}) e^{i \mathbf{p} \cdot \mathbf{x}} - \hat{c}_{\mathbf{p}}^{s\, \dagger} v^s(\mathbf{p}) e^{-i \mathbf{p} \cdot \mathbf{x}} \right),
\]
and finally we can write the Hamiltonian operator as
\[
    \begin{aligned}
        \hat{H} & = \int \mathrm{d}^3 x \, \hat{\overline{\psi}}(x) \left( - i \gamma^i \partial_i + m \right) \hat{\psi}(x)                                                                                                                                                                                                                                                                                                                                                                                                                                                     \\
                & = \sum_{r,s} \int \frac{\mathrm{d}^{3}\mathbf{x} \mathrm{d}^3 p \mathrm{d}^3 q}{(2 \pi)^6} \frac{1}{2}\sqrt{\frac{E_{\mathbf{p}}}{ E_{\mathbf{q}}}} \left( \hat{b}_{\mathbf{q}}^{r\, \dagger} \overline{u}^r(\mathbf{q}) e^{-i \mathbf{q} \cdot \mathbf{x}} + \hat{c}_{\mathbf{q}}^r \overline{v}^r(\mathbf{q}) e^{i \mathbf{q} \cdot \mathbf{x}} \right) \gamma^{0} \gamma^0 \left( \hat{b}_{\mathbf{p}}^s u^s(\mathbf{p}) e^{i \mathbf{p} \cdot \mathbf{x}} - \hat{c}_{\mathbf{p}}^{s\, \dagger} v^s(\mathbf{p}) e^{-i \mathbf{p} \cdot \mathbf{x}} \right),
    \end{aligned}
\]
since \(\overline{\psi} = \psi^{\dagger} \gamma^0\) and \(\gamma^0 \gamma^0 = \mathbb{I}\). Now we can perform the integration over \(\mathbf{x}\) which gives us a delta function \((2 \pi)^3 \delta^{(3)}(\mathbf{p} - \mathbf{q})\) or \((2 \pi)^3 \delta^{(3)}(\mathbf{p} + \mathbf{q})\) depending on the exponentials, so that we get
\[
    \dots = \sum_{r,s} \int \frac{\mathrm{d}^3 p \mathrm{d}^3 \mathbf{q}}{(2 \pi)^3} \frac{1}{2} \sqrt{\frac{E_{\mathbf{p}}}{E_{\mathbf{q}}}} \left( \hat{b}_{\mathbf{q}}^{r\, \dagger} \hat{b}_{\mathbf{p}}^s u^{r,\,\dagger}(\mathbf{q}) u^s(\mathbf{p}) \delta^{(3)}(\mathbf{p} -\mathbf{q}) + \dots  \right),
\]
where we can now use the delta functions to perform the integration over \(\mathbf{q}\):\footnote{Remember that we are working with four dimensional vectors, but the Hamiltonian is either a number or an operator: in the end we ought to have products among row vectors (daggered) and column ones.}
\[
    \dots = \sum_{r,s} \int \frac{\mathrm{d}^3 p}{(2 \pi)^3} \frac{1}{2} \left( \hat{b}_{\mathbf{p}}^{r\, \dagger} \hat{b}_{\mathbf{p}}^s u^{r,\,\dagger}(\mathbf{p}) u^s(\mathbf{p}) - \hat{c}^r_{\mathbf{p}}\hat{c}^{s,\,\dagger}_{\mathbf{p}} v^{s,\,\dagger}(\mathbf{p}) v^s(\mathbf{p}) - bc + cb [\dots]  \right).
\]
If we remember the inner product formulae for the spinors
\[
    \begin{dcases}
        u^{n\,\dagger}(\mathbf{p}) u^m(\mathbf{p}) = 2 E_{\mathbf{p}} \delta^{nm}, \\
        v^{n\,\dagger}(\mathbf{p}) v^m(\mathbf{p}) = 2 E_{\mathbf{p}} \delta^{nm}, \\
        u^{n\,\dagger}(\mathbf{p}) v^m(-\mathbf{p}) = \dots = 0,                   \\
        v^{n\,\dagger}(\mathbf{p}) u^m(-\mathbf{p}) = \dots = 0,
    \end{dcases}
\]
we note that the last two terms can be simplified since they are proportional to the same inner product with different sign on the momentum, so they vanish. Thus we get
\[
    \hat{H} = \sum_{s} \int \frac{\mathrm{d}^3 p}{(2 \pi)^3} E_{\mathbf{p}} \left( \hat{b}_{\mathbf{p}}^{s\, \dagger} \hat{b}_{\mathbf{p}}^s - \hat{c}^s_{\mathbf{p}} \hat{c}^{s\, \dagger}_{\mathbf{p}} \right).
\]
After this long computation, we chack the physical consistency of this Hamiltonian operator: we are spanning spins and momenta, then we have an energy term \(E_{\mathbf{p}} = \sqrt{\mathbf{p}^2 + m^2} > 0\) multiplied by number operators for particles and antiparticles. The problem is that the c-particle term \(c^s_{\mathbf{p}}c^{s,\,\dagger}\) has a minus sign, associeted than to negative energy contribution. Thus we have fixed the norm problem, but now we have negative energy states, which are also unphysical.

We try to solve it by reordering the operators in the antiparticle term, so that we have \(- \hat{c}^s_{\mathbf{p}} \hat{c}^{s\, \dagger}_{\mathbf{p}} = - \hat{c}^{s\, \dagger}_{\mathbf{p}} \hat{c}^s_{\mathbf{p}} - \left[commutator\right]\) such that
\[
    \hat{H} = \sum_{s} \int \frac{\mathrm{d}^3 p}{(2 \pi)^3} E_{\mathbf{p}} \left( \hat{b}_{\mathbf{p}}^{s\, \dagger} \hat{b}_{\mathbf{p}}^s - \hat{c}^{s\, \dagger}_{\mathbf{p}} \hat{c}^s_{\mathbf{p}} + (2 \pi)^3 \delta^{(3)}(0) \right),
\]
where we have used the commutation relations to swap the operators. Now we have positive energy contributions only, but we have introduced an infinite constant term proportional to \(\delta^{(3)}(0)\), which is divergent, but we can remove it by redefining the zero point of energy and normal ordering, as we did for the scalar field.

Let us verify explicitly that b-type particles have positive energy while c-typ particles have negative energy if we define them with positive norm, by computing the commutator of the Hamiltonian with the creation operators:
\[
    \begin{aligned}
        \left[\hat{H} ,\, b_{\mathbf{p}}^{s,\, \dagger} \right] & = \dots                                         \\
                                                                & = E_{\mathbf{p}} b_{\mathbf{p}}^{s,\, \dagger},
    \end{aligned}
\]
while for the c-type particles we have similarly
\[
    \begin{aligned}
        \left[\hat{H} ,\, c_{\mathbf{p}}^{s,\, \dagger} \right] & = \dots                                         \\
                                                                & = E_{\mathbf{p}} c_{\mathbf{p}}^{s,\, \dagger},
    \end{aligned}
\]
so that both particles have positive energy excitations, but we have to face the negative norm problem again for c-type particles.\TODO{change notation until now so that we dont speak of part and antipart but b-type and c-type particles.}
Now the action of the Hamiltonian on one-particle states is
\[
    \begin{aligned}
        \hat{H} \left( \hat{b}^{s,\,\dagger}_{\mathbf{p}} \ket{0}\right) & = E_{\mathbf{p}} \left( \hat{b}^{s,\,\dagger}_{\mathbf{p}} \ket{0}\right) + \hat{b}^{s,\,\dagger}_{\mathbf{p}} \hat{H} \ket{0} = E_{\mathbf{p}} \left( \hat{b}^{s,\,\dagger}_{\mathbf{p}} \ket{0}\right), \\
        \hat{H} \left( \hat{c}^{s,\,\dagger}_{\mathbf{p}} \ket{0}\right) & = E_{\mathbf{p}} \left( \hat{c}^{s,\,\dagger}_{\mathbf{p}} \ket{0}\right) + \hat{c}^{s,\,\dagger}_{\mathbf{p}} \hat{H} \ket{0} = E_{\mathbf{p}} \left( \hat{c}^{s,\,\dagger}_{\mathbf{p}} \ket{0}\right),
    \end{aligned}
\]
so that both types of particles have positive energy excitations, but we still have the negative norm problem for c-type particles.

If we try quantizing the Dirac field by imposing commutation relations we end up with either negative norm states or negative energy states, both unphysical:
\begin{itemize}
    \item \textbf{b-type} particles are associated to positive frequency solutions, have positive norm and positive energy;
    \item \textbf{c-type} particles are associated to negative frequency solutions, give us problems:
          \begin{enumerate}
              \item if we define them with positive energy states (\(\hat{c}^{s,\,\dagger}_{\mathbf{p}}\) defined as creation operator) they have negative norm: ill-defined Hilbert space;
              \item if we define them with positive norm (\(\hat{c}^s_{\mathbf{p}}\) defined as creation operator) they have negative energy states: energy unbounded from below.\footnote{This is similar to what happened for the classical Dirac equation, where negative frequency solutions had to be reinterpreted as antiparticles to avoid negative energy states; we say unbounded from below because we could create more and more negative energy states by adding more and more c-type particles, leading to an unstable vacuum.}
          \end{enumerate}
\end{itemize}
We thus have to change paradigm of quantization.

\subsection{Quantizing with canonical anticommutation relations}

If we recall what happened for \textbf{bosons} in Klein-Gordon theory, we imposed canonical \textbf{commutation} relations between the field and its conjugate momentum at equal times, leading to commutation relations for creation and annihilation operators
\[
    \ket{\mathbf{p},\,\mathbf{q}} = \hat{a}^{\dagger}_{\mathbf{p}} \hat{a}^{\dagger}_{\mathbf{q}} \ket{0} = \hat{a}^{\dagger}_{\mathbf{q}} \hat{a}^{\dagger}_{\mathbf{p}} \ket{0},
\]
since the creation operators commute
\[
    \left[\hat{a}^{\dagger}_{\mathbf{p}}, \hat{a}^{\dagger}_{\mathbf{q}}\right] = 0.
\]
This means that we can create multiple particles in the same state, leading to \textit{Bose-Einstein statistics}.

We do not find this acceptable for \textbf{fermions}, since they obey the \textit{Pauli exclusion principle}, which states that no two fermions can occupy the same quantum state simultaneously. We need a minus sign when swapping two fermionic operators, thus we have to impose \textbf{canonical anticommutation relations} between the field and its conjugate momentum at equal times, leading to anticommutation relations for creation and annihilation operators:
\[
    \begin{aligned}
        \{\hat{\psi}_{\alpha}(\mathbf{x}),\, \hat{\psi}_{\beta}(\mathbf{y})\} = \{\hat{\psi}_{\alpha}^{\dagger}(\mathbf{x}),\, \hat{\psi}_{\beta}^{\dagger}(\mathbf{y})\} = 0, \\
        \{\psi_{\alpha}(\mathbf{x}),\, \hat{\psi}^{\dagger}_{\beta}(\mathbf{y})\} = \delta_{\alpha \beta} \delta^{(3)}(\mathbf{x} - \mathbf{y}).
    \end{aligned}
\]
This leads to anticommutation relations for creation and annihilation operators:
\[
    \begin{aligned}
        \{\hat{b}_{\mathbf{p}}^s,\, \hat{b}_{\mathbf{q}}^{r\, \dagger}\} & = (2 \pi)^3 \delta^{(3)}(\mathbf{p} - \mathbf{q}) \delta^{s r}, \\
        \{\hat{c}_{\mathbf{p}}^s,\, \hat{c}_{\mathbf{q}}^{r\, \dagger}\} & = (2 \pi)^3 \delta^{(3)}(\mathbf{p} - \mathbf{q}) \delta^{s r},
    \end{aligned}
\]
while all other anticommutators vanish.

Now, as last time, we have to compute the commutator of the Hamiltonian operator
with the creation operators to check if the energy excitations are positive and associated to positive norm states. Repeating the same computation as before, but using anticommutation relations instead of commutation relations, we find
\[
    \hat{H} = \sum_{s} \int \frac{\mathrm{d}^3 \mathbf{p}}{(2 \pi)^3} E_{\mathbf{p}} \left( \hat{b}_{\mathbf{p}}^{s\, \dagger} \hat{b}_{\mathbf{p}}^s - \hat{c}^{s\, \dagger}_{\mathbf{p}} \hat{c}^s_{\mathbf{p}} \right),
\]
where we can now reorder the operators in the c-type particle term introducing a minus sign from the anticommutation relations:
\[
    \hat{H} = \sum_{s} \int \frac{\mathrm{d}^3 \mathbf{p}}{(2 \pi)^3} E_{\mathbf{p}} \left( \hat{b}_{\mathbf{p}}^{s\, \dagger} \hat{b}_{\mathbf{p}}^s + \hat{c}^{s\, \dagger}_{\mathbf{p}} \hat{c}^s_{\mathbf{p}} - (2 \pi)^3 \delta^{(3)}(0) \right),
\]
where now both types of particles have positive energy excitations, if we interpret \(\hat{c}^{\dagger}_{\mathbf{p}}\) as the creation operators and thus the second term as the c-type particle number operator; in the end we have a divergent constant term which we can remove by normal ordering as before.\footnote{This constant term is related to the vacuum energy, which in KG theory came with a plus sign.}
Thus we have finally obtained a consistent quantum theory for spin \(\tfrac12\) particles, by imposing anticommutation relations instead of commutation relations and a normal ordered Hamiltonian defined as
\begin{equation}
    \hat{H} = \sum_{s} \int \frac{\mathrm{d}^3 \mathbf{p}}{(2 \pi)^3} E_{\mathbf{p}} \left( \hat{b}_{\mathbf{p}}^{s\, \dagger} \hat{b}_{\mathbf{p}}^s + \hat{c}^{s\, \dagger}_{\mathbf{p}} \hat{c}^s_{\mathbf{p}} \right).
    \label{eq:Dirac_hamiltonian_normal_ordered}
\end{equation}

We can verify explicitly that both b-type and c-type particles have positive energy excitations, looking at the vacuum state with no particles
\[
    \hat{b}^s_{\mathbf{p}} \ket{0} = 0, \quad \hat{c}^s_{\mathbf{p}} \ket{0} = 0, \quad \forall \mathbf{p},\, s,
\]
and by computing the commutator of the Hamiltonian with the creation operators, with the idea to apply it to one-particle states:
\[
    \begin{aligned}
        \left[\hat{H} ,\, \hat{b}_{\mathbf{p}}^{s,\, \dagger} \right] = E_{\mathbf{p}} \hat{b}_{\mathbf{p}}^{s,\, \dagger}, \quad \left[\hat{H} ,\, \hat{c}_{\mathbf{p}}^{s,\, \dagger} \right] = E_{\mathbf{p}} \hat{c}_{\mathbf{p}}^{s,\, \dagger}, \\
        \left[\hat{H},\, \hat{b}_{\mathbf{p}}^{s}\right] = - E_{\mathbf{p}} \hat{b}_{\mathbf{p}}^{s}, \quad \left[\hat{H},\, \hat{c}_{\mathbf{p}}^{s}\right] = - E_{\mathbf{p}} \hat{c}_{\mathbf{p}}^{s}.
    \end{aligned}
\]
Now on \textbf{one-particle states} we have
\[
    \begin{aligned}
        \ket{\mathbf{p}, s}_b = \hat{b}_{\mathbf{p}}^{s,\, \dagger} \ket{0}, \\
        \ket{\mathbf{p} , s}_c = \hat{c}_{\mathbf{p}}^{s,\, \dagger} \ket{0},
    \end{aligned}
\]
so that the norm of these states is positive:
\[
    \begin{aligned}
        {}_b\bra{\mathbf{q}, r}\ket{\mathbf{p}, s}_b & = \bra{0} \hat{b}_{\mathbf{q}}^r \hat{b}_{\mathbf{p}}^{s,\, \dagger} \ket{0} = \bra{0} \{\hat{b}_{\mathbf{q}}^r,\, \hat{b}_{\mathbf{p}}^{s,\, \dagger}\} \ket{0} = (2 \pi)^3 \delta^{(3)}(0) \bra{0}\ket{0} > 0, \\
        {}_c\bra{\mathbf{q}, r}\ket{\mathbf{p}, s}_c & = \bra{0} \hat{c}_{\mathbf{q}}^r \hat{c}_{\mathbf{p}}^{s,\, \dagger} \ket{0} = \bra{0} \{\hat{c}_{\mathbf{q}}^r,\, \hat{c}_{\mathbf{p}}^{s,\, \dagger}\} \ket{0} = (2 \pi)^3 \delta^{(3)}(0) \bra{0}\ket{0} > 0,
    \end{aligned}
\]
wher the difference with respect to the commutation relation case is the anticommutator used to swap the operators, which does not introduce a minus sign as the commutator did.

Finally, we can compute the action of the Hamiltonian on one-particle states:
\[
    \begin{aligned}
        \hat{H} \left(\hat{b}^{s,\,\dagger}_{\mathbf{p}} \ket{0} \right) & = \left[\hat{H} ,\, \hat{b}^{s,\,\dagger}_{\mathbf{p}} \right] \ket{0} + \hat{b}^{s,\,\dagger}_{\mathbf{p}} \hat{H} \ket{0} = E_{\mathbf{p}} \left(\hat{b}^{s,\,\dagger}_{\mathbf{p}} \ket{0} \right), \\
        \hat{H} \left(\hat{c}^{s,\,\dagger}_{\mathbf{p}} \ket{0} \right) & = \left[\hat{H} ,\, \hat{c}^{s,\,\dagger}_{\mathbf{p}} \right] \ket{0} + \hat{c}^{s,\,\dagger}_{\mathbf{p}} \hat{H} \ket{0} = E_{\mathbf{p}} \left(\hat{c}^{s,\,\dagger}_{\mathbf{p}} \ket{0} \right).
    \end{aligned}
\]
Thus both b-type and c-type particles have positive norm states and positive energy excitations, solving all the problems we had before and leading to a consistent quantum theory for spin \(\tfrac12\) particles.

The question we are now left is: what is the difference between \(b-\)type and \(c-\)type particles? They are both spin \(\tfrac12\) particles with mass \(m\) and positive energy excitations; they are degenerate eigenstates of the Hamiltonian, momentum and spin. The answer is that there has to be another conserved associated to a noether charge symmetry, which distinguishes this two types of particles: this charge is the electric charge, and \(b-\)type particles are associated to particles with charge \(+e\) (like electrons), while \(c-\)type particles are associated to antiparticles with charge \(-e\) (like positrons), as we will see in the next section.

\subsection{Internal Global Symmetry}

We are looking for a conserved charge which distinguishes between \(b-\)type and \(c-\)type particles. To find it, we can search for a global (i.e., spacetime-independent) and internal (i.e., acting on the fields but not on spacetime coordinates) symmetry of the Dirac Lagrangian. It is easy to see that the Dirac Lagrangian is invariant under the global phase transformation
\[
    \psi(x) \to \psi'(x) = e^{-i q \theta} \psi(x), \quad \overline{\psi}(x) \to \overline{\psi}'(x) = e^{i q \theta} \overline{\psi}(x),
\]
where \(q\) is a constant associated to the charge of the field and \(\theta\) is a constant parameter. This is a symmetry since the Lagrangian density
\[
    \mathcal{L} = \overline{\psi}(i \gamma^{\mu} \partial_{\mu} - m) \psi,
\]
depends on \(\psi\) and \(\overline{\psi}\) only in the combinations \(\overline{\psi} \psi\) and \(\overline{\psi} \gamma^{\mu} \partial_{\mu} \psi\), which are invariant under this transformation.
\[
    \mathcal{L} \to \mathcal{L}^{\prime} = \overline{\psi}'(i \gamma^{\mu} \partial_{\mu} - m) \psi' = e^{i q \theta} \overline{\psi}(i \gamma^{\mu} \partial_{\mu} - m) e^{-i q \theta} \psi = \overline{\psi}(i \gamma^{\mu} \partial_{\mu} - m) \psi = \mathcal{L}.
\]
This is a global \(\mathrm{U}(1)\) symmetry, thus associated to a conserved noether current and a one dimensional charge.
\begin{remark}
    If we were to consider a coordinate-dependent phase \(\theta(x)\) we would have a local \(\mathrm{U}(1)\) symmetry, which is the gauge symmetry of quantum electrodynamics (QED), as we will see in the next chapter and the starter point for introducing interactions between the Dirac field and the electromagnetic field. We thus need a global \(\mathrm{U}(1)\) symmetry to have a conserved charge distinguishing between particles and antiparticles, since the derivatives in the Lagrangian would break the local symmetry.
\end{remark}
Using noether's theorem we can compute the conserved current as
\[
    \begin{aligned}
        J^{\mu} & = \dots                                        \\
                & = \dots = q \overline{\psi} \gamma^{\mu} \psi,
    \end{aligned}
\]
which is a conserved vector, and if we differentiate it we get
\[
    \begin{aligned}
        \partial_{\mu} J^{\mu} & = q \left( (\partial_{\mu} \overline{\psi}) \gamma^{\mu} \psi + \overline{\psi} \gamma^{\mu} (\partial_{\mu} \psi) \right) \\
                               & = q \left( - i \overline{\psi} m \psi + i \overline{\psi} m \psi \right) = 0,
    \end{aligned}
\]
where if we use the equations of mortions
\[
    i gamma^{\mu} \partial_{\mu} \psi = m \psi, \quad \partial_{\mu} \overline{\psi} i \gamma^{\mu} = - m \overline{\psi},
\]
we see that the current is indeed conserved
\[
    \partial_{\mu} J^{\mu} = q \left( - i \overline{\psi} m \psi + i \overline{\psi} m \psi \right) = 0.
\]
The associated conserved charge is\TODO{compute it}
\[
    \begin{aligned}
        \hat{Q} & = \int \mathrm{d}^3 \mathbf{x} J^0 = \dots                                                                                                                                               \\
                & = q \int \mathrm{d}^3 \mathbf{x} \, \hat{\overline{\psi}}(x) \gamma^0 \hat{\psi}(x)                                                                                                      \\
                & = q \int \mathrm{d}^3 \mathbf{x} \, \hat{\psi}^{\dagger}(x) \hat{\psi}(x)                                                                                                                \\
                & = q \sum_{s} \int \frac{\mathrm{d}^3 p}{(2 \pi)^3} \left( \hat{b}_{\mathbf{p}}^{s\, \dagger} \hat{b}_{\mathbf{p}}^s - \hat{c}_{\mathbf{p}}^{s\, \dagger} \hat{c}_{\mathbf{p}}^s \right),
    \end{aligned}
\]
where we have used the field expansion in terms of creation and annihilation operators and the inner product relations for the spinors to simplify the expression.
Thus we can now compute the associated charge of one-particle states:\TODO{expand computation for both cases}
\[
    \begin{aligned}
        \hat{Q} \left( \hat{b}_{\mathbf{p}}^{s\, \dagger} \ket{0} \right) & = q \sum_{r} \int \frac{\mathrm{d}^3 \mathbf{q}}{(2 \pi)^3} \left(\hat{b}_{\mathbf{q}}^{r\, \dagger} \hat{b}_{\mathbf{q}}^r - \hat{c}_{\mathbf{q}}^{r\, \dagger} \hat{c}_{\mathbf{q}}^r\right)\hat{b}_{\mathbf{p}}^{s\, \dagger} \ket{0} = q \left( \hat{b}_{\mathbf{p}}^{s\, \dagger} \ket{0} \right), \\
        \hat{Q} \left( \hat{c}_{\mathbf{p}}^{s\, \dagger} \ket{0} \right) & = - q \left( \hat{c}_{\mathbf{p}}^{s\, \dagger} \ket{0} \right),
    \end{aligned}
\]
so that b-type states are particles with charge \(+q\) while c-type states are antiparticles with charge \(-q\), so that we can summarize the results of our quantization procedure as follows:
\begin{itemize}
    \item \(\hat{b}^{s\, \dagger}_{\mathbf{p}}\) create \textbf{particles} with energy \(E_{\mathbf{p}}\), spin \(s\), momentum \(\mathbf{p}\) and charge \(+q\);
    \item \(\hat{c}^{s\, \dagger}_{\mathbf{p}}\) create \textbf{antiparticles} with energy \(E_{\mathbf{p}}\), spin \(s\), momentum \(\mathbf{p}\) and charge \(-q\).
\end{itemize}

\paragraph{Spin-statistics relations.}
We have seen that to quantize the Dirac field we had to impose anticommutation relations between the field and its conjugate momentum, leading to anticommutation relations for creation and annihilation operators. This is related to the \textit{spin-statistics theorem}, which states that particles with integer spin (bosons) obey Bose-Einstein statistics and thus commutation relations, while particles with half-integer spin (fermions) obey Fermi-Dirac statistics and thus anticommutation relations.

We can see its effect by looking at two-particle states: for two fermions we have
\[
    \ket{\mathbf{p},\,s;\; \mathbf{q},\,r} = \hat{b}_{\mathbf{p}}^{s\, \dagger} \hat{b}_{\mathbf{q}}^{r\, \dagger} \ket{0} = - \hat{b}_{\mathbf{q}}^{r\, \dagger} \hat{b}_{\mathbf{p}}^{s\, \dagger} \ket{0} = - \ket{\mathbf{q},\,r;\; \mathbf{p},\,s},
\]
which is antisymmetric under the exchange of the two particles, exactly as required by the \textbf{Pauli exclusion principle} and described by Fermi-Dirac statistics. In particular, if we try to create two fermions in the same state we have indeed \(\ket{\mathbf{p},\,s;\; \mathbf{p},\,s} =- \ket{\mathbf{p},\,s;\; \mathbf{p},\,s} = 0\), showing that no two fermions can occupy the same quantum state simultaneously.

In the beginning, when Dirac was thinking about his equation, he was not aware of this spin-statistics relation, and he tried to interpret the negative energy solutions of his equation as physical states, leading to an unstable vacuum. He then proposed the \textit{Dirac sea} idea, where all negative energy states are filled in the vacuum, and only holes in this sea (i.e., absence of negative energy electrons) can be interpreted as positrons, thus solving the negative energy problem. This creative idea was later abandoned in favor of the quantum field theory approach we have seen, where antiparticles arise naturally from the quantization procedure and the imposition of anticommutation relations for fermionic fields.

\section{Propagators}

If we now move to the Heisenberg picture, we know that the time evolution of operators is given by
\[
    \hat{\psi} (x), \quad x = (t, \mathbf{x}), \quad \partial_{0} \hat{\psi}(x) = i \left[\hat{H},\, \hat{\psi}(x)\right],
\]
which is solved by the time evolution operator
\[
    \hat{\psi}(x) = \sum_{s} \int \frac{\mathrm{d}^3 \mathbf{p}}{(2 \pi)^3} \frac{1}{2\sqrt{E_{\mathbf{p}}}} \left( \hat{b}_{\mathbf{p}}^s u^s(\mathbf{p}) e^{- i p \cdot x} + \hat{c}_{\mathbf{p}}^{s\, \dagger} v^s(\mathbf{p}) e^{i p \cdot x} \right),
\]
and
\[
    \hat{\psi}^{\dagger}(x) = \sum_{s} \int \frac{\mathrm{d}^3 \mathbf{p}}{(2 \pi)^3} \frac{1}{2\sqrt{E_{\mathbf{p}}}} \left( \hat{b}_{\mathbf{p}}^{s\, \dagger} u^{s\, \dagger}(\mathbf{p}) e^{i p \cdot x} + \hat{c}_{\mathbf{p}}^s v^{s\, \dagger}(\mathbf{p}) e^{- i p \cdot x} \right).
\]

Now to ensure that there is no measurable effect outside the light cone, we have to impose that the anticommutator of the field at spacelike separated points vanishes: we define the \textbf{fermionic propagator} as
\[
    \begin{aligned}
        i S_{\alpha \beta}= \{\hat{\psi}_{\alpha}(x),\, \hat{\overline{\psi}}_{\beta}(y)\} \\
        i S(x-y) = \{\hat{\psi}(x),\, \hat{\overline{\psi}}(y)\}.
    \end{aligned}
\]
Now by substituting the field expansions into this expression and using the anticommutation relations for creation and annihilation operators, we find
\[
    \begin{aligned}
        i S(x-y) & = \sum_{r,s} \int \frac{\mathrm{d}^3 \mathbf{p}\mathrm{d}^3 \mathbf{q}}{(2 \pi)^6} \frac{1}{2 E_{\mathbf{p}}E_{\mathbf{q}}} \left( \{\hat{b}_{\mathbf{p}}^s,\, \hat{b}_{\mathbf{q}}^{r,\,\dagger}\} + \dots\right) \\
                 & = \dots
    \end{aligned}
\]
where we have used the anticommutation relations to simplify the expression, and inner (outer?) products of spinors to get
\[
    iS(x-y) = \int \frac{\mathrm{d}^3 \mathbf{p}}{(2 \pi)^3} \frac{1}{2 E_{\mathbf{p}}} \left( (\slashed{p} + m) e^{- i p \cdot (x-y)} + (\slashed{p} - m) e^{i p \cdot (x-y)} \right).
\]

Now recalling the scalar field correlators for KG theory
\[
    D(x-y) = \int \frac{\mathrm{d}^3 \mathbf{p}}{(2 \pi)^3} \frac{1}{2 E_{\mathbf{p}}} \left( e^{- i p \cdot (x-y)} \right),
\]
if we differenciate it with respect to \(x^{\mu}\) we get
\[
    \partial_{\mu} D(x-y) = \int \frac{\mathrm{d}^3 \mathbf{p}}{(2 \pi)^3} \frac{1}{2 E_{\mathbf{p}}}(- i p_{\mu})e^{- i p \cdot (x-y)},
\]
so that we can rewrite the fermionic propagator as
\[
    i S(x-y) = (i \gamma^{\mu} \partial_\mu^{(x)} + m) \left( D(x-y) - D(y-x) \right).
\]
This propagator satisfies the following properties:
\begin{itemize}
    \item For spacelike propagated points \((x-y)^2<0\):
          \[
              D(x-y) - D(y-x) = 0, \implies S(x-y) = 0,
          \]
          meaning that there is no measurable effect outside the light cone, preserving causality for the Dirac theory;
    \item For bosons the last point was ensured by commutation relations
          \[
              \left[\hat{\varphi}(x), \hat{\varphi}(y)\right] = 0 \quad \text{for} \quad (x-y)^2 < 0,
          \]
          while for fermions it is ensured by anticommutation relations:
          \[
              \{\hat{\psi}_{\alpha}(x), \hat{\psi}_{\beta}(y)\} = 0 \quad \text{for} \quad (x-y)^2 < 0;
          \]
    \item There is a problem: we want observables to commute at spacelike separations, while fermionic fields anticommute. The solution is that observables are constructed as bilinear combinations of fermionic fields, like the Hamiltonian density or the current density, which do commute at spacelike separations, thus preserving causality: the hamiltonian for example
          \[
              \hat{H} = \int \frac{\mathrm{d}^3 \mathbf{p}}{(2 \pi)^3} E_{\mathbf{p}} \left( \hat{b}_{\mathbf{p}}^{s\, \dagger} \hat{b}_{\mathbf{p}}^s + \hat{c}^{s\, \dagger}_{\mathbf{p}} \hat{c}^s_{\mathbf{p}} \right),
          \]
          where the two main terms are fermionic bilinears respecting anti-commutation relations, thus their product commutes at spacelike separations:
          \[
              \begin{aligned}
                  \hat{O}_1 = A_1 B_1, \quad \hat{O}_2 = A_2 B_2, \\
                  \hat{O}_1 \hat{O}_2 = A_1 B_1 A_2 B_2 = - A_1 A_2 B_1 B_2 = A_2 A_1 B_2 B_1 = \hat{O}_2 \hat{O}_1,
              \end{aligned}
          \]
          if each of the operators \(A_i, B_i\) anticommute with each other, like our fermionic fields.
\end{itemize}

The last thing we can check is that the fermionic propagator satisfies the Dirac equation:
\[
    i \gamma^{\mu} \partial_{\mu}^{(x)} S(x-y) - m S(x-y) =0,
\]
since we can write the propagator in terms of the scalar propagator as
\[
    \begin{aligned}
        \frac{1}{i}(i \gamma^{\mu} \partial_{\mu}^{(x)} - m)(i \gamma^{\mu} \partial_{\mu}^{(x)} + m) (D(x-y) - D(y-x)) & = \dots                                                                                                                                                                        \\
                                                                                                                        & = \frac{-1}{i} (\partial_{\mu}^{(x)} \partial^{\mu\,(x)} + m^2) (D(x-y) - D(y-x))                                                                                              \\
                                                                                                                        & = \frac{1}{i} \int \frac{\mathrm{d}^3 \mathbf{p}}{(2 \pi)^3} \frac{1}{2 E_{\mathbf{p}}} (p_{\mu}p^{\mu} - m^2) \left( e^{- i p \cdot (x-y)} - e^{i p \cdot (x-y)} \right) = 0,
    \end{aligned}
\]
where we have used the KG equation satisfied by the scalar propagator.